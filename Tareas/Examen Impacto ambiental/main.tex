% Template:     Informe LaTeX
% Documento:    Archivo principal
% Versión:      8.3.6 (23/08/2024)
% Codificación: UTF-8
%
% Autor: Pablo Pizarro R.
%        pablo@ppizarror.com
%
% Manual template: [https://latex.ppizarror.com/informe]
% Licencia MIT:    [https://opensource.org/licenses/MIT]

% CREACIÓN DEL DOCUMENTO
\documentclass[
	spanish, % Idioma: spanish, english, etc.
	letterpaper, oneside
]{article}

% INFORMACIÓN DEL DOCUMENTO
\def\documenttitle {Examen}
\def\documentsubtitle {}
\def\documentsubject {Líneas eléctrica en CC de alta tensión}

\def\documentauthor {Erik Saez}
\def\coursename {Impacto Ambiental y Social de Proyectos Eléctricos}
\def\coursecode {EL6046-1}

\def\universityname {Universidad de Chile}
\def\universityfaculty {Facultad de Ciencias Físicas y Matemáticas}
\def\universitydepartment {Departamento de Ingeniería Eléctrica}
\def\universitydepartmentimage {departamentos/die}
\def\universitydepartmentimagecfg {height=1.57cm}
\def\universitylocation {Santiago de Chile}

% INTEGRANTES, PROFESORES Y FECHAS
\def\authortable {
	\begin{tabular}{ll}
		Integrantes:
		& \begin{tabular}[t]{l}
			Erik Saez
		\end{tabular} \\
		Profesor:
		& \begin{tabular}[t]{l}
			Aldo Di Biase F. 
		\end{tabular} \\

		\multicolumn{2}{l}{Fecha de entrega: 10 de diciembre de 2024} \\
		\multicolumn{2}{l}{\universitylocation}
	\end{tabular}
}

% IMPORTACIÓN DEL TEMPLATE
\input{template}

% INICIO DE PÁGINAS
\begin{document}

% PORTADA
\templatePortrait

% CONFIGURACIÓN DE PÁGINA Y ENCABEZADOS
\templatePagecfg

% RESUMEN O ABSTRACT
\begin{abstractd}
En el presente informe se analizan los efectos de los campos electromagnéticos (CEM) generados por líneas de corriente continua de alta tensión (HVDC) en dispositivos médicos implantables, como marcapasos, bombas de insulina y desfibriladores automáticos implantables (DAI). Se presentan estudios que investigan cómo los CEM pueden interferir con el funcionamiento de estos dispositivos, destacando los riesgos potenciales y las recomendaciones para minimizar dichos riesgos.\\\\
Además, se realiza un análisis crítico de los estudios revisados, señalando las limitaciones en términos de tamaño de muestra, condiciones experimentales controladas y variabilidad en el diseño y tecnología de los dispositivos. Se destaca la necesidad de investigaciones adicionales con tamaños de muestra más grandes, condiciones experimentales más representativas del mundo real y una mayor diversidad de dispositivos y participantes para obtener conclusiones más concluyentes y generalizables. También se subraya la importancia de considerar la evolución tecnológica de los dispositivos médicos en estudios futuros para garantizar que las recomendaciones de seguridad se basen en la tecnología más reciente.\\\\
Finalmente, se discuten las regulaciones y estándares internacionales que establecen límites de exposición a los CEM para proteger tanto a los pacientes con dispositivos médicos implantables como a los trabajadores que operan cerca de líneas HVDC. 
\end{abstractd}

% TABLA DE CONTENIDOS - ÍNDICE
\templateIndex

% CONFIGURACIONES FINALES
\templateFinalcfg

% ======================= INICIO DEL DOCUMENTO =======================

% Template:     Informe LaTeX
% Documento:    Archivo de ejemplo
% Versión:      8.3.6 (23/08/2024)
% Codificación: UTF-8
%
% Autor: Pablo Pizarro R.
%        pablo@ppizarror.com
%
% Manual template: [https://latex.ppizarror.com/informe]
% Licencia MIT:    [https://opensource.org/licenses/MIT]

% ------------------------------------------------------------------------------
% NUEVA SECCIÓN
% ------------------------------------------------------------------------------
% Las secciones se inician con \section, si se quiere una sección sin número se
% pueden usar las funciones \sectionanum (sección sin número) o la función
% \sectionanumnoi para crear el mismo título sin numerar y sin aparecer en el índice
\section{Lineas de transmision}
Una línea de transmisión microstrip es un tipo de línea de transmisión utilizada ampliamente en circuitos de alta frecuencia, especialmente en aplicaciones de microondas. Está formada por una cinta metálica (normalmente cobre) que se coloca sobre un sustrato dieléctrico, con un plano de masa debajo. El material dieléctrico entre la cinta y el plano de masa influye en la velocidad de propagación y las características de la señal transmitida, la representacion sigue el modelo clasico de lineas de tranmision.\\\\
El funcionamiento de una línea microstrip se basa en guiar las ondas electromagnéticas a través de la cinta metálica, con parte del campo eléctrico propagándose a través del dieléctrico y otra parte en el aire. A diferencia de otras líneas de transmisión, como las líneas coaxiales, el microstrip es más fácil de integrar en circuitos de microondas y RF (radiofrecuencia) porque puede fabricarse directamente en placas de circuito impreso.\\\\
La idea detrás de estas líneas es transmitir señales de alta frecuencia con bajas pérdidas y buena eficiencia, lo que es esencial en dispositivos como transmisores, receptores y otros sistemas de comunicación.Las antenas patch están directamente relacionadas con las líneas de transmisión microstrip porque suelen utilizarse en el mismo tipo de tecnología. En una antena patch, el parche metálico que actúa como elemento radiador se alimenta mediante una línea de transmisión microstrip. Este tipo de alimentación es ideal porque permite una integración directa de la antena en el mismo sustrato que el circuito, reduciendo el tamaño total del dispositivo y minimizando pérdidas en la conexión entre la antena y el resto del circuito. % Ejemplo, se puede borrar

% FIN DEL DOCUMENTO
\end{document}