\documentclass[
  11pt,
  letterpaper,
   addpoints,
   answers
  ]{exam}

\usepackage{../exercise-preamble}

\begin{document}

\noindent
\begin{minipage}{0.47\textwidth}
\includegraphics[width=\textwidth]{../fcfm_die}
\end{minipage}
\begin{minipage}{0.53\textwidth}
\begin{center} 
\large\textbf{Fundamentos de control de sistemas} (EL4111-1) \\
\large\textbf{Clase auxiliar 1} \\
\small Prof.~Roberto Cardenas Dobson\\
\small Prof.~Aux.~Osvaldo Jimenez - Erik Sáez\\
\small Ayudantes.~Simon Arenas- Juan Pablo Baez - Francisco Garces - Sofia Ibarra\\
\end{center}
\end{minipage}

\vspace{0.5cm}
\noindent
\vspace{.85cm}

\begin{questions}
    %%%%%%%%%%%%%%%%%%%%%%%%%%%
    \question 
    \begin{enumerate}
        \item Encuentre la funcion de transferencia del siguiente diagrama de bloques:
        \begin{figure}[h!]
            \centering
            \includegraphics[width=0.5\textwidth]{Auxiliar_1_1}
            \caption{Diagrama de bloques}    
        \end{figure}
        \item Demuestre que los ceros de un sensor o filtro afectan la estabilidad del sistema. Puede utilizar el sistema anterior para plantear su desarrollo:
        \item emuestre que para un sistema SISO, los ceros de lazo cerrado son iguales a los ceros que se encuentran en lazo directo (ceros de lazo abierto) más los polos que se encuentran
        en el lazo de retroalimentación.
    \end{enumerate}
    %%%%%%%%%%%%%%%%%%%%%%%%%%%
    \begin{solution}
        \subsection*{Resolución 1.1}
        En primera instancia, se debe reconocer que la función de transferencia de cualquier sistema de control está dada por la razón entre la salida y la entrada del mismo. En este caso, la función que se debe encontrar debe relacionar la salida y la entrada de la forma $\frac{C(s)}{R(s)}$.A partir del diagrama de bloques se puede plantear el siguiente sistema de ecuaciones:
        \begin{align}
            E(s)&=R(s)-B(s) \label{eq:1} \\ 
            C(s)&=E(s)G(s) \label{eq:2} \\
            B(s)&=C(s)H(s) \label{eq:3}
        \end{align}
        Reemplazando (\ref{eq:1}) y (\ref{eq:3}) en (\ref{eq:2}) podemos desarrollar lo siguiente:
        \begin{align}
            C(s)&=(R(s)-B(s))G(s) \nonumber \\
            C(s)&=(R(s)-C(s)H(s))G(s) \nonumber \\
            C(s)&=R(s)G(s)-C(s)H(s)G(s) \nonumber \\
            C(s)+C(s)H(s)G(s)&=R(s)G(s) \nonumber \\
            C(s)(1+G(s)H(s))&=R(s)G(s) \nonumber \\
            \frac{C(s)}{R(s)}&=\frac{G(s)}{1+G(s)H(s)} \label{eq:final} 
        \end{align}
        
        Finalmente, la función de transferencia equivalente del diagrama está dada por la expresión (\ref{eq:final})
\subsection*{Resolucion 1.2}
Tal como se dijo en la clase auxiliar, las funciones de transferencia se pueden expresar como una división de polinomios en el dominio de Laplace. Por lo tanto, una función de trasnferencia $X(s)$ arbitraria se puede expresar según (\ref{eq:b})
\begin{equation}
    X(s)=\frac{s^m+\alpha_{m-1} s^{m-1}+...+\alpha_0}{s^n + \beta_{n-1}s^{n-1}+...+\beta_0} \quad m\leq n \label{eq:b}
\end{equation}

Notar que las funciones de transferencia nunca pueden ser \textbf{IMPROPIAS}, puesto que al descomponer su respuesta en el dominio del tiempo nos resulta un sistema físico que no representa una situación real. Por lo tanto, cuando en un futuro diseñen constroladores, siempre verifiquen que la función de transferencia sea \textbf{PROPIA} ($m\leq n$) o \textbf{BIPROPIA} ($n=m$).\\

Ahora bien, considerando lo descrito en (\ref{eq:b}), nosotros podemos generalizar los polinomios del numerador y del denominador. Lo anterior lo hacemos utilizando la definición de polos y ceros. Los polos son las raíces del denominador, por lo que multiplicando las raíces podemos recuperar el polinomio original. Esto se puede realizar de manera análoga para el numerador. De esta forma, las funciones $G(s)$ y $H(s)$ se pueden expresar de la siguiente forma:
\begin{align}
    G(s)&=\frac{\Pi (s+z^{G})}{\Pi (s+p^{G})} \label{eq:g} \\
    H(s)&=\frac{\Pi (s+z^{H})}{\Pi (s+p^{H})} \label{eq:p}
\end{align}
Para poder demostrar lo que se pide, debemos probar que los ceros de la función $H(s)$ aparecen de alguna forma en los polos de la expresión a lazo cerrado del sistema. Esto se debe a que, para probar que "algo" influye en la estabilidad, la mayoría de las veces se analizan las raíces de la ecuación característica $1+G(s)H(s)$ (que es justamente el denominador de la expresión (\ref{eq:final})). Por lo tanto, basta con reemplazar las ecuaciones (\ref{eq:g}) y (\ref{eq:p}) en (\ref{eq:final}):
\begin{align}
    \frac{C(s)}{R(s)} &= \frac{\frac{\prod (s+z^{G})}{\prod (s+p^{G})}}{1 + \frac{\prod (s+z^{G})}{\prod (s+p^{G})}\frac{\prod (s+z^{H})}{\prod (s+p^{H})}} \nonumber \\
    &= \frac{\prod (s+z^{G}) \prod (s+p^{H})}{\prod (s+p^{G}) + \prod (s+z^{G}) \prod (s+z^{H})} \label{eq:b_final}
\end{align}
De la expresión (\ref{eq:b_final}) se puede ver que en el denominador se encuentran los términos asociados a los ceros del sensor $(s+z^H)$. Por lo tanto, dado que los ceros aparecen en el denominador, estos afectan a la ecuación caraterística y por ende a la estabilidad.

\subsection*{Resolucion 1.3}
Para responder esta pregunta, basta notar la expresión descrita en (\ref{eq:b_final}). De esta ecuación se puede observar que su numerador está compuesto por los términos $\Pi (s+z^{G}) \Pi (s+p^{H})$, los cuales corresponden a los ceros de la función $G(s)$ y a los polos de la función $H(s)$. Dado que en el numerador de la función de lazo cerrado se encuentran los polos de lazo cerrado, entonces queda demostrado lo que se pide.
    \end{solution}
    %%%%%%%%%%%%%%%%%%%%%%%%%%%
    \question Encuentre el LGR del siguiente sistema:
    \begin{figure}[h!]
        \centering
        \includegraphics[width=0.5\textwidth]{Auxiliar_1_2}
    \end{figure}
    \begin{enumerate}
        \item ¿Es posible saber cuántos polos de lazo cerrado hay sólo mirando el LGR?
        \item ¿Es posible saber dónde estarán ubicados los polos de lazo cerrado?
        \item ¿Cuáles son los polos dominantes del sistema?
    \end{enumerate}
%%%%%%%%%%%%%%%%%%%%%%%%%%%
\begin{solution}
\subsection*{Resolucion 2.1}
    Se busca obtener el lugar geometrico de la raiz (LGR), por lo que se debera tener varios aspectos en cuenta.En primera instancia, se debe reconocer que el LGR es una representación gráfica de la variación de los polos de lazo cerrado a medida que se varía un parámetro del sistema. En este caso, el parámetro que se varía es el valor de $K$. Por lo tanto, al variar $K$ se obtendrán distintas posiciones de los polos de lazo cerrado. De la primera pregunta obtuvimos que la funcion de transferencia vendra dada por:
    \begin{align}
        \frac{C(s)}{R(s) }&= \frac{G(s)}{1+G(s)H(s)}
    \end{align} 
    Donde G(s) corresponde a la pondearcion de todos los elementos del lazo directo, es decir que si existira un controlador $G_{c}$ y la planta $G_{p}$, luego $G(s) = G_{c} \cdot G_{p}$ , esto se obtiene directamente del desarrollo de bloques. Debemos recordar que la funcion de transferencia es una propiedad propia del sistema que dependera de variables de estados y no de las entradas (Recordartorio de su curso de Dinamicos). Se debe tener un gran cuidado entre la la diferencia de los polos de lazo abierto ( $G(s) \cdot H(s)$ y de lazo cerrado ( $1+G(s) \cdot H(s)$),debido a que este ultimo da cuenta del lazo de retroalimentacion.Como se menciono anteriormente el LGR nos entrega una representacion visual del movimiento de los polos \textbf{de lazo cerrado} de la funcion de transferencia vista anteriromente. Por lo tanto tendremos que esos polos deberan cumplir que:
    \begin{align}
        1+G(s)H(s) &= 0\\
        G(s)H(s) = -1
    \end{align}
    Por lo que para que se cumpla dicha condicion se deben cumplir dos criterios
    \begin{itemize}
        \item \textbf{Condicion de modulo:} Se debe cumplir que $|G(s)H(s)| = 1$ 
        \item \textbf{Condicion de angulo:} Se debe cumplir que $\angle G(s)H(s) = \pm 180^{\circ} + n360^{\circ}$ 
    \end{itemize}
Por lo que todos los puntos $s = \sigma j\theta$ que cumplan ambas condiciones perteneceran al LGR y seran los polos de lazo cerrados de su sistema. Comenzamos identificando los polos y ceros de la funcion de transferencia de lazo abierto (Que representaran nuestros puntos de partida) , dados por :
\begin{align}
    G(s)H(S)= \frac{1}{s} \frac{(s+2)}{(s^{2} +7s +12)} = \frac{1}{s} \frac{(s+2)}{(s+3)(s+4)}
\end{align}
(Consideramos que $H(s)=1$, dado que no tenemos informacion de esta funcion de transferencia), con lo que identificamos que los polos de lazo abierto son $s=0$ , $s=-3$ y $s=-4$ y los ceros son $s=-2$. Por lo que se posiciona en el LGR

%%AÑADIR IMAGEN DEL LGR

Luego se debe encontrar las zonas en el eje real que pertenecen al LGR, esto se puede obtener de dos forams
\end{solution}
%%%%%%%%%%%%%%%%%%%%%%%%%%%

\end{questions}
\newpage
%%%%%%%%%%%%%%%%%%%%%%%%%%%

\end{document}