\documentclass[
  11pt,
  letterpaper,
   addpoints,
   %answers
  ]{exam}

\usepackage{../exercise-preamble}
\usepackage{float}
\usepackage[most]{tcolorbox}

\begin{document}

\noindent
\begin{minipage}{0.47\textwidth}
\includegraphics[width=\textwidth]{../fcfm_die}
\end{minipage}
\begin{minipage}{0.53\textwidth}
\begin{center} 
\large\textbf{Circuitos Eléctricos Analógicos} (EL3202) \\
\large\textbf{Control 1} \\
\normalsize Prof Patricio Mendoza \\
\normalsize Prof Aux  Renato Planas , Erik Sáez Aravena
\end{center}
\end{minipage}

\begin{tcolorbox}[colback=gray!10!white,colframe=black!80!white,title=Instrucciones]
Dispone de 1 hora y media para contestar el control. No se permite el uso de calculadoras ni teléfonos. Recuerde escribir su nombre y RUT en todas las hojas. La sospecha de copia será sancionada.
\end{tcolorbox}

\begin{questions}
  %%%%%%%%%%%%%%%%%%%%%%%%%%%
  \question 
  Considere el circuito mostrado en la Figura \ref{fig:circuito_p676} con parámetros de transistor $\beta = 120$ y $V_A = \infty$. 
  
  \begin{figure}[H]
    \centering
    \includegraphics[width=0.8\textwidth]{Control_03_01}
    \caption{Circuito del problema 6.76}
    \label{fig:circuito_p676}
  \end{figure}
  
  \begin{parts}
    \part Determine los parámetros de pequeña señal $g_m$, $r_\pi$, y $r_o$ para ambos transistores.
    \begin{solution}
      Se buscan determinar los parámetros de pequeña señal para ambos transistores. Primero, se debe encontrar el punto de operación DC para cada transistor. Primeramente debemos dejar en Circuito abierto los condensadores.
      \begin{itemize}
        \item \textbf{Analisis Transistor Q1:}
        Tenemos que para el analisis en DC al dejar en abierto los condensadores, luego tenemos que el modelo de Thevenin equivalente sera:
        \begin{align}
          V_{TH} &= V_{CC} \cdot \frac{R_{2}}{R_{1} + R_{2}} = 12V \cdot \frac{12.7k\Omega}{12.7k\Omega + 67.3k\Omega} =  1.905[V] \\
          R_{TH} &= R_{1} \parallel R_{2} = 67.3k[\Omega] \parallel 12.7k[\Omega] = 10.68k[\Omega] 
        \end{align}
      De esta manera tendremos luego que la malla par vendra dada por:
     \begin{align}
      -V_{TH} + 0.7 + I_B R_{TH} + I_E R_E = 0 \\
      -V_[TH] + 0.7 + I_B R_{TH} + I_B(\beta + 1) R_E = 0 \\
      I_B = \frac{V_{TH} - 0.7}{R_{TH} + (\beta + 1) R_E} = \frac{1.905 - 0.7}{10.68k\Omega + (120 + 1) \cdot 2k\Omega} = 0.00477[mA]
     \end{align}
     De esta manera tendremos que $I_{C1}$ sera:
      \begin{align}
        I_{C1} = \beta \cdot I_B = 120 \cdot 0.00477[mA] = 0.572[mA]
      \end{align}
      Con estos datos, ya podemos realizar el analisis para pequeña señal, donde tenemos que:
      \begin{align}
        g_{m1} &= \frac{I_{C1}}{V_T} = \frac{0.572[mA]}{25[mV]} = 22m[A/V] \\
        r_{\pi 1} &= \frac{\beta}{g_{m1}} = \frac{120 \cdot 0.026}{0.572} = 5.45[k\Omega] \\
        r_{o1} &= \frac{V_A}{I_{C1}} = \frac{\infty}{0.572[mA]} = \infty[\Omega]
      \end{align}
      \item \textbf{Analisis Transistor Q2:}
      Para el transistor Q2, tenemos un analisis, similar dado por:
      \begin{align}
        V_{TH2} = V_{CC} \cdot \frac{R_4}{R_3 + R_4} = 12V \cdot \frac{45k\Omega}{45k\Omega + 15k\Omega} = 9[V] \\
      \end{align}
      Mientras que la resistencia de Thevenin sera:
      \begin{align}
        R_{TH2} = R_3 \parallel R_4 = 15k\Omega \parallel 45k\Omega = 11.25k\Omega
      \end{align}
      De esta manera tenemos que la malla para el transistor Q2 sera:
      \begin{align}
        -V_{TH2} + R_{TH2}I_{B2} + 0.7 + R_{E2}I_{E2} &= 0 \\
        -V_{TH2} + R_{TH2}I_{B2} + 0.7 + R_{E2}(\beta + 1)I_{B2} &= 0 \\
        I_{B2} &= \frac{V_{TH2} - 0.7}{R_{TH2} + (\beta + 1) R_{E2}}\\ 
        &= \frac{9 - 0.7}{11.25k\Omega + (120 + 1) \cdot 1.6k\Omega} = 0.0405[mA]
      \end{align}
      De esta manera tendremos que: 
      \begin{align}
        I_{C2} &= \beta \cdot I_{B2} = 120 \cdot 0.0405[mA] = 4.86[mA]
      \end{align}
      Con estos datos, ya podemos realizar el analisis para pequeña señal, donde tenemos que:
      \begin{align}
        g_{m2} &= \frac{I_{C2}}{V_T} = \frac{4.86[mA]}{0.026[V]} = 187m[A/V] \\
        r_{\pi 2} &= \frac{\beta}{g_{m2}} = \frac{120 \cdot 0.026}{4.86} = 0.642[k\Omega] \\
        r_{o2} &= \frac{V_A}{I_{C2}} = \frac{\infty}{4.86[mA]} = \infty[\Omega]
      \end{align}
    \end{itemize}
    Con lo que finalmente se obtienen todos los valores buscados.
  \end{solution}

  \part Determine la ganancia de voltaje de pequeña señal total $A_v = v_o/v_s$.
    
    \part Determine la resistencia de entrada $R_{is}$ y la resistencia de salida $R_o$.
  
  \end{parts}
  
\end{questions}
\end{document}