\section{Pauta P1}
\begin{enumerate}
    \item \textit{\textbf{(0.5 Puntos)} Establezca al menos tres hipótesis simplificatorias que crea pertinente para analizar su modelo.}\\
    Existe flexibilidad en cuanto a lo que se espera de esta pregunta por ende, puede haber más de una respuesta correcta. A continuación se presenta una lista de hipótesis posibles:
    \begin{itemize}
        \item El roce con el aire es despreciable.
        \item No existe otra resistencia por el cable más que la explicitada en la figura correspondiente a $R$.
        \item La bola de acera es una masa puntual, es decir, no posee dimensiones.
        \item El movimiento de la bola es solo en una dimensión.
        \item Tanto la estructura que sujeta la inductancia como la bola de acero tienen permeabilidad magnética infinita, es decir, no existen pérdidas magnéticas.
        \item Si $y$ se define tal cual como aparece en la figura del problema, entonces, $y>0$.
    \end{itemize}
    \item \textit{\textbf{(0.5 Puntos)} Indique las variables que corresponden a la/s entrada/s, salida/s y variable/s de estado del sistema. Luego, Replanteé el modelo como un sistema matricial de la forma $\dot{\Vec{X}}(t)=\Vec{F}(\Vec{X}(t), u(t))$
    con su respectiva salida $\Vec{Y}(t) = C\cdot\Vec{X}(t)$
    donde $\Vec{X}(t)$ representa el vector de estados, $C$ la matriz de salida y $u(t)$ la/s entrada/s al sistema.}\\
    
    Del enunciado, se puede inferir lo siguiente para la entrada, salida y en consecuencia, la variable de estado: "\textit{El objetivo del sistema es controlar
la posición de la bola} [salida $y_s(t) = y(t)$] \textit{ajustando la corriente en el electroimán mediante el voltaje de entrada
$e(t)$} [entrada $u(t) = e(t)$]". En resumen:
    \begin{itemize}
    \item Entrada: $u(t) = e(t)$
    \item Salida: $y(t)$
    \item Variables de estado: $\Vec{X}(t) = \begin{bmatrix} y(t) \\ \dot{y}(t) \\ i(t) \end{bmatrix}$
\end{itemize}

Luego, el modelo matricial se modela la dinámica del sistema queda expresado por:
\[
\dot{\Vec{X}}(t) = 
\begin{bmatrix}
\dot{y}(t) \\
\ddot{y}(t) \\
\dot{i}(t)
\end{bmatrix}
=
\begin{bmatrix}
\dot{y}(t) \\
 g - \frac{1}{M}\frac{i^2}{y} \\
-\frac{R}{L}i + \frac{1}{L}e(t)
\end{bmatrix}
\]
Mientras que la salida se expresa como:
\[
\Vec{Y}(t) = C \cdot \Vec{X}(t) = \begin{bmatrix} 1 & 0 & 0 \end{bmatrix} \cdot \Vec{X}(t) = y(t)
\]
\item \textit{\textbf{(1.5 Puntos)} Encuentre el punto de operación ($\Vec{X}_{op}$, $u_{op}$) (donde $u_{op}$ es una entrada particular a sistema) tal que la bola de acero permanezca suspendida en el aire en la posición $y_0$.}\\

Como la masa se encuentra quieta en una posición $y(t>t_0)=y_0$, entonces, $\dot{y}(t>t_0)=0$ y por ende $\ddot{y}(t>t_0)=0$. Entonces, del sistema no lineal se tiene:

\[
\dot{\Vec{X}}(t>t_0) = \begin{bmatrix} \dot{y}(t>t_0) \\ \ddot{y}(t>t_0) \\ \dot{i}(t>t_0) \end{bmatrix} = \begin{bmatrix} 0 \\ 0 \\ \dot{i}(t>t_0) \end{bmatrix} = \begin{bmatrix} 0 \\ g - \frac{1}{M} \frac{i^2}{y_0} \\ -\frac{R}{L} i(t>t_0) + \frac{1}{L}\cdot e(t>t_0) \end{bmatrix}
\]

Se puede rescatar la segunda ecuación \(
g - \frac{1}{M} \frac{i^2}{y_0} = 0
\), se obtiene:
    \[
g = \frac{1}{M} \frac{i^2}{y_0} \quad \Rightarrow \quad i^2 = M g y_0
\quad \Rightarrow \quad i(t>t_0) = \pm \sqrt{M g y_0}
\]
Tal y como dice el enunciado, solo debemos considerar la corriente positiva, $i(t>t_0) = \sqrt{M g y_0}$.\\
Como la corriente es constante, entonces, $\dot{i}(t>t_0) = 0$, es decir que de la tercera ecuación del sistema \(\dot{i}(t>t_0) = -\frac{R}{L} i(t>t_0) + \frac{1}{L}\cdot e(t>t_0)=0\) se obtiene:
\[
e(t>t_0) = R\cdot \sqrt{Mgy_0}
\]

Por lo tanto, el vector $\Vec{X}_{op}$ está dado por:
$$\Vec{X}_{op} = \begin{bmatrix} y_0 \\ 0 \\ \sqrt{Mgy_0} \end{bmatrix}$$

Mientras que la entrada $e$ está dada por:
$$e(t>t_0)= R\sqrt{Mgy_0}$$
\item \textit{\textbf{(1.5 Puntos)} Demuestre \textbf{de forma clara y ordenada} que el sistema linealizado en torno al punto de operación está dado por la siguiente expresión del enunciado.}\\

Considerando la función $F(\Vec{X}(t), e(t))$ tal que:
\[
\dot{\Vec{X}}(t) = 
\begin{bmatrix}
\dot{y}(t) \\
\ddot{y}(t) \\
\dot{i}(t)
\end{bmatrix}
= F(\Vec{X}(t), e(t)) =
\begin{bmatrix}
\dot{y}(t) \\
 g - \frac{1}{M}\frac{i^2}{y} \\
-\frac{R}{L}i + \frac{1}{L}e(t)
\end{bmatrix}
\]

Se puede calcular $A$ como:

\[
A = 
\begin{bmatrix}
\frac{d F(\Vec{X}(t), e(t))}{dy} & \frac{d F(\Vec{X}(t), e(t))}{d\dot{y}} & \frac{d F(\Vec{X}(t), e(t))}{d\dot{i}}
\end{bmatrix}_{(\Vec{X}_{op}, e)}
\]

Tal que:
\begin{itemize}
    \item Derivada con respecto a \(y\):\\
    \[
\frac{\partial F(\Vec{X}(t), e(t))}{\partial y} = 
\begin{bmatrix}
0 \\
\frac{\partial}{\partial y} \left( g - \frac{1}{M} \frac{i^2}{y} \right) \\
0
\end{bmatrix}
=
\begin{bmatrix}
0 \\
\frac{i^2}{M y^2} \\
0
\end{bmatrix}
\]
\item Derivada con respecto a \(\dot{y}\):\\
\[
\frac{\partial F(\Vec{X}(t), e(t))}{\partial \dot{y}} = 
\begin{bmatrix}
1 \\
0 \\
0
\end{bmatrix}
\]
\item Derivada con respecto a \(i\):\\
\[
\frac{\partial F(\Vec{X}(t), e(t))}{\partial i} = 
\begin{bmatrix}
0 \\
\frac{\partial}{\partial i} \left( g - \frac{1}{M} \frac{i^2}{y} \right) \\
-\frac{R}{L}
\end{bmatrix}
=
\begin{bmatrix}
0 \\
-\frac{2i}{M y} \\
-\frac{R}{L}
\end{bmatrix}
\]
\end{itemize}
Luego,
\[
A=
\begin{bmatrix}
0 & 1 & 0 \\
\frac{i^2}{M y^2} & 0 & -\frac{2i}{M y} \\
0 & 0 & -\frac{R}{L}
\end{bmatrix}_{(\Vec{X}_{op}, e)}
\]
Evaluando en el punto de operación,
\[
A=
\begin{bmatrix}
0 & 1 & 0 \\
\frac{(\sqrt{Mgy_0})^2}{M y_0^2} & 0 & -\frac{2\sqrt{Mgy_0}}{M y_0} \\
0 & 0 & -\frac{R}{L}
\end{bmatrix}
\]
Reordenando se llega a lo solicitado:
\[
A=
\begin{bmatrix}
0 & 1 & 0 \\
\frac{g}{y_0} & 0 & -2\sqrt{\frac{g}{My_0}} \\
0 & 0 & -\frac{R}{L}
\end{bmatrix}
\]
Por otra parte, para la matriz B, se debe calcular:

\[
B = 
\begin{bmatrix}
\frac{d F(\Vec{X}(t), e(t))}{de}
\end{bmatrix}_{(\Vec{X}_{op}, e)}
\]
Luego, es fácil ver que:
\[
B=
\begin{bmatrix}
0 \\
0 \\
\frac{1}{L}
\end{bmatrix}
\]

\item \textit{\textbf{(2 Puntos)} Obtenga la MTE del sistema linealizado en torno al punto de equilibrio y determine la estabilidad en torno al punto de operación. Además escriba de manera explicita la solución en base a la RENC y RESC considerando condiciones iniciales arbitrarias.}\\

Para obtener la MTE consideremos diagonalizar la matriz A. Para ello debemos encontrar los valores y vectores propios.

Los valores propios \( \lambda \) se encuentran resolviendo el determinante del polinomio característico \( \det(A - \lambda I) = 0 \), donde \( I \) es la matriz identidad de \(3 \times 3\). Esto nos lleva a resolver:

\[
\det\left(
\begin{bmatrix}
0 & 1 & 0 \\
\frac{g}{y_0} & 0 & -2\sqrt{\frac{g}{M y_0}} \\
0 & 0 & -\frac{R}{L}
\end{bmatrix}
- \lambda
\begin{bmatrix}
1 & 0 & 0 \\
0 & 1 & 0 \\
0 & 0 & 1
\end{bmatrix}
\right) = 0
\]

Sustituyendo y simplificando:

\[
\det
\begin{bmatrix}
-\lambda & 1 & 0 \\
\frac{g}{y_0} & -\lambda & -2\sqrt{\frac{g}{M y_0}} \\
0 & 0 & -\frac{R}{L} - \lambda
\end{bmatrix} = 0
\]

Expandiendo el determinante:

\[
\left( -\lambda \right) \left[ \left( -\lambda \right) \left( -\frac{R}{L} - \lambda \right) \right] + (-1)\left[ \frac{g}{y_0} \left( -\frac{R}{L} - \lambda \right) \right] = 0
\]

Factorizando convenientemente:

\[
 \left( -\frac{R}{L} - \lambda \right) \left[ \lambda^2 - \frac{g}{y_0} \right] = 0
\]

Esto nos lleva a los valores propios \( \lambda \):

\[
\lambda_1 = -\frac{R}{L}, \quad \lambda_{2} = -\frac{g}{y_0}, \quad \lambda_{3} =\frac{g}{y_0}
\]

En cuanto a los vectores propios:
\begin{itemize}
    \item Para \( \lambda_1 = -\frac{R}{L} \), resolvemos el sistema:

\[
(A - \lambda_1 I) \Vec{v_1} = 0
\]

donde la matriz \( A \) es:

\[
A = 
\begin{bmatrix}
0 & 1 & 0 \\
\frac{g}{y_0} & 0 & -2\sqrt{\frac{g}{M y_0}} \\
0 & 0 & -\frac{R}{L}
\end{bmatrix}
\]

Por lo tanto, necesitamos resolver el siguiente sistema lineal:

\[
\left( 
\begin{bmatrix}
0 & 1 & 0 \\
\frac{g}{y_0} & 0 & -2\sqrt{\frac{g}{M y_0}} \\
0 & 0 & -\frac{R}{L}
\end{bmatrix}
- 
\begin{bmatrix}
-\frac{R}{L} & 0 & 0 \\
0 & -\frac{R}{L} & 0 \\
0 & 0 & -\frac{R}{L}
\end{bmatrix}
\right)
\begin{bmatrix}
v_{11} \\
v_{12} \\
v_{13}
\end{bmatrix}
= 0
\]

Esto nos lleva a la siguiente matriz:

\[
\begin{bmatrix}
\frac{R}{L} & 1 & 0 \\
\frac{g}{y_0} & \frac{R}{L} & -2\sqrt{\frac{g}{M y_0}} \\
0 & 0 & 0
\end{bmatrix}
\begin{bmatrix}
v_{11} \\
v_{12} \\
v_{13}
\end{bmatrix}
= 0
\]

Lo que genera el siguiente sistema de ecuaciones:

1. \( \frac{R}{L}v_{11} + v_{12} = 0 \), que nos da:
   \[
   v_{12} = -\frac{R}{L}v_{11}
   \]
   
2. \( \frac{g}{y_0}v_{11} + \frac{R}{L}v_{12} - 2\sqrt{\frac{g}{M y_0}}v_{13} = 0 \). Sustituyendo \( v_{12} = -\frac{R}{L}v_{11} \), obtenemos:
   \[
   \frac{g}{y_0}v_{11} - \frac{R^2}{L^2}v_{11} - 2\sqrt{\frac{g}{M y_0}}v_{13} = 0
   \]
   Resolviendo para \( v_{13} \), tenemos:
   \[
   v_{13} = \frac{1}{2\sqrt{\frac{g}{M y_0}}} \left( \frac{g}{y_0} - \frac{R^2}{L^2} \right) v_{11}
   \]

Por lo tanto, el vector propio asociado a \( \lambda_1 = -\frac{R}{L} \) es:

\[
\Vec{v_1} = 
\begin{bmatrix}
1\\
-\frac{R}{L} \\
\frac{1}{2\sqrt{\frac{g}{M y_0}}} \left( \frac{g}{y_0} - \frac{R^2}{L^2} \right)
\end{bmatrix}
\]

\item Para el valor propio \( \lambda_2 = -\frac{g}{y_0} \), necesitamos resolver el sistema:

\[
(A - \lambda_2 I) \Vec{v_2} = 0
\]

donde la matriz \( A \) es:

\[
A = 
\begin{bmatrix}
0 & 1 & 0 \\
\frac{g}{y_0} & 0 & -2\sqrt{\frac{g}{M y_0}} \\
0 & 0 & -\frac{R}{L}
\end{bmatrix}
\]

Sustituyendo \( \lambda_2 = -\frac{g}{y_0} \), el sistema se vuelve:

\[
\left( 
\begin{bmatrix}
0 & 1 & 0 \\
\frac{g}{y_0} & 0 & -2\sqrt{\frac{g}{M y_0}} \\
0 & 0 & -\frac{R}{L}
\end{bmatrix}
- 
\begin{bmatrix}
-\frac{g}{y_0} & 0 & 0 \\
0 & -\frac{g}{y_0} & 0 \\
0 & 0 & -\frac{g}{y_0}
\end{bmatrix}
\right)
\begin{bmatrix}
v_{21} \\
v_{22} \\
v_{23}
\end{bmatrix}
= 0
\]

Esto nos lleva a la siguiente matriz:

\[
\begin{bmatrix}
\frac{g}{y_0} & 1 & 0 \\
\frac{g}{y_0} & \frac{g}{y_0} & -2\sqrt{\frac{g}{M y_0}} \\
0 & 0 & \frac{g}{y_0} - \frac{R}{L}
\end{bmatrix}
\begin{bmatrix}
v_{21} \\
v_{22} \\
v_{23}
\end{bmatrix}
= 0
\]

De la primera ecuación, notamos que:
\[ \frac{g}{y_0} v_{21} + v_{22} = 0 \]

Lo que nos da:
   \[
   v_{22} = -\frac{g}{y_0} v_{21}
   \]
   
   Además, de la tercera ecuación \(
   v_{23} = 0
   \):
   \[\]

Por lo tanto, el vector propio asociado a \( \lambda_2 = -\frac{g}{y_0} \) es:

\[
\Vec{v_2} = 
\begin{bmatrix}
1 \\
-\frac{g}{y_0} \\
0
\end{bmatrix}
\]

\item De forma análoga al paso anterior, para \( \lambda_3 = \frac{g}{y_0} \) se obtiene:

\[
\Vec{v_3} = 
\begin{bmatrix}
1 \\
\frac{g}{y_0} \\
0
\end{bmatrix}
\]
\end{itemize}
Luego, la matriz \( P \) formada por los vectores propios es:

\[
P = 
\begin{bmatrix}
1 & 1 & 1 \\
-\frac{R}{L} & -\frac{g}{y_0} & \frac{g}{y_0} \\
\frac{1}{2\sqrt{\frac{g}{M y_0}}} \left( \frac{g}{y_0} - \frac{R^2}{L^2} \right) & 0 & 0
\end{bmatrix}
\]
Considerando $R/L=2$, $g/y_0=1$ y $M=4$:
\[
P = 
\begin{bmatrix}
1 & 1 & 1 \\
-2 & -1 & 1 \\
-3 & 0 & 0
\end{bmatrix}
\]
y 
\[
\det(P) = -6
\]
Luego, la adjunta de $P$ corresponderá a

 Paso 1: Calcular los cofactores

Cofactor \( C_{11} \):

Eliminamos la primera fila y la primera columna:

\[
C_{11} = \det\begin{bmatrix}
-1 & 1 \\
0 & 0
\end{bmatrix} = (-1)(0) - (1)(0) = 0
\]

 Cofactor \( C_{12} \):

Eliminamos la primera fila y la segunda columna:

\[
C_{12} = \det\begin{bmatrix}
-2 & 1 \\
-3 & 0
\end{bmatrix} = (-2)(0) - (1)(-3) = 3
\]

Debido al signo en la posición \( C_{12} \), tenemos:

\[
C_{12} = -3
\]

Cofactor \( C_{13} \):

Eliminamos la primera fila y la tercera columna:

\[
C_{13} = \det\begin{bmatrix}
-2 & -1 \\
-3 & 0
\end{bmatrix} = (-2)(0) - (-1)(-3) = -3
\]

 Cofactor \( C_{21} \):

Eliminamos la segunda fila y la primera columna:

\[
C_{21} = \det\begin{bmatrix}
1 & 1 \\
0 & 0
\end{bmatrix} = (1)(0) - (1)(0) = 0
\]

Cofactor \( C_{22} \):

Eliminamos la segunda fila y la segunda columna:

\[
C_{22} = \det\begin{bmatrix}
1 & 1 \\
-3 & 0
\end{bmatrix} = (1)(0) - (1)(-3) = 3
\]

Cofactor \( C_{23} \):

Eliminamos la segunda fila y la tercera columna:

\[
C_{23} = \det\begin{bmatrix}
1 & 1 \\
-3 & 0
\end{bmatrix} = (1)(0) - (1)(-3) = 3
\]

Debido al signo en la posición \( C_{23} \), tenemos:

\[
C_{23} = -3
\]

 Cofactor \( C_{31} \):

Eliminamos la tercera fila y la primera columna:

\[
C_{31} = \det\begin{bmatrix}
1 & 1 \\
-1 & 1
\end{bmatrix} = (1)(-1) - (1)(1) - = -2
\]

 Cofactor \( C_{32} \):

Eliminamos la tercera fila y la segunda columna:

\[
C_{32} = \det\begin{bmatrix}
1 & 1 \\
-2 & 1
\end{bmatrix} = (1)(1) - (1)(-2) = 3
\]

Debido al signo en la posición \( C_{31} \), tenemos:

\[
C_{33} = -3
\]

 Cofactor \( C_{33} \):

Eliminamos la tercera fila y la tercera columna:

\[
C_{33} = \det\begin{bmatrix}
1 & 1 \\
-2 & -1
\end{bmatrix} = (1)(-1) - (1)(-2) = 1
\]

 Paso 2: Formar la matriz de cofactores

La matriz de cofactores es:

\[
\text{Cof}(P) = 
\begin{bmatrix}
0 & -3 & -3 \\
0 & 3 & -3 \\
2 & -3 & 1
\end{bmatrix}
\]

 Paso 3: Adjunta de \( P \)

La adjunta de \( P \) es la transpuesta de la matriz de cofactores:

\[
\text{adj}(P) = 
\begin{bmatrix}
0 & 0 & 2 \\
-3 & 3 & -3 \\
-3 & -3 & 1
\end{bmatrix}
\]
De esta forma, la inversa de P está dada por:
\[
P^{-1} = \frac{1}{-6} 
\begin{bmatrix}
0 & 0 & 2 \\
-3 & 3 & -3 \\
-3 & -3 & 1
\end{bmatrix}
\]

\[
P^{-1} = 
\begin{bmatrix}
0 & 0 & -\frac{1}{3} \\
\frac{1}{2} & -\frac{1}{2} & \frac{1}{2} \\
\frac{1}{2} & \frac{1}{2} & -\frac{1}{6}
\end{bmatrix}
\]

Reemplazando los valores de la tabla y recordando que $e^{Dt}$ está dada por:

\[
e^{Dt} = 
\begin{bmatrix}
e^{-2t} & 0 & 0 \\
0 & e^{-t} & 0 \\
0 & 0 & e^{t}
\end{bmatrix}
\]

y la matriz $P$ es:

\[
P = 
\begin{bmatrix}
1 & 1 & 1 \\
-2 & -1 & 1 \\
-3 & 0 & 0
\end{bmatrix}
\]

El producto \( P e^{Dt} \) es:

\[
P e^{Dt} =
\begin{bmatrix}
1 & 1 & 1 \\
-2 & -1 & 1 \\
-3 & 0 & 0
\end{bmatrix}
\begin{bmatrix}
e^{-2t} & 0 & 0 \\
0 & e^{-t} & 0 \\
0 & 0 & e^{t}
\end{bmatrix}
=
\begin{bmatrix}
e^{-2t} & e^{-t} & e^{t} \\
-2e^{-2t} & -e^{-t} & e^{t} \\
-3e^{-2t} & 0 & 0
\end{bmatrix}
\]

El siguiente paso es multiplicar \( P e^{Dt} P^{-1} \):

\[
P e^{Dt} P^{-1} =
\begin{bmatrix}
e^{-2t} & e^{-t} & e^{t} \\
-2e^{-2t} & -e^{-t} & e^{t} \\
-3e^{-2t} & 0 & 0
\end{bmatrix}
\begin{bmatrix}
0 & 0 & -\frac{1}{3} \\
\frac{1}{2} & -\frac{1}{2} & \frac{1}{2} \\
\frac{1}{2} & \frac{1}{2} & -\frac{1}{6}
\end{bmatrix}
\]

Realizando la multiplicación obtenemos la matriz de transición de estado \( \text{MTE} \):

\[
\Phi(t) =
\begin{bmatrix}
\frac{1}{2} (e^{-t} + e^{t}) & \frac{1}{2} (-e^{-t} + e^{t}) & -\frac{1}{3}e^{-2t}+\frac{1}{2}e^{-t}-\frac{1}{6} e^{t} \\
\frac{1}{2} (-e^{-t} + e^{t}) & \frac{1}{2} (e^{-t} + e^{t}) &\frac{2}{3}e^{-2t}-\frac{1}{2}e^{-t}-\frac{1}{6} e^{t} \\
0 & 0 & e^{-2t} 
\end{bmatrix}
\]

Ahora bien, para calcular la RESC, debemos recordar que la matriz $B$ reemplazando $L=1$ está dada por:
\[
B=
\begin{bmatrix}
0 \\
0 \\
1
\end{bmatrix}
\]
y
\[
C = \begin{bmatrix} 1 & 0 & 0 \end{bmatrix}
\]
Ahora, debemos calcular la multiplicación \( C \cdot \Phi(t - \tau) \cdot B \). Para esto, empezamos con \( \Phi(t - \tau) \cdot B \):

\[
\Phi(t - \tau) \cdot B =
\begin{bmatrix}
\frac{1}{2} (e^{-(t-\tau)} + e^{(t-\tau)}) & \frac{1}{2} (-e^{-(t-\tau)} + e^{(t-\tau)}) & -\frac{1}{3}e^{-2(t-\tau)}+\frac{1}{2}e^{-(t-\tau)}-\frac{1}{6} e^{(t-\tau)} \\
\frac{1}{2} (-e^{-(t-\tau)} + e^{(t-\tau)}) & \frac{1}{2} (e^{-(t-\tau)} + e^{(t-\tau)}) & \frac{2}{3}e^{-2(t-\tau)}-\frac{1}{2}e^{-(t-\tau)}-\frac{1}{6} e^{(t-\tau)} \\
0 & 0 & e^{-2(t-\tau)}
\end{bmatrix}
\begin{bmatrix}
0 \\
0 \\
1
\end{bmatrix}
\]

Multiplicando, obtenemos:

\[
\Phi(t - \tau) \cdot B =
\begin{bmatrix}
-\frac{1}{3}e^{-2(t-\tau)} + \frac{1}{2}e^{-(t-\tau)} - \frac{1}{6}e^{(t-\tau)} \\
\frac{2}{3}e^{-2(t-\tau)} - \frac{1}{2}e^{-(t-\tau)} - \frac{1}{6}e^{(t-\tau)} \\
e^{-2(t-\tau)}
\end{bmatrix}
\]

Ahora, multiplicamos \( C \cdot \Phi(t - \tau) \cdot B \):

\[
C \cdot \Phi(t - \tau) \cdot B =
\begin{bmatrix}
1 & 0 & 0
\end{bmatrix}
\begin{bmatrix}
-\frac{1}{3}e^{-2(t-\tau)} + \frac{1}{2}e^{-(t-\tau)} - \frac{1}{6}e^{(t-\tau)} \\
\frac{2}{3}e^{-2(t-\tau)} - \frac{1}{2}e^{-(t-\tau)} - \frac{1}{6}e^{(t-\tau)} \\
e^{-2(t-\tau)}
\end{bmatrix}
\]

Finalmente, el resultado de la multiplicación es:

\[
C \cdot \Phi(t - \tau) \cdot B = -\frac{1}{3}e^{-2(t-\tau)} + \frac{1}{2}e^{-(t-\tau)} - \frac{1}{6}e^{(t-\tau)}
\]

Multiplicando la entrada $e(\tau)$:

\[
 e(\tau) \left( -\frac{1}{3}e^{-2(t-\tau)} + \frac{1}{2}e^{-(t-\tau)} - \frac{1}{6}e^{(t-\tau)} \right)
\]

Luego, la RESC quedará como:

\[ RESC = \int_{t_0}^{t}
 e(\tau) \left( -\frac{1}{3}e^{-2(t-\tau)} + \frac{1}{2}e^{-(t-\tau)} - \frac{1}{6}e^{(t-\tau)} \right) d\tau
\]

Por otra parte, la RENC por otra parte quedará como:

\[ RENC = C \Phi(t) \Vec{Y_0} = \begin{bmatrix} 1 & 0 & 0 \end{bmatrix}
\begin{bmatrix}
\frac{1}{2} (e^{-t} + e^{t}) & \frac{1}{2} (-e^{-t} + e^{t}) & -\frac{1}{3}e^{-2t}+\frac{1}{2}e^{-t}-\frac{1}{6} e^{t} \\
\frac{1}{2} (-e^{-t} + e^{t}) & \frac{1}{2} (e^{-t} + e^{t}) &\frac{2}{3}e^{-2t}-\frac{1}{2}e^{-t}-\frac{1}{6} e^{t} \\
0 & 0 & e^{-2t} 
\end{bmatrix} \Vec{Y_0}
\]

\[ RENC = \begin{bmatrix} \frac{1}{2} (e^{-t} + e^{t}) & \frac{1}{2} (-e^{-t} + e^{t}) & -\frac{1}{3}e^{-2t}+\frac{1}{2}e^{-t}-\frac{1}{6} e^{t} \end{bmatrix} \cdot \begin{bmatrix}
y_{ci} \\
\dot{y}_{ci} \\
i_{ci}
\end{bmatrix} 
\]
Es decir,
\[
RENC = \frac{1}{2} (e^{-t} + e^{t}) y_{ci} + \frac{1}{2} (-e^{-t} + e^{t}) \dot{y}_{ci} + \left( -\frac{1}{3}e^{-2t} + \frac{1}{2}e^{-t} - \frac{1}{6} e^{t} \right) i_{ci}
\]

Por lo tanto, la respuesta completa del sistema estará dado por:
\begin{align*}
y(t) = & \, \frac{1}{2} (e^{-t} + e^{t}) y_{ci} + \frac{1}{2} (-e^{-t} + e^{t}) \dot{y}_{ci} \\
& + \left( -\frac{1}{3}e^{-2t} + \frac{1}{2}e^{-t} - \frac{1}{6} e^{t} \right) i_{ci} \\
& + \int_{t_0}^{t}
 e(\tau) \left( -\frac{1}{3}e^{-2(t-\tau)} + \frac{1}{2}e^{-(t-\tau)} - \frac{1}{6}e^{(t-\tau)} \right) d\tau
\end{align*}


\item \textit{ \textbf{(1.5 Puntos)} Considere el siguiente sistema lineal con condición inicial arbitraria $\vec{X_0}$. Determine únicamente la RENC del sistema.}
Primero se diagonaliza la matriz de estados
\[
P = 
\begin{bmatrix}
-1 & -1 \\
1 & 3 \\
\end{bmatrix}
\]
\[
e^{Dt} = 
\begin{bmatrix}
e^{-t} & 0 \\
0 & e^{-3t} \\
\end{bmatrix}
\]
\[
P^{-1} = 
\begin{bmatrix}
-\frac{3}{2} & -\frac{1}{2} \\
\frac{1}{2} & \frac{1}{2} \\
\end{bmatrix}
\]
Luego,
\begin{equation}
    \Phi(t)=
    \begin{bmatrix}
-1 & -1 \\
1 & 3 \\
\end{bmatrix}
\cdot
\begin{bmatrix}
e^{-t} & 0 \\
0 & e^{-3t} \\
\end{bmatrix}
\cdot
\begin{bmatrix}
-\frac{3}{2} & -\frac{1}{2} \\
\frac{1}{2} & \frac{1}{2} \\
\end{bmatrix}
\end{equation}
\begin{equation}
    \Phi(t)=
    \begin{bmatrix}
-1 & -1 \\
1 & 3 \\
\end{bmatrix}
\cdot
\begin{bmatrix}
-\frac{3}{2}e^{-t} & -\frac{1}{2}e^{-t} \\
\frac{1}{2}e^{-3t} & \frac{1}{2}e^{-3t} \\
\end{bmatrix}
\end{equation}
\begin{equation}
    \Phi(t) =
    \begin{bmatrix}
    \frac{3}{2}e^{-t} - \frac{1}{2}e^{-3t} & \frac{1}{2}e^{-t} - \frac{1}{2}e^{-3t} \\
    -\frac{3}{2}e^{-t} + \frac{3}{2}e^{-3t} & -\frac{1}{2}e^{-t} + \frac{3}{2}e^{-3t}
    \end{bmatrix}
\end{equation}
De forma que la RENC está dada por:
\begin{equation}
    \Vec{X_0} = \begin{bmatrix}
    \frac{3}{2}e^{-t} - \frac{1}{2}e^{-3t} & \frac{1}{2}e^{-t} - \frac{1}{2}e^{-3t} \\
    -\frac{3}{2}e^{-t} + \frac{3}{2}e^{-3t} & -\frac{1}{2}e^{-t} + \frac{3}{2}e^{-3t}
    \end{bmatrix} \Vec{X_0}
\end{equation}
\end{enumerate}