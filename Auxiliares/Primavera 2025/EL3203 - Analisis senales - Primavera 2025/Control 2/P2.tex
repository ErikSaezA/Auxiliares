\setlength\parindent{0pt} \textbf{\large Pregunta 2} \\
Considere un sistema caracterizado por el siguiente modelo en variables de estado:

\begin{align}
\dot{x}(t) &= A\cdot  x(t) + B \cdot  u(t), \\
y(t) &= C \cdot x(t),
\end{align}
\text{donde las matrices } A, B \text{ y } C \text{ son:}
\begin{align}
A &= \begin{pmatrix}
1 & 0 \\
0 & -2
\end{pmatrix}, \quad
B = \begin{pmatrix}
1 \\
1
\end{pmatrix}, \quad
C = \begin{pmatrix}
1 & 1
\end{pmatrix}.
\end{align}
%---------------------------------------------
\begin{enumerate}
    \item \textbf{(0.5 Puntos)} Analice la controlabilidad del sistema
    \item \textbf{(0.5 Puntos)} Analice la observabilidad del sistema
    \item \textbf{(2 Puntos)} Diseñe un controlador de la forma $u = -K\cdot x(t)$ tal que su sistema sea estable (\textit{Se desea obtener los polos $\lambda= -1$ y $\lambda = -5$}). 
    \item \textbf{(2 Puntos)} Diseñe un observador de Luenberger que le permita estimar el estado de su sistema \textit{(considere que los polos del observador deben ser 3 veces más negativos que todos los polos del controlador.)}.
     \item \textbf{(1 Puntos)} Suponga que los estados del sistema no se pueden medir directamente. ¿Cómo implementaría el controlador diseñado en 2.? Comente y provea un diagrama de bloques de su solución.
\end{enumerate}
%--------------------------------------------

\begin{enumerate}
    \item \textbf{(0.5 Puntos)} Analice la controlabilidad del sistema\\
    Se busca analizar la controlabilidad del sistema, para esto se utiliza la matrices A y B del sistema anterior.
    \begin{align}
        \mathcal{C} = \begin{pmatrix} B & AB & A^{2}B & \dots & A^{n-1}B \end{pmatrix}
    \end{align}
    Dado que en nuesto caso particular n=2,se tiene que:
    \begin{align}
        \mathcal{C} = \begin{pmatrix} B & AB \end{pmatrix}
    \end{align}
    Por lo que se obtiene:
    \begin{align}
        \mathcal{C} = \begin{pmatrix} B & AB \end{pmatrix}
    \end{align}
    \begin{align}
        \mathcal{C} = \begin{pmatrix} 1 & 1 \\ 1 & -2 \end{pmatrix}
    \end{align}
    Luego observa que filas y columnas son li, por lo tanto el sistema es controlable. (\textit{Tambien es posible realizarlo mediante que el determinante sea distinto de 0, para analizar que sea de rango completo})
    \item \textbf{(0.5 Puntos)} Analice la observabilidad del sistema\\
    
    Se busca analizaar la controlabilidad del sistema, para esto se utiliza la matrices A y C del sistema anterior. Por tanto
    \begin{align}
        \Theta &= \begin{pmatrix} C \\ CA \end{pmatrix}
    \end{align}
    Por lo que se obtiene:
    \begin{align}
        \Theta &= \begin{pmatrix} C \\ CA \end{pmatrix}
    \end{align}
    \begin{align}
        \Theta &= \begin{pmatrix} 1 & 1 \\ 1 & -2 \end{pmatrix}
    \end{align}
    Luego observa que filas y columnas son li, por lo tanto el sistema es observable. (\textit{Tambien es posible realizarlo mediante que el determinante sea distinto de 0, para analizar que sea de rango completo})
    \item \textbf{(2 Puntos)} Diseñe un controlador de la forma $u = -K\cdot x(t)$ tal que su sistema sea estable (\textit{Se desea obtener los polos $\lambda= -1$ y $\lambda = -5$}).\\
    Se busca diseñar un controlador tal que el sistema sea estable, y en particular obtener los polos $\lambda= -1$ y $\lambda = -5$.Por lo tanto se considera que la entrada viene dada por:
    \begin{align}
        u(t)= r(t) - Kx(t)
    \end{align}
    De esta manera se tendra que el sistema tendra la siguiente forma:
    \begin{align}
        \dot{x}(t) &= (A-BK)x(t) + Br(t)\\
        y(t) &= Cx(t)
    \end{align}
    Con lo que deberemos encontrar la matriz k, tal de moverlo slos valores propios donde nos interesa. Es por esto que se tiene que:
    \begin{align}
        K=\begin{pmatrix} k_{1} & k_{2} \end{pmatrix}
    \end{align}
    Por lo tanto los valores propios vendran dados por:
    \begin{align}
        |A - BK - \lambda I| &= \left| \begin{matrix} 1 - k_1 - \lambda & -k_2 \\ -k_1 & -2 - k_2 - \lambda \end{matrix} \right| \\
        &= (1 - k_1 - \lambda)(-2 - k_2 - \lambda) - k_1 k_2 \\
        &= \lambda^2 + (2 + k_1 + k_2)\lambda + (- 2 + 2k_1 -k_2)
    \end{align}
    Tenemos que los valores propios que se buscan son $\lambda = -1$ y $\lambda = -5$, por lo que se tiene que:
    \begin{align}
        (\lambda + 5)(\lambda + 1) &= 0\\
        \lambda^2 + 6\lambda + 5 &= 0
    \end{align}
    De esta manera se forman el sistema de ecuaciones
    \begin{align}
        2 + k_1 + k_2 &= 6\\
        - 2 + 2k_1 -k_2 &= 5
    \end{align}
    Dadon como resultado que $k_1 = \frac{11}{3}$ y $k_2 = \frac{1}{3}$, con lo que se obtiene la matriz k dada por:
    \begin{align}
        K = \begin{pmatrix} \frac{11}{3} & \frac{1}{3} \end{pmatrix}
    \end{align}

    \item \textbf{(2 Puntos)} Diseñe un observador de Luenberger que le permita estimar el estado de su sistema \textit{(considere que los polos del observador deben ser 3 veces más negativos que todos los polos del controlador.)}.
    Se busca diseñar un observador de Luenberger, considerando que los polos son 3 veces más negativos que los polos del controlador. Por lo que se tiene que los polos del observador son $\lambda = -3$ y $\lambda = -15$. Por lo que se tiene que la matriz F vendra dada por:
    
    \begin{align}
         F = \begin{pmatrix} f_1 \\ f_2 \end{pmatrix}
    \end{align}
    Por lo tanto los valores propios vendrán dados por:
    \begin{align}
     |A - FC - \lambda I| &= \left| \begin{matrix} 1 - f_1 - \lambda & -f_1 \\ -f_2 & -2 - f_2 - \lambda \end{matrix} \right| \\
        &= (1 - f_1 - \lambda)(-2 - f_2 - \lambda) - f_1 f_2 \\
        &= \lambda^2 + (2 + f_1 + f_2)\lambda + (-2 + 2f_1 - f_2)
    \end{align}
    Tenemos que los valores propios que se buscan son $\lambda = -3$ y $\lambda = -15$, por lo que se tiene que:
    \begin{align}
        (\lambda + 15)(\lambda + 3) &= 0\\
        \lambda^2 + 18\lambda + 45 &= 0
    \end{align}
    De esta manera se forman el sistema de ecuaciones:
    \begin{align}
        2 + f_1 + f_2 &= 18\\
        - 2 + 2f_1 -f_2 &= 45
    \end{align}
    Dadon como resultado que $f_1 = 21$ y $f_2 = -5$, con lo que se obtiene la matriz F dada por:
    \begin{align}
        F = \begin{pmatrix} 21 \\ -5 \end{pmatrix}
    \end{align}
     \item \textbf{(1 Puntos)} Suponga que los estados del sistema no se pueden medir directamente. ¿Cómo implementaría el controlador diseñado en 2.? Comente y provea un diagrama de bloques de su solución.
     En ese caso, se deberia realizar un observador lo suficientemente bueno para estimar los estados del sistema, es decir que el error converga de manera rapida a 0. De esta manera se podra asumir que dicha observacion corresponde al estado y por tanto diseñar un controlador que se base en dicha observacion.
\end{enumerate}

