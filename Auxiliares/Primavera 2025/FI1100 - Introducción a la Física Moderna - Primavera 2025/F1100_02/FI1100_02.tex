\begin{questions}
%--------------------------
\question Un bloque de masa $m$ cuelga verticalmente de un resorte de constante el\'astica $k$ y largo natural $\ell_0$. El sistema se encuentra inicialmente en reposo en su posici\'on de equilibrio. A partir de $t=0$, el punto de suspensi\'on (techo) comienza a oscilar verticalmente como $y_t(t)=A\cos(\Omega t)$, tomando positivo hacia abajo. Considere la aceleraci\'on de gravedad $g$ dirigida hacia abajo.

\begin{parts}
  \part Encuentre la posici\'on de equilibrio del bloque (elongaci\'on est\'atica respecto del largo natural).
  \part Obtenga la ecuaci\'on diferencial de movimiento para la coordenada del bloque $y(t)$ medida desde un origen fijo, usando $y_t(t)$ como entrada. Exprese su resultado en la forma est\'andar de un oscilador forzado.
  \part Determine la amplitud (respuesta estacionaria) del movimiento del bloque en r\'egimen permanente en funci\'on de $A$, $m$, $k$ y $\Omega$.
\end{parts}
%--------------------------
\end{questions}