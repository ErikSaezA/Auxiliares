\documentclass[
  11pt,
  letterpaper,
  % addpoints,
   answers
  ]{exam}
\usepackage{float}
\usepackage{../exercise-preamble}

\begin{document}

\noindent
\begin{minipage}{0.47\textwidth}
  \includegraphics[width=\textwidth]{../fcfm_die}
\end{minipage}
\begin{minipage}{0.53\textwidth}
\begin{center} 
\large\textbf{Electromagnetismo Aplicado} (EL3103) \\
\large\textbf{Clase auxiliar 3} \\
\normalsize Prof.~\professor\\
\normalsize Prof. Aux. Lucas Palomino
\\
\normalsize Ayudantes: \ayudanteA~-~\ayudanteB
\end{center}
\end{minipage}

\vspace{0.5cm}
\noindent
\vspace{.85cm}
\section{Resumen:}
\subsection*{Ecuación de Laplace en 2D}
Cuando se tienen potenciales que dependen de dos variables, la ecuación de Laplace se representará de la siguiente manera:
\begin{align}
    \nabla^2 V_{(x,y)} &= \frac{\partial^2 V}{\partial x^2} + \frac{\partial^2V}{\partial y^2} = 0 
\end{align}
Es decir, la EDO ahora se convierte en una EDP lineal y homogénea, sujeta a sus respectivas condiciones de borde. Esta se podrá resolver mediante el método de variables separables.
\subsection*{Método de variables separables}
La EDP se podrá resolver siguiendo estos pasos: \\
\\
\\
\underline{Paso 1:} Suponemos una solución de la forma
\begin{align}
    V_{(x,y)} &= X_{(x)} \cdot Y_{(y)}
\end{align}
Y reemplazamos en la EDP:
\begin{align}
\frac{\partial^2}{\partial x^2}[X(x)Y(y)] + \frac{\partial^2}{\partial y^2}[X(x)Y(y)] &= 0\\
X''(x)Y(y) + Y''(y)X(x) &= 0 \\
X''(x)Y(y) &= - Y''(y)X(x) \\
\frac{X''(x)}{X(x)} &= -\frac{Y''(y)}{Y(y)}
\end{align}
Además, igualamos a una constante $\lambda$ y queda:
\begin{align}
    \frac{X''(x)}{X(x)} &= -\frac{Y''(y)}{Y(y)} = \lambda
\end{align}
De esta forma, tenemos dos EDOS:
\begin{align}
    \frac{X''(x)}{X(x)} &= \lambda \\
    -\frac{Y''(y)}{Y(y)} &= \lambda
\end{align}
\underline{Paso 2:} Elegir una de los EDOS (idealmente la que sea más simple) y resolverla, poniéndose en casos de $\lambda$:
\begin{align}
    \frac{X''(x)}{X(x)} &= \lambda \\
\end{align}
Resolvemos para los casos $\lambda = 0$, $\lambda > 0 \rightarrow \lambda = k^2$, y finalmente el caso $\lambda < 0 \rightarrow \lambda = -k^2$. También se despejan las constantes que aparecen al resolver la EDO utilizando las condiciones de borde que tengamos. 
\\
\\
\textit{Observación: Suele ocurrir que los dos primeros casos ($\lambda = 0$ y $\lambda > 0$) nos entregan soluciones triviales igual a 0 que no nos sirven, siendo el tercer caso el único útil. De igual forma, debemos revisar los tres casos ya que a veces los dos primeros entregan soluciones.}
\\
\\
\underline{Paso 3:} Ya teniendo la solución de la primera variable (en este caso $X(x)$), reemplazamos en la EDP original. Ahora nos quedará otra EDO para encontrar la segunda variable ($Y(y)$). También se despejarán las constantes utilizando las condiciones de borde pertinentes. 
\\
\\
\underline{Paso 4:} La solución a la EDP será de la forma:
\begin{align}
    V(x,y) = \sum_{n=1}^{\infty} X(x) Y(y)
\end{align}
Donde la solución es una serie de Fourier. 

\subsection*{Serie de Fourier:} Sea $f : [-\tau , \tau] \rightarrow \mathbb{R}$ función integrable. Se define la serie de Fourier de $f$ como:
\begin{align}
    S_f(x) &= \frac{a_0}{2} + \sum_{n=1}^\infty (a_n \cos{(\frac{n \pi x}{\tau})} + b_n \sin{(\frac{n \pi x}{\tau})})
\end{align}
Donde:
\begin{align}
    a_0 = \frac{1}{\tau} \int_{-\tau}^\tau f(x)\, dx && a_n = \frac{1}{\tau} \int_{-\tau}^\tau f(x) \cos\!\left(\frac{n \pi x}{\tau}\right) dx && b_n = \frac{1}{\tau} \int_{-\tau}^\tau f(x) \sin\!\left(\frac{n \pi x}{\tau}\right) dx
\end{align}
\vspace{0.2cm}

\subsection*{¿Cómo resolver una EDO por polinomio característico?}
\underline{Caso 1:}
Si tenemos una EDO de la forma:
\begin{align}
    y'' - \omega^2 y = 0
\end{align}
Entonces utilizamos el operador diferencial tal que:
\begin{align}
    y'' \rightarrow D^2 \\
    y' \rightarrow D \\ 
    y \rightarrow 1
\end{align}
Reemplazando en la EDO:
\begin{align}
    D^2 - \omega^2 = 0
\end{align}
Despejando D:
\begin{align}
     D^2 &= \omega^2 \\
     D &= \sqrt{\omega^2} \\
     D &= \pm \omega
\end{align}
De esta forma, la solución es de la forma:
\begin{align}
    y(x) &= Ae^{\omega x} + Be^{-\omega x}
\end{align}

\underline{Caso 2:}
Si tenemos una EDO de la forma:
\begin{align}
    y'' + \omega^2 y = 0
\end{align}
Reemplazamos por el operador diferencial y despejamos:
\begin{align}
    D^2 + \omega^2 &= 0 \\
    D^2 &= -\omega^2 \\
    D &= \sqrt{-\omega^2} \\
    D &= \sqrt{-1} \cdot \sqrt{\omega^2} \\
    D &= \pm \omega \cdot j
\end{align}
Y la solución es de la forma:
\begin{align}
    y(x) &= A \sin{(\omega x)} + B \cos{(\omega x)}
\end{align}
\vspace{0.2cm}
\subsection*{Clasificación de materiales según sus propiedades magnéticas:}

\begin{table}[h!]
\centering
\renewcommand{\arraystretch}{1.3}
\begin{tabular}{|c|c|c|c|c|}
\hline
\textbf{Tipo de material} & \(\mu_r\) (orden) & \(\chi_m\) (orden) & \textbf{Linealidad} & \textbf{Ejemplos} \\
\hline
Diamagnético & \(\sim 1 - 10^{-5}\) & \(-10^{-6} \; \text{a} \; -10^{-5}\) & Lineales & Plomo, Mercurio \\
\hline
Paramagnético & \(\sim 1 + 10^{-3}\) & \(+10^{-5} \; \text{a} \; +10^{-3}\) & Lineales & Aluminio, Platino \\
\hline
Ferromagnético & \(10^{2} \; \text{a} \; 10^{5}\) & \(10^{2} \; \text{a} \; 10^{5}\) & No lineales (loop de histéresis) & Hierro, Níquel \\
\hline
\end{tabular}
\caption{Clasificación de materiales magnéticos según permeabilidad relativa, susceptibilidad y linealidad.}
\end{table}
\textit{Observación: $\chi_m$ es adimensional. Por otro lado, si bien $\mu$ tiene unidad de medida $[H/m]$, $\mu_r = \frac{\mu}{\mu_0}$ es adimensional.}



\newpage
\section{Ejercicios:}

\begin{questions}
\question \label{rectangular_plate} Considere una placa rectangular de dimensiones $0 < x < a$, $0 < y < b$, donde el potencial $V(x,y)$ satisface la ecuación de Laplace, 
\begin{align}
    \nabla^2 V(x,y) &= \frac{\partial^2 V}{\partial x^2} + \frac{\partial^2 V}{\partial y^2} = 0
\end{align}
con condiciones de borde:
\begin{align}
    V(0,y) = 0, && V(a,y) = 0, && V(x,0) = 0, && V(x,b) = f(x)
\end{align}
\begin{parts}
    \part Determine el potencial $V(x,y)$ de forma explícita.
    \part Obtenga el campo eléctrico.
\end{parts}
\vspace{0.2cm}
\begin{solution}
    \begin{parts}
        \part Seguiremos el paso a paso del resumen:
        \\
        \underline{Paso 1:} Suponemos que el potencial es de la forma:
        \begin{align}
            V(x,y) = X(x)\cdot Y(y)
        \end{align}
        Y reemplazamos en la EDP e igualamos a $\lambda$:
        \begin{align}
            \frac{X''(x)}{X(x)} = -\frac{Y''(y)}{Y(y)} = \lambda
        \end{align}
        Podemos separar en dos EDOS:
        \begin{align}
            \frac{X''(x)}{X(x)} &= \lambda \\
            \frac{Y''(y)}{Y(y)} &= -\lambda
        \end{align}
        \textit{Observación: Es importante hacer la separación de variables en las condiciones de borde.}
        \\
        \\
        Las condiciones de borde quedan:
        \begin{align}
            V(0,y) &= 0 \rightarrow X(0) \cdot Y(y) = 0 \rightarrow X(0) = 0 \\
            V(a,y) &= 0 \rightarrow X(a) \cdot Y(y) = 0 \rightarrow X(a) = 0 \\
            V(x,0) &= 0 \rightarrow X(x) \cdot Y(0) = 0 \rightarrow Y(0) = 0 \\
            V(x,b) &= f(x) \rightarrow X(x) \cdot Y(b) = f(x)
        \end{align}
        \underline{Paso 2:} Partiremos resolviendo la EDO que depende de $x$, ya que tenemos más condiciones de borde. Así, partimos por el caso $\lambda = 0$:
        \begin{align}
            \frac{X''(x)}{X(x)} &= 0 \\
            X(x) &= Ax + B
        \end{align}
        Despejamos las constantes usando las condiciones de borde:
        \begin{align}
            X(0) &= 0 = A \cdot 0 + B \rightarrow B = 0 \rightarrow X(x) = Ax \\
            X(a) &= 0 = A \cdot a \rightarrow A = 0 \rightarrow X(x) = 0, \forall x \in [0,a]
            \end{align}
            Es decir, es una solución trivial y no nos sirve.
            \\
            \\
        \underline{Si $\lambda > 0$}: Tomamos $\lambda = k^2$, esto nos facilitará la vida después.
        \begin{align}
            \frac{X''(x)}{X(x)} &= k^2 \\
            X''(x) &= k^2 \cdot X(x)
        \end{align}
        Por polinomio característico:
        \begin{align}
            D^2 &= k^2 \\
            D &= \pm k \\
            \rightarrow X(x) &= A \cdot e^{kx} + B \cdot e^{-kx}
        \end{align}
        Usamos condiciones de borde:
        \begin{align}
            X(0) &= A \cdot e^{0} + B \cdot e^{0} = A+B = 0 \rightarrow A = -B \rightarrow X(x) = A \cdot e^{kx} - A \cdot e^{-kx}\\
            X(a) &= 0 = A \cdot e^{ka} - A \cdot e^{-ka} \rightarrow A = 0 \rightarrow X(x) = 0, \forall x \in [0,a]
            \end{align}
            Es decir, otra solución trivial.
            \\
            \\
        \underline{Si $\lambda < 0$}: Es decir, tomamos $\lambda = -k^2$. Nos queda:
        \begin{align}
            \frac{X''(x)}{X(x)} &= -k^2 \\
            X''(x) = -k^2 \cdot X(x)
        \end{align}
        Por polinomio característico:
        \begin{align}
            D^2 &= -k^2 \\
            D &= \sqrt{-k^2} \\
            D &= \sqrt{k^2} \cdot \sqrt{-1} \\
            D &= \pm j \cdot k \\
            \rightarrow X(x) &= A\cdot \sin{(kx)} + B \cdot \cos{(kx)}
        \end{align}
        Usando las condiciones de borde:
        \begin{align}
            X(0) &= 0 = B \rightarrow B = 0 \rightarrow X(x) = A\cdot \sin{(kx)}\\
            X(a) &= 0 = A\cdot \sin{(k \cdot a)}
        \end{align}
        Si nos fijamos, nuevamente llegaríamos a que $A=0$ y tendríamos otra solución trivial. Es por esto que imponemos que $A \neq 0$ para obtener la solución. Como $A \neq 0$, entonces la única opción es que $\sin{(k \cdot a)} = 0$. Resolviendo:
        \begin{align}
            X(a) &= 0 = A \cdot \sin{(k \cdot a)} \\
            A &\neq 0 \rightarrow \sin{(k \cdot a)} \rightarrow k \cdot a = n \cdot \pi , n \in \mathbb{N} \setminus{0} \rightarrow k_n = \frac{n \cdot \pi}{a}
        \end{align}
        De esta forma, nos queda que:
        \begin{align}
            X_n(x) &= A_n\cdot \sin\!\left(\frac{n \pi x}{a}\right)
        \end{align}
        \underline{Paso 3:} Reemplazamos $X_n(x)$ en la EDP del principio, primero calculamos la segunda derivada:
        \begin{align}
            X_n(x) &= A_n\cdot \sin\!\left(\frac{n \pi x}{a}\right) \\
            \rightarrow X'_n(x) &= A_n\cdot \left(\frac{n \pi}{a}\right) \cos\!\left(\frac{n \pi x}{a}\right) \\
            \rightarrow X''_n(x) &= -A_n\cdot \left(\frac{n \pi}{a}\right)^2 \sin\!\left(\frac{n \pi x}{a}\right)
        \end{align}
        Reemplazando:
        \begin{align}
            \frac{-A_n\cdot \left(\frac{n \pi}{a}\right)^2 \sin\!\left(\frac{n \pi x}{a}\right)}{A_n\cdot \sin\!\left(\frac{n \pi x}{a}\right)} &= -\frac{Y''(y)}{Y(y)} \\
            \frac{Y''(y)}{Y(y)} &= \left(\frac{n \pi}{a}\right)^2
        \end{align}
        Resolviendo esa EDO por polinomio característico se obtendrá:
        \begin{align}
            Y_n(y) &= C_n \cdot e^{\frac{n \pi y}{a}} + D_n \cdot e^{-\frac{n \pi y}{a}}
        \end{align}
        Usando la condición de borde que nos sirve (la tercera):
        \begin{align}
            Y_n(0) &= 0 = C_n + D_n \rightarrow D_n = -C_n \rightarrow Y_n(y) = C_n \cdot e^{\frac{n \pi y}{a}} - C_n \cdot e^{-\frac{n \pi y}{a}} \\
            Y_n(y) &= C_n \cdot \bigl[e^{\frac{n \pi y}{a}} - e^{-\frac{n \pi y}{a}}\bigr]
        \end{align}
        Sabemos que $\frac{e^{x} - e^{-x}}{2} = \sinh{(x)}$, así:
        \begin{align}
            Y_n(y) &= C_n \cdot \bigl[e^{\frac{n \pi y}{a}} - e^{-\frac{n \pi y}{a}}\bigr] = 2 C_n \cdot \sinh\!\left(\frac{n \pi y}{a}\right)
        \end{align}
        Definiendo $B_n = 2C_n$:
        \begin{align}
            Y_n(y) = B_n \cdot \sinh\!\left(\frac{n \pi y}{a}\right)
        \end{align}
    \underline{Paso 4:} La solución a la ecuación de Laplace queda:
    \begin{align}
    V(x,y) &= \sum_{n=1}^{\infty} X(x) Y(y) \\ 
    V(x,y) &= \sum_{n=1}^{\infty} A_n \cdot \sin\!\left(\frac{n \pi x}{a}\right) \cdot B_n \cdot \sinh\!\left(\frac{n \pi y}{a}\right)
    \end{align}
    Definiendo $\hat{A_n} = A_n \cdot B_n$:
    \begin{align}
    V(x,y) &= \sum_{n=1}^{\infty} \hat{A_n} \cdot \sin\!\left(\frac{n \pi x}{a}\right) \cdot \sinh\!\left(\frac{n \pi y}{a}\right)
    \end{align}
    Finalmente, utilizando la última ecuación de borde, se puede despejar la constante y obtener el potencial:
    \begin{align}
    V(x,b) &= \sum_{n=1}^{\infty} \hat{A_n} \cdot \sinh\!\left(\frac{n \pi b}{a}\right) \cdot \sin\!\left(\frac{n \pi x}{a}\right) = f(x)
    \end{align}
    Sabemos que la función obtenida es una serie de Fourier. Tenemos solo el término del seno de la serie de Fourier, por ende, podemos igualar la constante con la constante conocida de la serie de Fourier. Entonces:
    \begin{align}
    \hat{A_n} \cdot \sinh\!\left(\frac{n \pi b}{a}\right) &= \frac{2}{a} \int_{0}^{a} f(x) \sin\!\left(\frac{n \pi x}{a}\right) dx \\
    \hat{A_n} &= \frac{2}{a} \cdot \frac{1}{\sinh\!\left(\frac{n \pi b}{a}\right)}\int_{0}^{a} f(x) \sin\!\left(\frac{n \pi x}{a}\right) dx
    \end{align}
    Y de esta forma, el potencial nos queda:
    \begin{align}
    V(x,y) &= \sum_{n=1}^{\infty} \frac{2}{a} \cdot \frac{1}{\sinh\!\left(\frac{n \pi b}{a}\right)} \left[\int_{0}^{a} f(x) \sin\!\left(\frac{n \pi x}{a}\right) dx\right] \cdot \sin\!\left(\frac{n \pi x}{a}\right) \cdot \sinh\!\left(\frac{n \pi y}{a}\right)
    \end{align}

    \part Para obtener el campo, simplemente calculamos $\mathbf{E} = -\nabla V$:
    \begin{align}
        \mathbf{E} &= -\nabla \mathbf{V}\\
        \mathbf{E} &= -(\frac{\partial V}{\partial x} \mathbf{\hat{x}} + \frac{\partial V}{\partial y} \mathbf{\hat{y}})
    \end{align}
    \begin{multline}
        \mathbf{E} = -\sum_{n=1}^{\infty} \Bigg[ \frac{2}{a} \cdot \frac{1}{\sinh\!\left(\frac{n \pi b}{a}\right)}\int_{0}^{a} f(x) \sin\!\left(\frac{n \pi x}{a}\right) dx \, \frac{n \pi}{a} \, \cos\!\left(\frac{n \pi x}{a}\right) \, \sinh\!\left(\frac{n \pi y}{a}\right) \, \mathbf{\hat{x}} \\
        +\, \frac{2}{a} \cdot \frac{1}{\sinh\!\left(\frac{n \pi b}{a}\right)}\int_{0}^{a} f(x) \sin\!\left(\frac{n \pi x}{a}\right) dx \, \frac{n \pi}{a} \, \sin\!\left(\frac{n \pi x}{a}\right) \, \cosh\!\left(\frac{n \pi y}{a}\right) \, \mathbf{\hat{y}} \Bigg]
    \end{multline}
    
    
\end{parts}
\end{solution}

\newpage

\question Sea el esquema mostrado en la \cref{fig:cylinder}, el cual consiste en un par de polos con un corte transversal. Cilindro y polos se extienden infinitamente. Se busca obtener un campo con una expresión $\mathbf{H}$ = $H_{0}\cos(\theta)\uvecs{r}$ en la superficie cilíndrica $(r = a)$:
  \begin{parts}

    \part Indique el tipo de material del cilindro a partir del valor de $\mu$. ¿Qué debe cumplir el material para ser lineal, homogéneo e isotrópico?
    \part{Encuentre una expresión generalizada para el potencial escalar magnético mediante separación de variables (no es necesario despejar las constantes).}
    \part{Considerando ($n=1$) determine una expresión para el potencial escalar magnético y para $\mathbf{H}(r,\theta)$ en todo el espacio. (Despeje las constantes).}
  \end{parts}

  \begin{center}
    \begin{tikzpicture}
      \edef\radius{2}
      \edef\dist{3}
      \edef\angle{60}
      \coordinate (O) at (0, 0);

      \draw[thick] (O) circle (\radius);
      \draw[->] (O) -- ({\radius/sqrt(2)}, {\radius/sqrt(2)}) node[midway, fill=white, rotate=45] {$a$};

      \draw[thick] (O) ++ (\dist, -.5) coordinate (B1) -- ++ (0, 1) coordinate (A1);
      \draw[thick] (O) ++ (-\dist, -.5) coordinate (B2) -- ++ (0, 1) coordinate (A2);

      \draw[thick] (A1) -- ++ ({2*\radius * cos(\angle)}, {2*\radius * sin(\angle)}) node[midway,above, rotate=\angle] {$V_m=-NI$};
      \draw[thick] (B1) -- ++ ({2*\radius * cos(\angle)}, -{2*\radius * sin(\angle)});

      \draw[thick] (A2) -- ++ (-{2*\radius * cos(\angle)}, {2*\radius * sin(\angle)}) node[midway,above, rotate=-\angle] {$V_m=NI$};;
      \draw[thick] (B2) -- ++ (-{2*\radius * cos(\angle)}, -{2*\radius * sin(\angle)}) node[xshift=1cm] {$\epsilon_0, \mu_0$};

      \draw[dashed, -latex] (O) ++ (-2*\dist, 0) -- ++ (4*\dist, 0) node[below] {$x$};
      \draw[dashed, -latex] (O) ++ (0, -1.5*\dist) -- ++ (0, 3*\dist) node[left] {$y$};

      \node[yshift=-.5cm,fill=white] at (O) {$\mu=\infty$};
      \draw (O) ++ (-{\radius/sqrt(2)}, -{\radius/sqrt(2)}) node[below,rotate=-45] {$V_m=0$};

      \draw[<->|] (O) ++ (0, -\dist) -- ++ (\dist, 0) node[midway, fill=white] {$b$};

    \end{tikzpicture}
    \captionof{figure}{Corte transversal o \textit{cross-section} de un diagrama complejo compuesto por un cilindro y polos infinitamente largos (en el eje $z$).} \label{fig:cylinder}
  \end{center}

  \begin{solution}
    \begin{parts}
    \part Es un material ferromagnético ya que $\mu$ tiene un valor muy grande. Recordemos que en materiales paramagnéticos ($\chi_m > 0$) y diamagnéticos ($\chi_m < 0$) el orden de la permeabilidad es de $\mu \sim 1$, mientras que en ferromagnéticos es de $\mu \sim 10^2$ a $10^5$. (Pueden calcular el orden de $\mu$ en materiales paramagnéticos y diamagnéticos usando la ecuación $\mu_r = (1 + \chi_m)$ y ver que efectivamente tiene valores mucho menores que en materiales ferromagnéticos). Las condiciones que se deben cumplir son:
    \\
    \\
    \underline{Linealidad:} Se deben cumplir las ecuaciones constitutivas:
    \begin{align}
        \mathbf{P} &= \epsilon_0 \cdot \chi_e \cdot \mathbf{E} \rightarrow \mathbf{P} \propto \mathbf{E} \\
         \mathbf{M} &= \chi_m \cdot \mathbf{H} \rightarrow \mathbf{M} \propto \mathbf{H}
    \end{align}
    \textit{Observación: Los materiales ferromagnéticos casi nunca son lineales, por ende, en el cilindro no se cumpliría ninguna de las ecuaciones constitutivas e incluso se podrían generar problemas con las ecuaciones de Maxwell. Aunque para el desarrollo de los ejercicios ignoraremos esto y asumimos que el material es lineal.}
    \\
    \\
    \underline{Homogeneidad:} Las propiedades del cilindro no varían en el espacio.
    \\
    \\
    \underline{Isotropía:} Las propiedades del cilindro no dependen de la dirección.
    
    \part{Se busca obtener una expresión para el potencial escalar magnético $V_{m}$. Dada la geometría no regular de la figura se tendrá que el potencial dependerá de dos componentes, es decir $V_{m}(r,\theta)$. Además, dada la geometría es preferible utilizar coordenadas cilíndricas, luego el laplaciano en $\uvecs{r}$ y $\uvecs{\theta}$ se expresará de la siguiente manera:}
    \begin{align}
        \nabla^{2}V_{m} =  \frac{1}{r}\frac{\partial}{\partial r} \del{ r \frac{\partial V_{m}}{\partial r}} + \frac{1}{r^{2}}\del{\frac{\partial^{2}V_{m}}{\partial\theta^{2}}} &= 0\\
        \frac{1}{r} \frac{\partial V_{m}}{\partial r} + \frac{\partial^{2}V_{m}}{\partial r^2} + \frac{1}{r^{2}}\frac{\partial^{2}V_{m}}{\partial \theta^{2}}&=0
    \end{align}
    \underline{Paso 1:}
    Se tendrá una dependencia de dos variables, por lo tanto se utilizará separación de variables para su resolución, por lo que  expresando el potencial como $V_{m} = M(r)N(\theta)$, se tendrá que:
    \begin{align}
        \frac{\partial V_{m}}{ \partial r} &= M(r)'N(\theta)\\
        \frac{\partial^{2} V_{m}}{\partial r^2} &= M(r)''N(\theta)\\
        \frac{\partial^{2}V_{m}}{\partial\theta ^2} &= M(r)N(\theta)''
    \end{align}
    Reemplazamos en la EDP:
    \begin{align}
        \frac{1}{r}M'(r)N(\theta) + M''(r)N(\theta)+\frac{1}{r^2}M(r)N''(\theta) &= 0 \\
        \frac{1}{r}M'(r)N(\theta) + M''(r)N(\theta) &= -\frac{1}{r^2}M(r)N''(\theta) \\
        r^2\frac{M''(r)}{M(r)} + r\frac{M'(r)}{M(r)} &= -\frac{N''(\theta)}{N(\theta)} = \lambda
    \end{align}
    \underline{Paso 2:} Como no tenemos condiciones de borde definidas, partiremos por la EDO más fácil de resolver, es decir, resolviendo $N(\theta)$:
    \\
    \\
    \underline{caso $\lambda = 0$:}
    \begin{align}
        \frac{N''(\theta)}{N(\theta)} &= 0
    \end{align}
    Por polinomio característico se obtiene:
    \begin{align}
        N(\theta) = A \theta + B
    \end{align}
    Al estar trabajando en coordenadas cilíndricas, se debe cumplir la condición de periodicidad para el ángulo $\theta$, es decir, $N(\theta) = N(\theta + 2 \pi)$. De esta manera:
    \begin{align}
        A \theta + B &= A(\theta + 2 \pi) + B \\
        A \theta &= A \theta + 2 \pi A \\
        0 &= 2 \pi A \\
        A &= 0
    \end{align}
    De esta manera, del caso $\lambda = 0$ obtenemos la solución $N(\theta) = B_0$

    \underline{Caso $\lambda > 0 \rightarrow \lambda = k^2$}: \\
    \\
    La EDO queda:
    \begin{align}
        \frac{N''(\theta)}{N(\theta)} &= - k^2
    \end{align}
    Por polinomio característico se obtiene:
    \begin{align}
        N(\theta) &= A \sin(k \theta) + C \cos{(k \theta)}
    \end{align}
    Como necesitamos que dependa de n, nos queda:
    \begin{align}
        N(\theta) &= A_n \sin(n \theta) + C_n \cos{(n \theta)}
    \end{align}
    Por condición de periodicidad, se debe cumplir que:
    \begin{align}
        N(\theta) &= N(\theta + 2 \pi) \\
        \rightarrow A_n \sin(n \theta) + C_n \cos{(n \theta)} &= A_n \sin(n (\theta + 2 \pi) ) + C_n \cos{(n (\theta + 2 \pi))} 
    \end{align}
    Las funciones trigonométricas son períodicas por separado (se cumple tanto para el seno como para el coseno), por ende, claramente se cumple que $N(\theta) = N(\theta + 2 \pi)$. Entonces, la solución que obtenemos  para este caso es:
    \begin{align}
        N(\theta) = A_n \sin(n \theta) + C_n \cos{(n \theta)}
    \end{align}
    
    \underline{Caso $\lambda < 0 \rightarrow \lambda = - k^2$}
    \\
    \\
    Al reemplazar en la EDO queda:
    \begin{align}
        \frac{N''(\theta)}{N(\theta)} &= k^2
    \end{align}
    Y por polinomio característico se obtiene:
    \begin{align}
        N(\theta) &= D_n \cdot e^{k \theta} + E_n \cdot e^{-k \theta}
    \end{align}
    O equivalentemente:
    \begin{align}
      N(\theta) &= D_n \cdot e^{n \theta} + E_n \cdot e^{-n \theta}  
    \end{align}
    Nuevamente hay que aplicar condición de periodicidad, es decir:
    \begin{align}
        N(\theta) &= N(\theta + 2 \pi) \\
        \rightarrow D_n \cdot e^{n \theta} + E_n \cdot e^{-n \theta} &= D_n \cdot e^{n( \theta + 2 \pi)} + E_n \cdot e^{-n (\theta + 2 \pi)} \\
        D_n \cdot e^{n \theta} + E_n \cdot e^{-n \theta} &= D_n \cdot e^{n \theta} \cdot e^{2 n \pi} + E_n \cdot e^{-n \theta} \cdot e^{-2 n \pi}
    \end{align}
    Notando que $e^{2 n \pi}$ y $e^{-2 n \pi}$ son constantes que denotaremos $F_n$ y $G_n$:
    \begin{align}
        D_n \cdot e^{n \theta} + E_n \cdot e^{-n \theta} &= D_n \cdot F_n \cdot e^{n \theta} + E_n \cdot G_n \cdot e^{-n \theta}
    \end{align}
    Agrupando constantes:
    \begin{align}
        D_n \cdot e^{n \theta} + E_n \cdot e^{-n \theta} &= \hat{D_n} \cdot e^{n \theta} + \hat{E_n} \cdot e^{-n \theta}
    \end{align}
    Debido a que las constantes no son iguales a ambos lados, la única forma de que esta expresión se cumpla es si:
    \begin{align}
        D_n = E_n = \hat{D_n} = \hat{E_n} = 0
    \end{align}
    Es decir, la solución que nos queda es:
    \begin{align}
        N(\theta) = 0
    \end{align}
Pero todavía nos falta encontrar $M(r)$. Si reemplazáramos $N(\theta)$ en la EDP original y despejaramos (lo que hicimos en la pregunta 1), nos quedaría una EDO muy complicada de resolver. Por eso, es mejor resolver la EDO que obtuvimos al hacer variables separables. Es decir, hay que resolver lo siguiente:
\begin{align}
    r^2\frac{M''(r)}{M(r)} + r\frac{M'(r)}{M(r)} &= \lambda \\
    r^2 M''(r) + r M'(r) &= \lambda M(r) \\
    r^2 M''(r) + r M'(r) - \lambda M(r) &= 0
\end{align}
\underline{Caso $\lambda = 0$:}
\begin{align}
    r^2 M''(r) + r M'(r) &= 0 \\
    r^2 \frac{\partial ^2M(r)}{\partial r ^2} + r \frac{\partial M(r)}{\partial r} &= 0
\end{align}
    Esta EDO la hemos resuelto anteriormente (revisar los auxiliares anteriores). El resultado es:
    \begin{align}
        M(r) &= A \ln{(r)} + B
    \end{align}
    Aunque hay que considerar que en $r=0$, el logaritmo natural se indetermina. por ende, ignoraremos esta solución.
    \\
    \\
    \underline{Caso $\lambda \neq 0$:}
    Es una EDO de Cauchy-Euler. Una manera rápida de reconocer que corresponde a una EDO de este tipo es observar que las constantes deben tener el mismo orden que $M^n$. Por lo que las soluciones serán de la forma $M = r^{\alpha}$. Calculemos las derivadas que necesitamos:
    \begin{align}
        M(r) &= r^{\alpha} \\
        M'(r) &= \alpha \cdot r^{\alpha - 1} \\
        M''(r) &= \alpha (\alpha - 1) \cdot r^{\alpha - 2}
    \end{align}
    Reemplazando en la EDO:
    \begin{align}
        r^{2}\alpha(\alpha-1)r^{\alpha-2} + r \alpha r^{\alpha-1} - n^{2}r^{\alpha} &= 0\\
        r^{\alpha}( \alpha(\alpha -1) + \alpha - n^{2}) &= 0
    \end{align}
    Simplificando $r^{\alpha}$ tenemos que:
    \begin{align}
        (\alpha^2 - \alpha + \alpha - n^{2}) &= 0\\
        \alpha &= \pm n
    \end{align}
    De esta manera, la solución final para $M(r)$ es:
    \begin{align}
        M(r) &= D_n r^n + E_n r^{-n}
    \end{align}

    Y entonces la solución a la ecuación de Laplace es:
    \begin{align}
        V_m(r, \theta) &= \sum_{n=1}^\infty M(r) \cdot N(\theta) \\
        V_m(r, \theta) &= B_0 +\sum_{n=1}^\infty \left[(A_n \sin (n \theta) + C_n \cos(n \theta)) \cdot (D_n r^n + E_n r^{-n})\right]
    \end{align}

    
    \part Se considera el caso en que $n = 1$ por enunciado. Luego, la serie se reduce a:
    \begin{align}
        V_{m} = M_{1}N_{1} = B_0 + \left[(A \sin{(\theta)} +  C \cos{(\theta)}) \cdot (D r + \frac{E}{r}) \right]
    \end{align}
    Luego $\mathbf{H}$ se podrá escribir como el gradiente del potencial:
    \begin{align}
        \mathbf{H} = - \nabla V_{m} &= -\frac{\partial V_{m}}{\partial r} \uvecs{r}  - \frac{1}{r}\frac{\partial V_{m}}{\partial \theta} \uvecs{\theta}
    \end{align}
    Finalmente obtenemos la expresión para $\mathbf{H}$ en función de $A,B,C,D$:
    \begin{align}
        \mathbf{H} &= -\del{D - \frac{E}{r^2}}\del{A\sin(\theta) + C\cos(\theta)} \uvecs{r} - \frac{1}{r}\del{Dr + \frac{E}{r}}\del{-C\sin(\theta) + A\cos(\theta)} \uvecs{\theta}
    \end{align}
    Una vez obtenida la expresión general para $\mathbf{H}$ en relación a la EDO resultante con anterioridad, se evaluarán las condiciones de borde para obtener las constantes que caracterizan al sistema.

    \uline{Primera condición de borde $V_{m}(r=a,\theta) = 0$}
    \begin{align}
            V_{m}\del{r=a} = (A \sin(\theta) + C \cos(\theta))\del{Da + \frac{E}{a}} = 0
    \end{align}
    Luego, la única forma de que esto se cumpla es cuando el término de la derecha sea igual a 0, ya que el término de la izquierda se debe cumplir $\forall \theta$. Por ende:
    \begin{align}
            \del{Da + \frac{E}{a}}&= 0\\
            E= -Da^{2}
    \end{align}
    \uline{Segunda condición de borde $V_{m}(r=b,\theta=0) = -NI$}
    \begin{align}
            V_{m}\del{r=b,\theta=0} = (A \sin(0) + C \cos(0))\del{Db + \frac{E}{b}} &= -NI\\
            C\del{Db + \frac{E}{b}} &=-NI 
    \end{align}
    Con lo que se deriva la segunda ecuación dadas las condiciones de borde.

    \uline{Tercera condición de borde $\mathbf{H}(r=a,\theta) = H_{0}\cos(\theta) \uvecs{r}$}: Es importante notar que la condición está dada para una dirección en particular, es decir $\uvecs{r}$, por lo tanto no consideramos el término de $\mathbf{H}$ que depende de $\uvecs{\theta}$: 
    \begin{align}
        \mathbf{H}(r=a,\theta) = H_{0}\cos(\theta)  = -\left(D- \frac{E}{a^{2}}\right)(A\sin(\theta) + C \cos(\theta)) 
    \end{align}
    Para que esta condición se cumpla necesariamente cuando:
    \begin{align}
            A=0 \quad,\quad \del{\frac{E}{a^2}-D}C = H_{0}
    \end{align}
    Con esto, se tendrán 4 ecuaciones que permitan despejar las 4 constantes que caracterizan el sistema, obteniendo así una expresión explícita para $\mathbf{H}$.
    \begin{align}
        \mathbf{H}(r,\theta) = \frac{H_{0}\cos(\theta)}{2}\del{1 + \frac{a^{2}}{r^2}} \uvecs{r} - \frac{H_{0}\sin(\theta)}{2} \del{1 - \frac{a^{2}}{r^2}} \uvecs{\theta}
    \end{align}
    \end{parts}
    
  \end{solution}

\end{questions}

\end{document}