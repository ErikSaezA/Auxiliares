\documentclass[
  11pt,
  letterpaper,
   addpoints,
  answers
  ]{exam}

% Carga el preámbulo localizado en la carpeta superior
\NeedsTeXFormat{LaTeX2e}[2023/04/30]

% Provide the name of your page, the date it was last updated, and a comment about what it's used for
\ProvidesPackage{../exercise-preamble}[2023/04/30 Prof. Cassanelli custom LaTeX style]

% \usepackage{printlen}
% \uselengthunit{in}\printlength{\textwidth}

% PACKAGES
\usepackage[dvipsnames]{xcolor}

\usepackage{graphicx}
\graphicspath{{../figures}}
\usepackage{amsmath,amsthm,amssymb,mathtools,mathrsfs}
\usepackage{commath}
\usepackage{upgreek}
\usepackage{cancel}
\usepackage{enumerate}
\usepackage[font=small]{caption}
\usepackage[normalem]{ulem}
\usepackage{steinmetz}

\usepackage[left=1.5cm, right=1.5cm, top=1cm]{geometry}

% REFERENCES AND OTHERS
\usepackage{../aas_macros}
\usepackage{natbib}
\bibpunct{(}{)}{;}{a}{}{,}

\usepackage{tikz}
\usepackage{tikz-3dplot}
\usepackage{circuitikz}
\usepackage{pgfplots}
\pgfplotsset{compat=1.15}
\usepgfplotslibrary{smithchart}
\usetikzlibrary{
  decorations.pathmorphing,
  decorations.markings,
  calc,
  patterns,
  decorations,
  angles,
  quotes,
  ext.topaths.arcthrough,
  shapes
  }

\usepackage{siunitx}
\sisetup{
    range-phrase=\text{--},
    range-units=single,
    separate-uncertainty=true,
    print-unity-mantissa=false
    }
\DeclareSIUnit{\gauss}{G}
\DeclareSIUnit{\jansky}{Jy}

\newcommand{\iu}{\mathrm{i}\mkern1mu}
\newcommand{\ju}{\mathrm{j}\mkern1mu}
\newcommand{\euler}{\mathrm{e}}
\newcommand{\exponential}[1]{\mathrm{exp}\left[#1\right]}
\newcommand{\uvec}[1]{\widehat{\mathbf{#1}}}
\newcommand{\uvecs}[1]{\boldsymbol{\widehat{#1}}}
\newcommand{\bvec}[1]{\boldsymbol{\mathcal{#1}}}

\usepackage{hyperref}
\hypersetup{
    % bookmarks=true,
    unicode=true,
    pdftoolbar=true,
    pdfmenubar=true,
    pdffitwindow=false,
    pdfstartview={FitH},
    pdftitle={EL3103},
    pdfauthor={Tomas Cassanelli},
    pdfcreator={Tomas Cassanelli},
    pdfnewwindow=true,
    colorlinks=true,
    linkcolor=Violet,
    citecolor=Violet,
    urlcolor=Violet
    }

% Exam document class
\renewcommand{\figurename}{Figura}
\renewcommand{\tablename}{Cuadro}
\pagestyle{empty}

\usepackage[spanish]{cleveref}

\crefname{question}{\protect{pregunta}}{\protect{preguntas}}
\Crefname{question}{\protect{Pregunta}}{\protect{Preguntas}}
\creflabelformat{question}{#2{#1}#3}

\renewcommand{\solutiontitle}{\noindent\textbf{Solución:}\par\noindent}
\bracketedpoints
\pointname{~puntos}

\endinput

% Paquetes locales
\usepackage{float}
\usepackage{booktabs} % para \toprule, \midrule, \bottomrule
\usepackage{xcolor} % para colores

% Macros locales
\newcommand{\Rel}{\mathfrak{R}} % símbolo para la reluctancia

\begin{document}

\noindent
\begin{minipage}{0.47\textwidth}
\includegraphics[width=\textwidth]{../fcfm_die}
\end{minipage}
\begin{minipage}{0.53\textwidth}
\begin{center} 
\large\textbf{Análisis de Sistemas Dinámicos y Estimación} (EL3204-1) \\
\large\textbf{Clase auxiliar 4} \\
\normalsize Prof.~ Marcos Orchard - Sebastián Espinosa.\\
\normalsize Prof.~Aux.~Erik Sáez
\end{center}
\end{minipage}

\vspace{0.5cm}
\noindent
\vspace{.85cm}

\begin{questions}
    %%%%%%%%%%%%%%%%%%%%%%%%%%%
  \question Considere el sistema en tiempo continuo dado por
  \begin{equation}
      \ddot y(t) - 2\dot y(t) - 8 y(t) = 3\dot u(t) + 3 u(t),
      \label{eq:plant}
  \end{equation}
  donde $u(t)$ es la entrada y $y(t)$ la salida.

  \begin{parts}
    \part Obtenga la función de transferencia $G(s)=\dfrac{Y(s)}{U(s)}$ del sistema (condiciones iniciales nulas).
    \part A partir de $G(s)$, proponga una representación en espacio de estados $(\dot x = Ax + Bu,\; y = Cx + Du)$ en forma controlable.
    \part Determine si el sistema es controlable y observable. Justifique mediante los rangos de las matrices de controlabilidad y observabilidad.
    \part Diseñe un controlador por realimentación de estados $u = -Kx + r$ que ubique los polos a lazo cerrado en $s=-5$ y $s=-3$. Indique el vector $K$.
    \part Suponga ahora que sólo se mide la salida $y(t)$ y no el estado completo. Diseñe un compensador dinámico (controlador con observador de estado) que mantenga los polos de lazo cerrado en $-5$ y $-3$. Indique los polos del observador y el vector de ganancias $L$.
  \end{parts}
  % Comienzo de soluciones dentro del entorno questions
\begin{solution}
  \subsection*{Resolución 1.1}
  Dado que estamos trabajando en el dominio temporal continuo, pasamos al dominio de Laplace. Aplicando transformada de Laplace (CI nulas) a \eqref{eq:plant}:
  \begin{align}
      s^2 Y(s) - 2s Y(s) - 8 Y(s) &= 3s U(s) + 3 U(s) \nonumber \\
      (s^2 - 2s - 8)Y(s) &= (3s + 3)U(s). \label{eq:plant_laplace}\\
      \Longrightarrow \quad G(s)=\frac{Y(s)}{U(s)} &= \frac{3s+3}{s^{2}-2s-8}. \nonumber
  \end{align}
Luego debemos aplicar fracciones parciales para escribir $G(s)$ como suma de términos simples. Factorizamos el denominador: $s^{2}-2s-8 = (s-4)(s+2)$. Usamos fracciones parciales
\begin{align}
      G(s)=\frac{\alpha}{s-4}+\frac{\beta}{s+2} = \frac{\alpha(s+2)+\beta(s-4)}{(s-4)(s+2)}= \frac{3s+3}{(s-4)(s+2)}
\end{align}
De esta manera tenemos el siguiente sistema de ecuaciones para el numerador:
\begin{align}
  \alpha(s+2)+\beta(s-4) &= 3s+3 \nonumber \\
  (\alpha+\beta)s + (2\alpha - 4\beta) &= 3s + 3 \nonumber \\
\end{align}
De esta manera tenemos que:
\begin{align}
  \alpha + \beta &= 3 \label{eq:sistema1} \\
  2\alpha - 4\beta &= 3 \label{eq:sistema2}
\end{align}
Con lo que finalmente se tiene que $\alpha = \tfrac{5}{2}$ y $\beta = \tfrac{1}{2}$. Por lo tanto:
\begin{align}
  G(s) = \left(\frac{5}{2}\right)\frac{1}{s-4} + \left(\frac{1}{2}\right)\frac{1}{s+2}
\end{align}
Así se obtiene la función de transferencia del sistema.

  \subsection*{Resolución 1.2}
Recordando que es posible escribir $G(s)= \frac{Y(s)}{U(s)}$, la salida será:
\begin{align}
      Y(s) = G(s)U(s) = \frac{3s+3}{s^{2}-2s-8}U(s) = \left(\frac{5}{2}\right)\frac{U(s)}{s-4} + \left(\frac{1}{2}\right)\frac{U(s)}{s+2}
\end{align}
De esta manera es posible escribir el vector $C$ de forma conveniente dada por:
\begin{align}
  Y(s) = (1 \quad 1) \begin{pmatrix} \left(\frac{5}{2}\right)\frac{U(s)}{s-4} \\ \left(\frac{1}{2}\right)\frac{U(s)}{s+2} \end{pmatrix}
\end{align}
Notar que tambien podriamos formularlo como:
\begin{align}
  Y(s) = \begin{pmatrix} \frac{5}{2} & \frac{1}{2}\end{pmatrix} \begin{pmatrix} \frac{U(s)}{s-4} \\ \frac{U(s)}{s+2} \end{pmatrix}
\end{align}
Es equivalente dado que los polos nos damos que unicamente se ponderan pero no cambian su valor, por lo que la estabilidad no se ve afectada. En este caso se escoge la primer forma. Definiendo:
\begin{align}
  C= \begin{pmatrix} 1 & 1 \end{pmatrix}, \quad X(s) = \begin{pmatrix} X_1(s) \\ X_2(s) \end{pmatrix} = \begin{pmatrix} \left(\frac{5}{2}\right)\frac{U(s)}{s-4} \\ \left(\frac{1}{2}\right)\frac{U(s)}{s+2} \end{pmatrix}
\end{align}
De esta manera tendremos que:
\begin{align}
  X_1(s) = \left(\frac{5}{2}\right)\frac{U(s)}{s-4} \quad &\Longrightarrow \quad sX_1(s) - 4X_1(s) = \left(\frac{5}{2}\right)U(s) \quad \Longrightarrow \quad \dot{x}_1(t) = 4x_1(t) + \left(\frac{5}{2}\right)u(t) \\
\end{align}
Análogamente, para $X_2(s)$:
\begin{align}
  X_2(s) = \left(\frac{1}{2}\right)\frac{U(s)}{s+2} \quad &\Longrightarrow \quad sX_2(s) + 2X_2(s) = \left(\frac{1}{2}\right)U(s) \quad \Longrightarrow \quad \dot{x}_2(t) = -2x_2(t) + \left(\frac{1}{2}\right)u(t)
\end{align}
Con lo que finalmente es posible formular la representación en espacio de estados del sistema como:
\begin{align}
  \dot{x}_1(t) &= 4x_1(t) + \left(\frac{5}{2}\right)u(t) \\
  \dot{x}_2(t) &= -2x_2(t) + \left(\frac{1}{2}\right)u(t) \\
\end{align}
Donde en forma matricial se tendra que:
\begin{align}
  \begin{pmatrix}
    \dot{x}_1(t) \\ \dot{x}_2(t)
  \end{pmatrix}
  &=
  \begin{pmatrix}
    4 & 0 \\
    0 & -2
  \end{pmatrix}
  \begin{pmatrix}
    x_1(t) \\ x_2(t)
  \end{pmatrix}
  +
  \begin{pmatrix}
   5/2 \\
    1/2
  \end{pmatrix}
  u(t) 
\end{align}

Donde reconocemos las matrices $A$ y $B$ dadas por:
\begin{align}
  A &= \begin{pmatrix}
    4 & 0 \\
    0 & -2
  \end{pmatrix}, \quad
  B = \begin{pmatrix}
    5/2 \\ 1/2
  \end{pmatrix}
\end{align}

Por otro lado, tendremos que la salida del sistema será:
\begin{align}
  y(t) &= \begin{pmatrix} 1 & 1 \end{pmatrix} \begin{pmatrix} x_1(t) \\ x_2(t) \end{pmatrix} + 0 \cdot u(t) \\
  &= \begin{pmatrix} 1 & 1 \end{pmatrix} x(t) 
\end{align}
  

  \subsection*{Resolución 1.3}
Recordemos que para determinar si un sistema es controlable u observable, debemos calcular las matrices de controlabilidad y observabilidad respectivamente y luego determinar sus rangos. La matriz de controlabilidad viene dada por:
\begin{align}
  \mathcal C = \begin{pmatrix} B & AB & A^2B & \cdots & A^{n-1}B \end{pmatrix}
\end{align}
Que para nuestro caso particular tenemos que $n=2$, por lo que:
\begin{align}
  \mathcal C = \begin{pmatrix} B & AB \end{pmatrix}
\end{align}
Luego, calculamos $AB$:
\begin{align}
  AB = A \cdot B = \begin{pmatrix} 4 & 0 \\ 0 & -2 \end{pmatrix} \cdot \begin{pmatrix} 5/2 \\ 1/2 \end{pmatrix} = \begin{pmatrix} 4 \cdot (5/2) + 0 \cdot (1/2) \\ 0 \cdot (5/2) + (-2) \cdot (1/2) \end{pmatrix} = \begin{pmatrix} 10 \\ -1 \end{pmatrix}
\end{align}
Por lo que finalmente tenemos que:
\begin{align}
  \mathcal C = \begin{pmatrix} B & AB \end{pmatrix} = \begin{pmatrix} 5/2 & 10 \\ 1/2 & -1 \end{pmatrix}
\end{align}
Vemos que la matriz es Linealmente independiente tanto en columnas como en filas, por lo que su rango es 2. Por lo tanto el sistema es controlable. Otra forma de verificar esto es calculando el determinante de la matriz y este debe ser diferente de cero. Calculamos:
\begin{align}
  \det(\mathcal C) = \left( \frac{5}{2} \cdot (-1) \right) - (10 \cdot \frac{1}{2}) = -\frac{5}{2} - 5 = -\frac{15}{2} \neq 0
\end{align}
Por lo que el sistema es controlable. Luego analizamos la matriz de observabilidad, la cual viene dada por:
\begin{align}
  \mathcal O = \begin{pmatrix} C \\ CA \\ CA^2 \\ \vdots \\ CA^{n-1} \end{pmatrix}
\end{align}
Que para nuestro caso particular tenemos que $n=2$, por lo que:
\begin{align}
  \mathcal O = \begin{pmatrix} C \\ CA \end{pmatrix}
\end{align}
Calculamos $CA$:
\begin{align}
  CA = C \cdot A = \begin{pmatrix} 1 & 1 \end{pmatrix} \cdot \begin{pmatrix} 4 & 0 \\ 0 & -2 \end{pmatrix} = \begin{pmatrix} 1 \cdot 4 + 1 \cdot 0 & 1 \cdot 0 + 1 \cdot (-2) \end{pmatrix} = \begin{pmatrix} 4 & -2 \end{pmatrix}
\end{align}
Por lo que finalmente tenemos que:
\begin{align}
  \mathcal O = \begin{pmatrix} C \\ CA \end{pmatrix} = \begin{pmatrix} 1 & 1 \\ 4 & -2 \end{pmatrix}
\end{align}
Nuevamente vemos que la matriz es Linealmente independiente tanto en columnas como en filas, por lo que su rango es 2. Por lo tanto el sistema es observable. Otra forma de verificar esto es calculando el determinante de la matriz y este debe ser diferente de cero. Calculamos:
\begin{align}
  \det(\mathcal O) = (1 \cdot (-2)) - (1 \cdot 4) = -2 - 4 = -6 \neq 0
\end{align}
Por lo que el sistema es observable.
  \subsection*{Resolución 1.4}
  Dado que verificamos previamente que el sistema es controlable, podemos proceder a diseñar un controlador por realimentación de estados. La ley de control es $u = r - Kx$, donde $r$ es la referencia (aquí se asume constante) y $K$ es el vector de ganancias a determinar. El sistema en lazo cerrado queda
  Se busca $K$ tal que los polos de $A-BK$ estén en $-5$ y $-3$. Esta metodología se utiliza porque anteriormente obtuvimos que los polos del sistema eran $4$ y $-2$; el 4 es positivo y produce inestabilidad. Mediante esta realimentación se busca estabilizar el sistema. El sistema original se puede ver mediante el diagrama de bloques:
  \begin{figure}[H]\centering
    \includegraphics[width=.9\textwidth]{../figures/Auxiliar_4_1.png}
    \caption{El diagrama muestra la estructura estándar de un sistema lineal en espacio de estados: la entrada $u(t)$ es multiplicada por $B$ y sumada al término asociado a la dinámica interna $Ax(t)$ antes de integrarse (bloque $\tfrac{1}{s}$ que representa $\int$) para generar el estado $x(t)$. La salida se obtiene aplicando $C$ a $x(t)$. La realimentación con $A$ en el lazo interno ilustra simbólicamente la contribución de la dinámica interna $Ax$ al acumulador. De manera conceptual, cada rama representa uno de los componentes del estado $x(t)$.}
  \end{figure} 
Por otro lado, dado que ajustaremos la entrada $u$ mediante la realimentación de estados, el sistema en lazo cerrado se puede ver como:
 \begin{figure}[H]\centering
    \includegraphics[width=.9\textwidth]{../figures/Auxiliar_4_2.png}
    \caption{El bloque punteado corresponde al observador que genera $\hat x(t)$ a partir de $u(t)$ y $y(t)$. La señal de innovación $e_y(t)= y(t)-\hat y(t)$ (comparador a la derecha) se amplifica por $F$ y se inyecta en la misma estructura dinámica $(A, B)$ que replica el modelo interno. Esto corrige continuamente la estimación: si $\hat y$ es menor que $y$, el término $F e_y$ empuja la trayectoria estimada en la dirección adecuada. Afuera del bloque, el controlador usa la estimación (en lugar del estado real) para formar la acción $u = r - K\hat x$.}
  \end{figure}
  Volviendo sobre nuestra representación en espacio de estados, tenemos:
  \begin{align}
      \dot x &= Ax + Bu = Ax + B(r - Kx) = (A-BK)x + Br \\
      y &= Cx + Du = Cx + 0(r - Kx) = Cx
  \end{align}
  Por lo que el sistema en lazo cerrado queda $\dot x = (A-BK)x + Br$, $y = Cx$. La estabilidad del sistema en lazo cerrado está dada por los autovalores de $A-BK$ que denominaremos $\hat{A}$, notamos que el vector K es de la forma $K = (k_1 \; k_2)$, y al ser una variable libre podemos elegir sus valores para ubicar los polos de $\hat{A}$ en las posiciones deseadas., por lo tanto se obtienen primeramente los polos dados por:
  \begin{align}
    \det(A-BK -\lambda I) &= 0\\
    \lambda^{2} + \left(-2 + \frac{5}{2}k_1 + \frac{1}{2}k_2\right)\lambda + \left(-8 + 5k_1 - 2k_2\right) &= 0
  \end{align}
  Estos valores obtenidos deben ser iguales a los polos deseados, que en este caso son $-5$ y $-3$. Por lo que el polinomio característico deseado es:
  \begin{align}
    (\lambda + 5)(\lambda + 3) = \lambda^{2} + 8\lambda + 15
  \end{align}
  Igualando los coeficientes de ambos polinomios, se obtiene el siguiente sistema de ecuaciones:
  \begin{align}
    -2 + \frac{5}{2}k_1 + \frac{1}{2}k_2 &= 8 \\
    -8 + 5k_1 - 2k_2 &= 15
  \end{align}
  Resolviendo el sistema de ecuaciones, se obtiene que $k_1 = \frac{21}{5}$ y $k_2 = -1$. Por lo tanto, el vector de ganancias $K$ es:
  \begin{align}
    K = \left( \frac{21}{5} \; -1 \right)
  \end{align}
  Con lo que finalmente vemos el vector de ganancias $K$ que ubica los polos del sistema en lazo cerrado en $-5$ y $-3$. La ley de control es $u = r - Kx$.

  \subsection*{Resolución 1.5}
  En el análisis que sigue debemos cuidar la idea central que se quiere transmitir. Ahora no tenemos acceso al estado completo, sólo a la salida $y$. La pregunta natural es: ¿cómo estimar $x$ a partir de $y$? La respuesta es usar un observador de estados. El observador de Luenberger replica la dinámica del sistema original e incorpora la información de la salida medida para corregir la estimación. Se diseña de modo que el error entre el estado real y el estimado converge a cero con el tiempo, garantizando una estimación precisa. Matemáticamente:
  \begin{align}
    \dot{\hat x} &= A\hat x + Bu + F(y - \hat y) \\
    \hat y &= C\hat x
  \end{align}
  Donde $\hat x$ es la estimación del estado, $\hat y$ es la salida estimada, y $F$ es el vector de ganancias del observador a determinar. Esto, en un diagrama de bloques, se observa como:
    \begin{figure}[H]\centering
    \includegraphics[width=.9\textwidth]{../figures/Auxiliar_4_3.png}
    \caption{El observador replica la ecuación $\dot{\hat x}=A\hat x + Bu + F(y-\hat y)$. La rama principal implementa el modelo del sistema ($A$, $B$ y el integrador). La rama vertical superior lleva la salida estimada $\hat y = C\hat x$ al comparador donde se obtiene el error de salida $e_y$. Ese error, tras el bloque de ganancia $F$, se suma como término corrector que desplaza los polos de la dinámica del error a posiciones elegidas (más rápidos), garantizando convergencia de $\hat x$ hacia $x$. De este modo el diseño de $F$ es independiente del de $K$, ilustrando el principio de separación.}
  \end{figure}
El error estará dado por:
  \begin{align}
    e = \hat x - x
  \end{align}
  Y su dinámica estará dada por:
  \begin{align}
    \dot{e}(t) =& A\hat{x(t)} + Bu(t) + F(y(t) - \hat{y}(t)) - (Ax(t) + Bu(t)) \\
    =& A(\hat{x(t)} - x(t)) + F(Cx(t) - C\hat{x(t)})\\
     &= (A - FC)e(t)
  \end{align}
  La dinámica del error depende de la matriz $A - FC$. Para garantizar que el error converja a cero es necesario que los autovalores de $A - FC$ tengan parte real negativa. Esto se logra eligiendo adecuadamente $F$. Como el sistema es observable podemos ubicar los polos de $A - FC$ libremente. Usualmente se eligen más rápidos ( tres veces más negativos) que los del controlador para asegurar convergencia rápida. Procedemos a calcular $F$ para ubicar polos en $-15$ y $-9$. Sea
  \begin{align}
    F = \begin{pmatrix} f_1 \\ f_2 \end{pmatrix}
  \end{align}
  Lo que permite que obtener los polos de la matriz como:
  \begin{align}
    \det(sI - (A - FC)) = \lambda^{2} - (2 - (f_1 + f_2))\lambda + (-8 + 2f_1 - 4f_2) = 0
  \end{align}
  De esta manera dado que queremos ubicar los polos en $-15$ y $-9$, el polinomio característico deseado es:
  \begin{align}
    (\lambda + 15)(\lambda + 9) = \lambda^{2} + 24\lambda + 135
  \end{align}
  Igualando los coeficientes de ambos polinomios, se obtiene el siguiente sistema de ecuaciones:
  \begin{align}
    -(2 - (f_1 + f_2)) &= 24 \\
    -8 + 2f_1 - 4f_2 &= 135
  \end{align}
  Resolviendo el sistema de ecuaciones, se obtiene que $f_1 = \frac{247}{6}$ y $f_2 = -\frac{91}{6}$. Por lo tanto, el vector de ganancias $F$ es:
  \begin{align}
    F = \begin{pmatrix} \frac{247}{6} \\ -\frac{91}{6} \end{pmatrix}
  \end{align}
  % Explicación ampliada del principio de separación
   El controlador por realimentación de estados diseña $K$ para colocar los polos de la matriz $A-BK$. Cuando agregamos un observador de Luenberger, la dinámica conjunta (control + observador) en las coordenadas $x$ (estado real) y $e = \hat x - x$ (error de estimación) queda
  \[
     \dot x = (A-BK)x + Br, \qquad \dot e = (A-FC)e.
  \]
  Obsérvese que $e$ no aparece en la ecuación de $\dot x$ (no hay términos cruzados). Esto significa que la evolución del estado real (y, por ende, del lazo de control) no depende de cómo transcurre transitoriamente el error del observador: mientras $K$ estabilice/ubique los polos deseados de $(A-BK)$, esa parte permanecerá válida. En paralelo, $F$ se elige para que $(A-FC)$ haga que $e(t)\to 0$ rápidamente. Así, el diseño se separa en dos problemas independientes: (i) colocación de polos de control y (ii) colocación de polos del observador.

  \end{solution}
%----------------------------  Segunda pregunta
  \question Considere un motor eléctrico DC que impulsa un carrito, como se muestra en la figura.
  \begin{center}
    \includegraphics[width=.75\textwidth]{../figures/Auxiliar_4_4.png}
  \end{center}
  Suponga que los parámetros del sistema son:
  \[
     k_m = 1\;\text{Nm/A},\quad k_e = 1\;\text{Vs},\quad R_a = 1\,\Omega,\quad L_a = 1\,\text{H},        \\
     J = 0.1\,\text{kgm}^2,\quad B = 0.2\,\text{Nms}, \quad k_g = 0.01\,\text{m/rad}.
  \]
  \begin{parts}
    \part Formule un modelo dinámico del sistema en variables de estado, indicando claramente las hipótesis simplificatorias (por ejemplo: juego de engranajes ideal, sin pérdidas; fricción viscosa lineal; flujo magnético constante, etc.).
    \part Encuentre la función de transferencia (MTE) desde el voltaje de armadura $v_a(t)$ hasta la velocidad lineal $\dot z(t)$ del carrito.
    \part Determine si el sistema es estable (según los polos de la MTE / matriz $A$).
    \part Obtenga la respuesta al impulso de la salida $\dot z(t)$.
    \part Exprese la respuesta del sistema en un tiempo arbitrario $t$ para condiciones iniciales y entrada arbitraria (solución general usando convolución o solución de estado).
    \part Determine si el sistema es completamente controlable y observable (especifique la elección de estados utilizada).
    \part En caso de ser observable, diseñe un observador cuyos polos se ubiquen en $-10$ y $-5$.
  \end{parts}
% (Sin solución incluida para esta pregunta de momento.)
\end{questions}
\begin{solution}
\subsection*{Resolución 2.1}
Para modelar el sistema motor DC - carrito, consideramos un motor de corriente continua con armadura controlada que mueve un carrito a través de un sistema de engranajes. Suponiendo una carga fija, podemos modelar el circuito eléctrico de la armadura de la siguiente manera:

\begin{figure}[H]
\centering
\begin{circuitikz}[american voltages, european resistors]
  \draw
    (0,0) to[sV, l_={$v_a(t)$}] (0,3)    % fuente de armadura
           to[R,  l={$R_a$}]     (3,3)    % resistencia
           to[L,  l={$L_a$}]     (6,3)    % inductancia
           to[V,  l_={$e(t)=k_e\,\omega(t)$}] (6,0) % fem de retroceso (+ arriba)
           -- (0,0);                        % cierre del lazo

  % Anotamos la corriente de armadura encima del resistor (opcional)
  \draw[->] (3.3,3.25) -- (3.8,3.25) node[midway, above] {$i_a(t)$};
\end{circuitikz}
\caption{Modelo eléctrico de la armadura con fuerza contraelectromotriz $e(t)=k_e\,\omega(t)$.}
\label{fig:armadura}
\end{figure}

El modelo eléctrico de la armadura, mostrado en la Fig.~\ref{fig:armadura}, se obtiene aplicando la ley de tensiones de Kirchhoff:
\begin{align}
v_a(t) &= R_a\,i_a(t) + L_a\,\frac{di_a(t)}{dt} + k_e\,\omega(t)
\end{align}
donde $k_e\,\omega(t)$ representa la fuerza contraelectromotriz generada por la rotación del motor.

Despejando la derivada de la corriente:
\begin{align}
\frac{di_a(t)}{dt} &= -\frac{R_a}{L_a}\,i_a(t) - \frac{k_e}{L_a}\,\omega(t) + \frac{1}{L_a}\,v_a(t)
\end{align}
Por otro lado, para la parte mecánica consideramos que el torque generado por el motor es proporcional a la corriente de armadura:
\begin{align}
\tau_m(t) = k_m\,i_a(t)
\end{align}
donde $k_m$ es la constante de torque del motor.

Aplicando la segunda ley de Newton para rotación al eje del motor, el torque neto es igual a la suma de los torques de inercia y fricción viscosa:
\begin{align}
\tau_m(t) &= J\,\frac{d\omega(t)}{dt} + B\,\omega(t)
\end{align}
donde $J$ es el momento de inercia del sistema rotativo y $B$ es el coeficiente de fricción viscosa.

Sustituyendo la expresión del torque del motor:
\begin{align}
k_m\,i_a(t) &= J\,\frac{d\omega(t)}{dt} + B\,\omega(t)
\end{align}

Despejando la derivada de la velocidad angular:
\begin{align}
\frac{d\omega(t)}{dt} &= -\frac{B}{J}\,\omega(t) + \frac{k_m}{J}\,i_a(t)
\end{align}
Luego, tomando las dos ecuaciones diferenciales obtenidas, tenemos el sistema de ecuaciones que describe la dinámica del motor DC:
\begin{align}
 \frac{d\omega(t)}{dt} &= -\frac{B}{J}\,\omega(t) + \frac{k_m}{J}\,i_a(t) \\
  \frac{di_a(t)}{dt} &= -\frac{R_a}{L_a}\,i_a(t) - \frac{k_e}{L_a}\,\omega(t) + \frac{1}{L_a}\,v_a(t)
\end{align}

Definimos el vector de estado y la entrada como:
\begin{equation}
  x(t) = \begin{bmatrix}\omega(t)\\ i_a(t)\end{bmatrix}, \qquad
  u(t) = v_a(t)
\end{equation}

De esta manera, el sistema se puede escribir en la forma estándar de espacio de estados $\dot{x}(t) = Ax(t) + Bu(t)$, donde las matrices $A$ y $B$ quedan definidas como:
\begin{equation}
  \dot{x}(t) = 
  \underbrace{\begin{bmatrix}
    -\dfrac{B}{J} & \dfrac{k_m}{J}\\[6pt]
    -\dfrac{k_e}{L_a} & -\dfrac{R_a}{L_a}
  \end{bmatrix}}_{A}\,x(t)
  +
  \underbrace{\begin{bmatrix}
    0\\[2pt]\dfrac{1}{L_a}
  \end{bmatrix}}_{B}\,u(t)
\end{equation}
La salida requerida es la velocidad lineal del carrito, que es proporcional a la velocidad angular del eje del motor a través de la relación de engranajes:
\begin{align}
  y(t) = \dot{z}(t) = k_g\,\omega(t)
\end{align}
donde $k_g$ es la constante de proporcionalidad que relaciona la velocidad angular del motor con la velocidad lineal del carrito.

Por lo tanto, la ecuación de salida en forma matricial es:
\begin{align}
  y(t) &= \underbrace{\begin{bmatrix}k_g & 0\end{bmatrix}}_{C}\,x(t) + \underbrace{0}_{D}\,u(t) \\
  &= \begin{bmatrix} k_g & 0 \end{bmatrix} \begin{bmatrix} \omega(t)\\ i_a(t) \end{bmatrix}
\end{align}
Algunas hipótesis simplificatorias consideradas en este modelo son:
\begin{itemize}
  \item Sistema de engranajes ideal, sin pérdidas por fricción, juego mecánico o elasticidad.
  \item Fricción viscosa lineal, donde el torque de fricción es proporcional a la velocidad angular.
  \item Flujo magnético constante y ausencia de saturación magnética.
  \item El motor opera dentro de sus límites nominales en región lineal.
  \item El carrito se mueve en una superficie plana sin inclinación.
  \item No hay fuerzas externas como viento, vibraciones o cargas variables.
  \item Todas las constantes del sistema ($R_a$, $L_a$, $J$, $B$, $k_m$, $k_e$, $k_g$) son valores fijos.
  \item Modelo de parámetros concentrados, despreciando efectos distribuidos.
\end{itemize}
\subsection*{Resolución 2.2}

Sustituyendo los parámetros numéricos dados: $k_m=1$, $k_e=1$, $R_a=1~\Omega$, $L_a=1~\mathrm{H}$, $J=0.1~\mathrm{kg\,m^2}$, $B=0.2~\mathrm{N\,m\,s}$ y $k_g=0.01~\mathrm{m/rad}$, las matrices del sistema quedan:

\begin{align}
A&=
\begin{bmatrix}
-\dfrac{B}{J} & \dfrac{k_m}{J}\\[4pt]
-\dfrac{k_e}{L_a} & -\dfrac{R_a}{L_a}
\end{bmatrix}
=
\begin{bmatrix}
-2 & 10\\
-1 & -1
\end{bmatrix},\qquad
B=
\begin{bmatrix}
0\\[2pt] \dfrac{1}{L_a}
\end{bmatrix}
=
\begin{bmatrix}
0\\ 1
\end{bmatrix},\qquad
C=\begin{bmatrix} k_g & 0\end{bmatrix}=\begin{bmatrix}0.01 & 0\end{bmatrix}.
\end{align}

Para calcular la matriz de transición $\Phi(t) = e^{At}$, observamos que la matriz $A$ no es diagonal, pero sí es diagonalizable. Utilizaremos la descomposición $A=TDT^{-1}$, donde $D$ es la matriz diagonal de autovalores y $T$ la matriz de autovectores. Primero calculamos el polinomio característico de $A$:
\begin{align}
\det(\lambda I-A) &= \det\begin{bmatrix} \lambda+2 & -10 \\ 1 & \lambda+1 \end{bmatrix} \\
&= (\lambda+2)(\lambda+1) - (-10)(1) \\
&= \lambda^2 + 3\lambda + 2 + 10 = \lambda^2+3\lambda+12
\end{align}

Resolviendo la ecuación característica $\lambda^2+3\lambda+12=0$ usando la fórmula cuadrática:
\begin{align}
\lambda_{1,2} &= \frac{-3 \pm \sqrt{9-48}}{2} = \frac{-3 \pm \sqrt{-39}}{2} = -\frac{3}{2} \pm j\frac{\sqrt{39}}{2}
\end{align}

Por lo tanto, los autovalores son:
\begin{align}
  \lambda_1 &= -1.5 + j\frac{\sqrt{39}}{2}, \quad \lambda_2 = -1.5 - j\frac{\sqrt{39}}{2}
\end{align}
Por lo tanto, los autovalores son:
\begin{align}
  \lambda_1 &= -1.5 + j\frac{\sqrt{39}}{2}, \quad \lambda_2 = -1.5 - j\frac{\sqrt{39}}{2}
\end{align}

La matriz diagonal de autovalores es:
\begin{equation}
  D = \begin{pmatrix} \lambda_1 & 0 \\ 0 & \lambda_2 \end{pmatrix} = \operatorname{diag}(\lambda_1,\lambda_2)
\end{equation}

Ahora calculamos los autovectores. Para el autovalor $\lambda_1$, resolvemos $(A - \lambda_1 I)v_1 = 0$:
\begin{align}
  \begin{bmatrix}
    -2 - \lambda_1 & 10 \\
    -1 & -1 - \lambda_1
  \end{bmatrix}
  \begin{bmatrix}
    v_{11} \\ v_{12}
  \end{bmatrix}
  = \begin{bmatrix}
    0 \\ 0
  \end{bmatrix}
\end{align}

Sustituyendo $\lambda_1 = -1.5 + j\frac{\sqrt{39}}{2}$:
\begin{align}
  \begin{bmatrix}
    -0.5 - j\frac{\sqrt{39}}{2} & 10 \\
    -1 & 0.5 - j\frac{\sqrt{39}}{2}
  \end{bmatrix}
  \begin{bmatrix}
    v_{11} \\ v_{12}
  \end{bmatrix}
  = \begin{bmatrix}
    0 \\ 0
  \end{bmatrix}
\end{align}

De la primera ecuación: $\left(-0.5 - j\frac{\sqrt{39}}{2}\right)v_{11} + 10v_{12} = 0$, despejamos:
\begin{align}
v_{12} = \frac{0.5 + j\frac{\sqrt{39}}{2}}{10}v_{11} = \left(0.05 + j\frac{\sqrt{39}}{20}\right)v_{11}
\end{align}

Tomando $v_{11}=1$, obtenemos el autovector:
\begin{align}
  v_1 = \begin{bmatrix} 1 \\ 0.05 + j\frac{\sqrt{39}}{20} \end{bmatrix}
\end{align}
Tomando $v_{11}=1$, obtenemos el autovector:
\begin{align}
  v_1 = \begin{bmatrix} 1 \\ 0.05 + j\frac{\sqrt{39}}{20} \end{bmatrix}
\end{align}

Para $\lambda_2$ (conjugado de $\lambda_1$), el autovector correspondiente será el conjugado de $v_1$:
\begin{align}
v_2 = \begin{bmatrix} 1 \\ 0.05 - j\frac{\sqrt{39}}{20} \end{bmatrix}
\end{align}

Definimos $a = 0.05 + j\frac{\sqrt{39}}{20}$ para simplificar la notación. Entonces las matrices de transformación son:
\begin{align}
T &= [v_1 \; v_2] = \begin{bmatrix} 1 & 1 \\ a & a^* \end{bmatrix}
\end{align}

Para calcular $T^{-1}$, utilizamos la fórmula para matrices $2 \times 2$:
\begin{align}
T^{-1} &= \frac{1}{\det(T)} \begin{bmatrix} a^* & -1 \\ -a & 1 \end{bmatrix}
\end{align}

El determinante es: $\det(T) = 1 \cdot a^* - 1 \cdot a = a^* - a = -j\frac{\sqrt{39}}{10}$, por lo que:
\begin{align}
T^{-1} &= \frac{10j}{\sqrt{39}} \begin{bmatrix} a^* & -1 \\ -a & 1 \end{bmatrix}
\end{align}

La matriz de transición de estados se obtiene mediante:
\begin{equation}
\Phi(t) = e^{At} = T\,e^{Dt}\,T^{-1}
\end{equation}

donde $e^{Dt}$ es la matriz diagonal:
\begin{align}
  e^{Dt} &= \begin{pmatrix} e^{\lambda_1 t} & 0 \\ 0 & e^{\lambda_2 t} \end{pmatrix} = \begin{pmatrix} e^{(-1.5 + j\frac{\sqrt{39}}{2})t} & 0 \\ 0 & e^{(-1.5 - j\frac{\sqrt{39}}{2})t} \end{pmatrix}
\end{align}

Realizando la multiplicación matricial $\Phi(t) = T e^{Dt} T^{-1}$ y factorizando $e^{-1.5t}$:
\begin{align}
\Phi(t) &= e^{-1.5t} \begin{bmatrix}
\cos\left(\frac{\sqrt{39}}{2}t\right) - \frac{1}{\sqrt{39}}\sin\left(\frac{\sqrt{39}}{2}t\right) & \frac{20}{\sqrt{39}}\sin\left(\frac{\sqrt{39}}{2}t\right) \\[6pt]
\frac{1}{10}\sin\left(\frac{\sqrt{39}}{2}t\right) & \cos\left(\frac{\sqrt{39}}{2}t\right) + \frac{1}{\sqrt{39}}\sin\left(\frac{\sqrt{39}}{2}t\right)
\end{bmatrix}
\end{align}

Esta expresión se obtiene utilizando las identidades de Euler para convertir las exponenciales complejas en funciones trigonométricas reales, aprovechando que los autovalores son complejos conjugados.
\subsection*{Resolución 2.3}

Para determinar la estabilidad del sistema, analizamos los autovalores de la matriz $A$. Un sistema lineal e invariante en el tiempo (LTI) continuo será asintóticamente estable si y sólo si todos los autovalores tienen parte real negativa:
\begin{equation}
\Re(\lambda_i) < 0 \quad \forall\,\lambda_i \in \mathrm{eig}(A)
\end{equation}

De los cálculos realizados en la resolución 2.2, obtuvimos que los autovalores de $A$ son:
\begin{equation}
\lambda_{1,2} = -\frac{3}{2} \pm j\,\frac{\sqrt{39}}{2}
\end{equation}

Analizando la parte real de ambos autovalores:
\begin{align}
\Re(\lambda_1) &= \Re\left(-\frac{3}{2} + j\,\frac{\sqrt{39}}{2}\right) = -\frac{3}{2} < 0 \\
\Re(\lambda_2) &= \Re\left(-\frac{3}{2} - j\,\frac{\sqrt{39}}{2}\right) = -\frac{3}{2} < 0
\end{align}

Como ambos autovalores tienen parte real negativa ($-1.5$), se cumple la condición de estabilidad asintótica. Esto significa que:
\begin{itemize}
\item La matriz de transición $\Phi(t) = e^{At} \to 0$ cuando $t \to \infty$
\item Cualquier condición inicial del sistema convergerá a cero en ausencia de entrada
\item El sistema es estable asintóticamente
\end{itemize}

Físicamente, esto indica que el motor DC con carrito retornará a su estado de equilibrio (velocidad cero) después de cualquier perturbación, con un comportamiento oscilatorio amortiguado debido a la parte imaginaria de los autovalores.
\subsection*{Resolución 2.4}

Para obtener la respuesta al impulso del sistema, necesitamos calcular la función de transferencia $H(s)$ y luego aplicar la transformada inversa de Laplace. La función de transferencia se define como:
\begin{equation}
H(s) = C(sI-A)^{-1}B
\end{equation}

Utilizando las matrices numéricas obtenidas anteriormente:
\begin{align}
A=\begin{bmatrix}-2&10\\-1&-1\end{bmatrix}, \quad
B=\begin{bmatrix}0\\1\end{bmatrix}, \quad
C=\begin{bmatrix}0.01&0\end{bmatrix}
\end{align}

Primero calculamos $(sI-A)$:
\begin{align}
sI-A &= \begin{bmatrix}s&0\\0&s\end{bmatrix} - \begin{bmatrix}-2&10\\-1&-1\end{bmatrix} = \begin{bmatrix}s+2&-10\\ 1&s+1\end{bmatrix}
\end{align}

Para calcular la inversa de una matriz $2 \times 2$, usamos la fórmula:
\begin{align}
(sI-A)^{-1} &= \frac{1}{\det(sI-A)} \begin{bmatrix}s+1&10\\ -1&s+2\end{bmatrix}
\end{align}

El determinante es:
\begin{align}
\det(sI-A) &= (s+2)(s+1) - (-10)(1) = s^2+3s+2+10 = s^2+3s+12
\end{align}

Por lo tanto:
\begin{align}
(sI-A)^{-1} &= \frac{1}{s^2+3s+12} \begin{bmatrix}s+1&10\\ -1&s+2\end{bmatrix}
\end{align}

Ahora calculamos la función de transferencia:
\begin{align}
H(s) &= C(sI-A)^{-1}B \\
&= \begin{bmatrix}0.01&0\end{bmatrix} \cdot \frac{1}{s^2+3s+12} \begin{bmatrix}s+1&10\\ -1&s+2\end{bmatrix} \cdot \begin{bmatrix}0\\1\end{bmatrix} \\
&= \frac{1}{s^2+3s+12} \begin{bmatrix}0.01&0\end{bmatrix} \begin{bmatrix}10\\ s+2\end{bmatrix} \\
&= \frac{0.01 \cdot 10}{s^2+3s+12} = \frac{0.1}{s^2+3s+12}
\end{align}

Para obtener la respuesta al impulso $h(t) = \mathcal{L}^{-1}\{H(s)\}$, utilizamos el método de fracciones parciales. Como ya conocemos los autovalores del sistema:
\begin{align}
\lambda_{1,2} = -\frac{3}{2} \pm j\frac{\sqrt{39}}{2}
\end{align}

podemos factorizar el denominador como $s^2+3s+12=(s-\lambda_1)(s-\lambda_2)$. 

Aplicando fracciones parciales:
\begin{align}
H(s) &= \frac{0.1}{(s-\lambda_1)(s-\lambda_2)} = \frac{A}{s-\lambda_1} + \frac{B}{s-\lambda_2}
\end{align}

Para encontrar las constantes $A$ y $B$, utilizamos:
\begin{align}
A &= \frac{0.1}{\lambda_1-\lambda_2} = \frac{0.1}{j\sqrt{39}} = \frac{-j \cdot 0.1}{\sqrt{39}} \\
B &= \frac{0.1}{\lambda_2-\lambda_1} = \frac{0.1}{-j\sqrt{39}} = \frac{j \cdot 0.1}{\sqrt{39}}
\end{align}

Por lo tanto:
\begin{align}
H(s) &= \frac{-j \cdot 0.1}{\sqrt{39}} \cdot \frac{1}{s-\lambda_1} + \frac{j \cdot 0.1}{\sqrt{39}} \cdot \frac{1}{s-\lambda_2}
\end{align}

Aplicando la transformada inversa de Laplace:
\begin{align}
h(t) &= \mathcal{L}^{-1}\{H(s)\} \\
&= \frac{-j \cdot 0.1}{\sqrt{39}} e^{\lambda_1 t} + \frac{j \cdot 0.1}{\sqrt{39}} e^{\lambda_2 t}, \quad t \geq 0 \\
&= \frac{-j \cdot 0.1}{\sqrt{39}} e^{\left(-\frac{3}{2}+j\frac{\sqrt{39}}{2}\right)t} + \frac{j \cdot 0.1}{\sqrt{39}} e^{\left(-\frac{3}{2}-j\frac{\sqrt{39}}{2}\right)t}, \quad t \geq 0
\end{align}

Para expresar el resultado en términos de funciones trigonométricas reales, factorizamos $e^{-\frac{3}{2}t}$ y aplicamos las identidades de Euler. Después de simplificar algebraicamente:
\begin{align}
h(t) &= \frac{0.2}{\sqrt{39}} e^{-\frac{3}{2}t} \sin\left(\frac{\sqrt{39}}{2}t\right), \quad t \geq 0
\end{align}

Esta respuesta al impulso muestra el comportamiento característico de un sistema subamortiguado: una exponencial decreciente modulada por una función senoidal, reflejando la naturaleza oscilatoria amortiguada del sistema motor DC-carrito.
\subsection*{Resolución 2.5}

La salida para condiciones iniciales arbitrarias $x(0)=\begin{bmatrix}x_1(0)\\ x_2(0)\end{bmatrix}$ y entrada $u(t)$ se escribe como
\begin{equation}
y(t)=C\,\Phi(t)\,x(0)\;+\;(h*u)(t),\qquad h(t)=\mathcal{L}^{-1}\!\{H(s)\},
\end{equation}
con $C=\begin{bmatrix}0.01&0\end{bmatrix}$, $D=0$ y $H(s)=C(sI-A)^{-1}B=\dfrac{0.1}{s^2+3s+12}$.

Usando la forma real de $\Phi(t)$ obtenida antes, con
\[
\beta=\frac{\sqrt{39}}{2},
\]
se tiene
\begin{align}
C\,\Phi(t)
&=0.01\,e^{-1.5t}\!
\begin{bmatrix}
\cos(\beta t)-\dfrac{1}{\sqrt{39}}\sin(\beta t) &
\dfrac{20}{\sqrt{39}}\sin(\beta t)
\end{bmatrix}.
\end{align}
Por lo tanto, la respuesta libre queda
\begin{align}
y_{\text{libre}}(t)
&=0.01\,e^{-1.5t}
\Bigg(
\Big[\cos(\beta t)-\frac{1}{\sqrt{39}}\sin(\beta t)\Big]\,x_1(0)
+\frac{20}{\sqrt{39}}\sin(\beta t)\,x_2(0)
\Bigg).
\end{align}

Para la parte forzada, usando $H(s)$ anterior,
\begin{align}
h(t)
&=\mathcal{L}^{-1}\!\{H(s)\}
=\frac{0.2}{\sqrt{39}}\,e^{-1.5t}\,\sin(\beta t)\,\mathbf{1}_{t\ge 0}
\;=\;\frac{-j\,0.1}{\sqrt{39}}\,e^{(-\frac{3}{2}+j\beta)t}
+\frac{j\,0.1}{\sqrt{39}}\,e^{(-\frac{3}{2}-j\beta)t}.
\end{align}
Así, la respuesta total es
\begin{align}
y(t)
&=0.01\,e^{-1.5t}
\Bigg(
\Big[\cos(\beta t)-\frac{1}{\sqrt{39}}\sin(\beta t)\Big]\,x_1(0)
+\frac{20}{\sqrt{39}}\sin(\beta t)\,x_2(0)
\Bigg)
+\int_{0}^{t} h(\tau)\,u(t-\tau)\,d\tau .
\end{align}

\subsection*{Resolución 2.6}
Para determinar si el sistema es completamente controlable y observable, utilizamos las matrices de controlabilidad y observabilidad. La matriz de controlabilidad $\mathcal{C}$ para $n=2$ viene dada por:
\begin{align}
\mathcal{C} = [B \quad AB]
\end{align}
Calculamos $AB$:
\begin{align}
AB = A \cdot B = \begin{bmatrix} -2 & 10
\\ -1 & -1 \end{bmatrix} \cdot \begin{bmatrix} 0 \\ 1 \end{bmatrix} = \begin{bmatrix} 10 \\ -1 \end{bmatrix}
\end{align}
Por lo que la matriz de controlabilidad queda:
\begin{align}
\mathcal{C} = \begin{bmatrix} 0 & 10 \\ 1 & -1 \end{bmatrix}
\end{align}
Observamos que tanto las filas como las columnas son linealmente independientes, por lo que la matriz es de rango completo. Otra manera de verificar esto es calculando el determinante:
\begin{align}
\det(\mathcal{C}) = (0)(-1) - (10)(1) = -10 \neq 0
\end{align}
Como el determinante es diferente de cero, confirmamos que la matriz es de rango completo, por lo que el sistema es controlable. 

La matriz de observabilidad $\mathcal{O}$ para $n=2$ viene dada por:
\begin{align}
\mathcal{O} = \begin{bmatrix} C \\ CA \end{bmatrix}
\end{align}
Calculamos $CA$:
\begin{align}
CA = C \cdot A = \begin{bmatrix} 0.01 & 0 \end{bmatrix} \cdot \begin{bmatrix} -2 & 10 \\ -1 & -1 \end{bmatrix} = \begin{bmatrix} -0.02 & 0.1 \end{bmatrix}
\end{align}
Por lo que la matriz de observabilidad queda:
\begin{align}
\mathcal{O} = \begin{bmatrix} 0.01 & 0 \\ -0.02 & 0.1 \end{bmatrix}
\end{align}
Nuevamente observamos que tanto las filas como las columnas son linealmente independientes, por lo que la matriz es de rango completo. Verificamos esto calculando el determinante:
\begin{align}
\det(\mathcal{O}) = (0.01)(0.1) - (0)(-0.02) = 0.001 \neq 0
\end{align}
Como el determinante es diferente de cero, confirmamos que la matriz es de rango completo, por lo que el sistema es observable. 

De esta manera concluimos que el sistema es completamente controlable y observable.
\subsection*{Resolución 2.7}
Dado que el sistema es observable, podemos diseñar un observador de estados. Queremos que los polos del observador se ubiquen en $-10$ y $-5$. Para esto utilizaremos el siguiente esquema:
  \begin{figure}[H]\centering
    \includegraphics[width=.6\textwidth]{../figures/Auxiliar_4_3.png}
    \caption{El observador replica la ecuación $\dot{\hat x}=A\hat x + Bu + L(y-\hat y)$. La rama principal implementa el modelo del sistema ($A$, $B$ y el integrador). La rama vertical superior lleva la salida estimada $\hat y = C\hat x$ al comparador donde se obtiene el error de salida $e_y$. Ese error, tras el bloque de ganancia $L$, se suma como término corrector que desplaza los polos de la dinámica del error a posiciones elegidas (más rápidos), garantizando convergencia de $\hat x$ hacia $x$. De este modo el diseño de $L$ es independiente del de $K$, ilustrando el principio de separación.}
  \end{figure}

Tenemos que el observador de estados vendrá dado por
\begin{align}
\dot{\hat x} &= A\hat x + Bu + L\big(y-\hat y\big)\\
\hat y &= C\hat x,
\end{align}
con error de estimación $e:=x-\hat x$. La dinámica del error de estimación se obtiene como:
\begin{align}
\dot e
&= \dot x - \dot{\hat x}
= (Ax+Bu) - (A\hat x + Bu + L(y-\hat y)) \\
&= A(x-\hat x) - L\big(Cx - C\hat x\big)
= (A-LC)e. \label{eq:error_dyn}
\end{align}
Por lo tanto:
\begin{align}
\dot e &= (A-LC)e
\end{align}
Los polos del observador son los autovalores de $A-LC$, por lo que debemos determinar $L$ tal que estos autovalores se ubiquen en $-10$ y $-5$. Definiendo:
\begin{align}
  L= \begin{bmatrix} \ell_1 \\ \ell_2 \end{bmatrix}\\
  LC = \begin{bmatrix} 0.01\ell_1 & 0 \\ 0.01\ell_2 & 0 \end{bmatrix}
\end{align}
De esta manera tenemos que:
\begin{align}
  \det(\lambda I - (A-LC)) &= \det\begin{bmatrix} \lambda + 2 + 0.01\ell_1 & -10 \\ 1 + 0.01\ell_2 & \lambda + 1 \end{bmatrix} \\
  &= (\lambda + 2 + 0.01\ell_1)(\lambda + 1) - (-10)(1 + 0.01\ell_2) \\
  &= \lambda^2 + (3 + 0.01\ell_1)\lambda + (12 + 0.01\ell_1 + 0.1\ell_2)
\end{align}
Dado que queremos ubicar los polos en $-10$ y $-5$, el polinomio característico deseado es:
\begin{align}
  (\lambda + 10)(\lambda + 5) = \lambda^2 + 15\lambda + 50
\end{align}
Igualando coeficientes, obtenemos el siguiente sistema de ecuaciones:
\begin{align}
  3 + 0.01\ell_1 &= 15 \\
  12 + 0.01\ell_1 + 0.1\ell_2 &= 50
\end{align}
Resolviendo el sistema de ecuaciones:
\begin{align}
  \ell_1 &= \frac{15-3}{0.01} = \frac{12}{0.01} = 1200 \\
  12 + 0.01(1200) + 0.1\ell_2 &= 50 \\
  12 + 12 + 0.1\ell_2 &= 50 \\
  24 + 0.1\ell_2 &= 50 \\
  0.1\ell_2 &= 26 \\
  \ell_2 &= 260
\end{align}
Por lo tanto, la matriz de ganancias del observador es:
\[
L=\begin{bmatrix}1200\\260\end{bmatrix}
\]
Con esta matriz $L$, el observador de estados tendrá polos ubicados en $-10$ y $-5$, asegurando una convergencia rápida del error de estimación hacia cero.
\end{solution}
\end{document}