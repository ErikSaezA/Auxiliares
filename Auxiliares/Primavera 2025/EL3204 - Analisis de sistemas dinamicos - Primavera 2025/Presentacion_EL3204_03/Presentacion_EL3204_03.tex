\documentclass[
    10pt,
    aspectratio=169,
    xcolor={dvipsnames},
    spanish,
    % handout,
    % notes=only,
    % notes,
    ]{beamer}

% BEAMER SETTINGS
\setbeamerfont{section in toc}{size=\normalsize, shape=\bfseries}
\mode<presentation>{
    \usetheme{Antibes}
    \setbeamercovered{transparent}
    \usecolortheme{rose}
    \setbeamertemplate{navigation symbols}{}
    }
\useoutertheme{infolines}

% PACKAGES
% \usepackage[spanish]{babel}  % uncomment for Spanish support
\usepackage{tikz,pgfplots}
\pgfplotsset{compat=1.13}
\usetikzlibrary{calc}
\usepackage{subcaption}
\usepackage{graphicx}
\graphicspath{{figures}}
\usepackage{booktabs}
\usepackage{upgreek}
\usepackage{commath}
\usepackage{amsmath,amsthm,amssymb,mathtools,mathrsfs}
\usepackage{cancel}
\usepackage{fontawesome5}
\usepackage{enumerate}
\usepackage{tensor}
\usepackage[font=footnotesize]{caption}
\usepackage{wasysym}

\usepackage[skins,theorems]{tcolorbox}
\tcbset{
    highlight math style={
        enhanced,
        coltext=black,
        colframe=black,
        colback=lightgray,
        arc=0pt,
        boxrule=.5pt
        }
}

% REFERENCES AND OTHERS
\usepackage{aas_macros}
\usepackage{natbib}
\bibpunct{(}{)}{;}{a}{}{,}

\usepackage{siunitx}
\sisetup{
    range-phrase=\text{--},
    range-units=single,
    separate-uncertainty=true,
    print-unity-mantissa=false
    }
\DeclareSIUnit{\gauss}{G}
\DeclareSIUnit{\jansky}{Jy}
\renewcommand{\figurename}{Fig.}

\usepackage{hyperref}
\hypersetup{
    % bookmarks=true,
    unicode=true,
    pdftoolbar=true,
    pdfmenubar=true,
    pdffitwindow=false,
    pdfstartview={FitH},
    pdftitle={ISI-Free Linear Combination Pulses with Better Performanc},
    pdfauthor={Erik Saez A.},
    pdfcreator={Erik Saez A.},
    pdfnewwindow=true,
    colorlinks=true,
    linkcolor=RoyalBlue,
    citecolor=RoyalBlue,
    urlcolor=RoyalBlue
    }

\title[Auxiliar \#3]{\bfseries Auxiliar \#3}
\subtitle{Análisis de sistemas dinámicos y estimación}
\author[Erik Saez A.]{Erik Saez A.}
\institute[UChile]{Department of Electrical Engineering \\ Universidad de Chile}

\date{\today}

\begin{document}

\begin{frame}
  \titlepage
  \centering
  \faIcon{envelope} \href{mailto:erik.saez@ug.uchile.cl}{erik.saez@ug.uchile.cl} \hspace{.2cm}
\end{frame}

\begin{frame}
  \frametitle{Contenidos}
  \centering
  \begin{columns}
    \begin{column}{0.4\textwidth}
      \tableofcontents
    \end{column}
    \begin{column}{0.5\textwidth}
      \begin{figure}
        \centering
        \includegraphics[width=\textwidth]{fcfm_die}
        \caption{Facultad de Ciencias Físicas y Matemáticas , Universidad de Chile.}
      \end{figure}
    \end{column}
  \end{columns}  
\end{frame}
%%%%%%%%%%%%%%%%%%%%%%%%%%%%%%%%%%%%%%%%%%

\section{Resumen}

%%%%%%%%%%%%%%%%%%%%%%%%
\begin{frame}{Función de transferencia}
\footnotesize
\begin{block}{¿Qué es y para qué sirve?}
  Describe la relación entrada–salida de un sistema LTI con condiciones iniciales nulas:
  $Y(s)=H(s)\,U(s)$ (o $Y(z)=H(z)\,U(z)$ en discreto). Resume la dinámica en el dominio transformado y permite:
  \begin{itemize}\itemsep2pt
    \item identificar polos y ceros (estabilidad y dinámica);
    \item analizar la respuesta en frecuencia y el desempeño;
    \item componer sistemas (cascada/paralelo/retroalimentación);
    \item pasar entre $H(\cdot)$ y realizaciones en variables de estado.
  \end{itemize}
\end{block}
\begin{columns}
  \begin{column}{0.52\textwidth}
    \begin{block}{Definición (continuo)}
      Para $\dot x=Ax+Bu$, $y=Cx+Du$:
      \[ H(s)=\frac{Y(s)}{U(s)}=C\,(sI-A)^{-1}B + D. \]
    \end{block}
  \end{column}
  \begin{column}{0.46\textwidth}
    \begin{block}{Discreto}
      Para $x[k+1]=Ax[k]+Bu[k]$, $y[k]=Cx[k]+Du[k]$:
      \[ H(z)=C\,(zI-A)^{-1}B + D. \]
    \end{block}
  \end{column}
\end{columns}
\end{frame}

%%%%%%%%%%%%%%%%%%%%%%%%
\begin{frame}{De EDO a variables de estado (resumen)}
\begin{columns}
  \begin{column}{0.48\textwidth}
    \begin{block}{Definición (VVEE)}
      \footnotesize
      Continuo: $\dot x=Ax+Bu$, $y=Cx+Du$. Discreto: $x[k+1]=Ax[k]+Bu[k]$, $y[k]=Cx[k]+Du[k]$.
      $x\in\mathbb{R}^n$, $u\in\mathbb{R}^m$, $y\in\mathbb{R}^p$.
    \end{block}
    \begin{block}{Método 1 (directo desde la EDO)}
      \footnotesize
      Para $\ddot y + 2\dot y - 15y = u$, defina $x_1=y,\;x_2=\dot y$:
      \[
      \dot{x}=\underbrace{\begin{pmatrix}0&1\\[2pt]15&-2\end{pmatrix}}_{A}\,x + \underbrace{\begin{pmatrix}0\\ 1\end{pmatrix}}_{B}u,\quad
      y=\underbrace{\begin{pmatrix}1&0\end{pmatrix}}_{C}x.
      \]
      Pros: directo. Contras: $A$ no diagonal \,$\Rightarrow$\, cálculos largos para $\boldsymbol{\Phi}(t)$.
    \end{block}
  \end{column}
  \begin{column}{0.48\textwidth}
    \begin{block}{Método 2 (desde $H(s)$)}
      \footnotesize
      $H(s)=\dfrac{1}{(s-3)(s+5)}=\dfrac{1/8}{s-3}-\dfrac{1/8}{s+5}$ \;$\Rightarrow$\; realización
      \[
      A=\mathrm{diag}(3,-5),\; B=\begin{pmatrix}1\\ 1\end{pmatrix},\; C=\begin{pmatrix}\tfrac{1}{8}&-\tfrac{1}{8}\end{pmatrix}.
      \]
      Ventajas: $A$ diagonal simplifica $\boldsymbol{\Phi}(t)$ e $h(t)$; polos de $H(s)$ en la diagonal de $A$.
    \end{block}
  \end{column}
\end{columns}
\end{frame}

%%%%%%%%%%%%%%%%%%%%%%%%
\begin{frame}{Matriz de transición de estados (MTE)}
\footnotesize
\begin{columns}
  \begin{column}{0.52\textwidth}
    \begin{block}{Definición y propiedades}
      	\textbf{Recordemos:} $\Phi(t,t_0)$ (o $\Phi[k,k_0]$) es la \emph{solución fundamental} que propaga estados:
      \[x(t)=\Phi(t,t_0)\,x(t_0),\qquad \Phi(t_0,t_0)=I.\]
      	\textbf{Dinámica general}
      \begin{align}
        \frac{\partial}{\partial t}\,\Phi(t,t_0)&=A(t)\,\Phi(t,t_0), && \Phi(t_0,t_0)=I, \\[2pt]
      \Phi[k{+}1,k_0]&=A[k]\,\Phi[k,k_0], && \Phi[k_0,k_0]=I.
      \end{align}
    \end{block}

    \begin{block}{Caso LTI}
      Matrices constantes:
      \begin{align}
      \Phi(t,t_0)&=e^{A(t-t_0)}=\mathcal{L}^{-1}\!\{(sI-A)^{-1}\},\\
      \Phi[k,k_0]&=A^{\,k-k_0}=\mathcal{Z}^{-1}\!\{\,z\,(zI-A)^{-1}\}.
      \end{align}
    \end{block}
  \end{column}
  \begin{column}{0.46\textwidth}
    \begin{block}{Variación de parámetros}
      Entrada no nula (continuo):
      \[x(t)=\Phi(t,t_0)\,x(t_0)+\int_{t_0}^{t}\!\Phi(t,\tau)\,B(\tau)\,u(\tau)\,d\tau.\]
    \end{block}
  \end{column}
\end{columns}
\end{frame}


%%%%%%%%%%%%%%%%%%%%%%%%
\begin{frame}{Diagonalización (idea y receta)}
\footnotesize
\begin{block}{Diagonalización}
  \noindent\textbf{¿Por qué diagonalizar?}
Escribir $A=TDT^{-1}$ desacopla la dinámica por modos y hace inmediata la MTE:
$\Phi(t)=e^{At}=T\,e^{Dt}\,T^{-1}$ (con $e^{Dt}$ diagonal). Esto simplifica el cálculo de
$f(A)$ (p.\,ej., $A^k$, $e^{At}$), clarifica la interpretación (autovalores $\lambda_i$ como polos/estabilidad)
y facilita análisis y diseño por modos (REN(C)/RESC, control y observación) al reducir operaciones
matriciales a exponenciales escalares.

\end{block}
\begin{columns}
  \begin{column}{0.52\textwidth}

    \begin{block}{Condición de diagonalizabilidad}
      $A$ es diagonalizable $\Longleftrightarrow$ la suma de \emph{multiplicidades geométricas} es $n$ (equiv.: para cada $\lambda$, mult. geom. $=$ mult. alg.).
    \end{block}
    \begin{block}{Receta (pasos prácticos)}
    \begin{enumerate}\itemsep2pt
      \item Calcular el polinomio característico $p(\lambda)=\det(\lambda I-A)$ y sus raíces $\lambda_i$ (autovalores).
      \item Para cada $\lambda_i$, resolver $(A-\lambda_i I)v=0$ y obtener una base del subespacio propio $\mathcal N(A-\lambda_i I)$.
      \item Si se obtienen $n$ autovectores linealmente independientes, formar $T=[\,v_1\;\cdots\;v_n\,]$ y $D=\operatorname{diag}(\lambda_1,\dots,\lambda_n)$.
    \end{enumerate}
    \end{block}

  \end{column}

  \begin{column}{0.46\textwidth}
  
    \begin{block}{Casos útiles}
    \begin{itemize}\itemsep2pt
      \item Si $A=A^\top$ (real simétrica) $\Rightarrow$ diagonalizable por matriz ortogonal: $A=Q\Lambda Q^\top$ (teorema espectral).
      \item Si no hay $n$ autovectores L.I. $\Rightarrow$ \emph{no} es diagonalizable: usar forma canónica de Jordan.
    \end{itemize}
    \end{block}

    \begin{block}{Conexión con MTE}
    Si $A=TDT^{-1}$, entonces $f(A)=Tf(D)T^{-1}$. En particular: $\Phi(t)=e^{At}=Te^{Dt}T^{-1}$ y $e^{Dt}=\operatorname{diag}(e^{\lambda_i t})$.
    \end{block}
  \end{column}
\end{columns}
\end{frame}


%%%%%%%%%%%%%%%%%%%%%%%%
\begin{frame}{Forma canónica de Jordan}
\footnotesize
Una matriz no diagonalizable se escribe $A=PJP^{-1}$, donde $J$ es \emph{block-diagonal} con bloques de Jordan $J_i(\lambda)$:
\[
J_i(\lambda)=\begin{pmatrix}
\lambda & 1 &  & 0\\
 & \lambda & \ddots & \\
 &  & \ddots & 1\\
0 &  &  & \lambda
\end{pmatrix}.
\]
Cada bloque corresponde a una cadena de autovectores generalizados. Consecuencia clave para $e^{At}$:
\begin{itemize}\itemsep2pt
  \item $e^{J(\lambda)t}=e^{\lambda t}$ multiplicado por una matriz triangular con potencias $t^k/k!$ sobre superdiagonales.
  \item Si hay bloques de tamaño $m$, aparecen términos $t^k e^{\lambda t}$, $k=0,\dots,m-1$ en las respuestas.
\end{itemize}
\end{frame}

%%%%%%%%%%%%%%%%%%%%%%%%
\begin{frame}{Respuesta impulsional y funciones base}
\footnotesize
Continuo: $\;h(t)=C\,\boldsymbol{\Phi}(t)\,B + D\,\delta(t)$.\; Discreto: $\;h[k]=C A^{k-1}B$ para $k\ge1$ y $h[0]=D$.

\medskip
Ejemplo (realización del Método 2):
\[h(t)=\tfrac{1}{8}e^{3t}-\tfrac{1}{8}e^{-5t},\quad t\ge 0.\]

Respuesta a entrada cero: $\;y_0(t)=C\,\boldsymbol{\Phi}(t)\,x(0)$.
Las funciones base son las componentes L.I. que aparecen en $y_0(t)$.

\medskip
Ejemplo: $\;\{e^{3t},\,e^{-5t}\}$. Si hay cadenas de Jordan de tamaño $m$, aparecen términos $t^k e^{\lambda t}$, $k=0,\dots,m-1$.
\end{frame}

%%%%%%%%%%%%%%%%%%%%%%%%
\begin{frame}{Estabilidad: BIBS y BIBO}
\footnotesize
	extbf{BIBS} (continuo): $\operatorname{Re}\{\lambda_i(A)\}<0$. \;\textbf{BIBS} (discreto): $|\lambda_i(A)|<1$. En la frontera puede haber comportamientos marginales.

\medskip
	extbf{BIBO}: continuo $\int_0^{\infty}|h(t)|\,dt<\infty$; discreto $\sum_{k\ge 0}|h[k]|<\infty$.

\medskip
Ejemplo: $\lambda_1=3,\;\lambda_2=-5$ \;$\Rightarrow$\; no BIBS. Como $h(t)$ contiene $e^{3t}$, tampoco es BIBO. Relación continuo-discreto: $z=e^{sT}$.
\end{frame}

%%%%%%%%%%%%%%%%%%%%%%%%
\begin{frame}{Controlabilidad y observabilidad (resumen)}
\footnotesize
	extbf{Controlable} si $\operatorname{rank}\,[\,B\;AB\;\cdots\;A^{n-1}B\,]=n$. \quad
	extbf{Observable} si $\operatorname{rank}\,\begin{bmatrix}C\\ CA\\ \vdots \\ CA^{n-1}\end{bmatrix}=n$.

\medskip
	extbf{PBH}: $\operatorname{rank}\,[\,sI-A\;\;B\,]=n$ y $\operatorname{rank}\,[\,sI-A\;\;C^T\,]=n$ para todo $s\in\mathbb{C}$.

\medskip
	extbf{Gramianos} (continuo, horizonte $[0,\infty)$, si $A$ Hurwitz):
\[A W_c+W_c A^T+BB^T=0,\qquad A^T W_o+W_o A+C^T C=0.\]
\end{frame}



%%%%%%%%%%%%%%%%%%%%%%%%
\section{Pregunta 1}
\begin{frame}{Pregunta \#1}
\begin{block}{Enunciado Pregunta \#1}
Considere un sistema modelado por la siguiente ecuación diferencial:
  \begin{equation}
    \ddot y + 2\dot y - 15 y = u.
  \end{equation}

  \begin{enumerate}
    \item Encuentre la función de transferencia del sistema.
    \item Formule el sistema en variables de estado.
    \item Obtenga la MTE del sistema y encuentre las funciones base.
    \item Encuentre la respuesta al impulso del sistema.
    \item Determine la estabilidad BIBS y BIBO del sistema.
  \end{enumerate}
\end{block}
\end{frame}
%%%%%%%%%%%%%%%%%%%%%%
\section{Pregunta 2}
\begin{frame}{Pregunta \#2}
  \begin{block}{Enunciado Pregunta \#2}
   Considere el siguiente sistema formulado en variables de estado:
\begin{equation}
  \dot{x}(t) =
  \begin{pmatrix}
    -2 & 1 & 0 & 0 \\
     0 & -2 & 1 & 0 \\
     0 &  0 & -2 & 0 \\
     0 &  0 &  0 & 1
  \end{pmatrix} x(t)
  +
  \begin{pmatrix}
    1 \\
    2 \\
    3 \\
    0
  \end{pmatrix} u(t),
  \qquad
  y(t) = \begin{pmatrix} 1 & 0 & 1 & 1 \end{pmatrix} x(t).
\end{equation}

\begin{enumerate}
  \item Encuentre la MTE y las funciones base del sistema.
  \item Encuentre la respuesta al impulso del sistema.
  \item Determine estabilidad BIBS y BIBO.
  \item Determine observabilidad y controlabilidad.
\end{enumerate}

  \end{block}
\end{frame}

%%%%%%%%%%%%%%%%%%%%%%


\end{document}
