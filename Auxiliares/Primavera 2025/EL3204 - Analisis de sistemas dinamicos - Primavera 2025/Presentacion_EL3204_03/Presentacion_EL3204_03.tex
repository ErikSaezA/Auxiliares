\documentclass[
    10pt,
    aspectratio=169,
    xcolor={dvipsnames},
    spanish,
    % handout,
    % notes=only,
    % notes,
    ]{beamer}

% BEAMER SETTINGS
\setbeamerfont{section in toc}{size=\normalsize, shape=\bfseries}
\mode<presentation>{
    \usetheme{Antibes}
    \setbeamercovered{transparent}
    \usecolortheme{rose}
    \setbeamertemplate{navigation symbols}{}
    }
\useoutertheme{infolines}

% PACKAGES
% \usepackage[spanish]{babel}  % uncomment for Spanish support
\usepackage{tikz,pgfplots}
\pgfplotsset{compat=1.13}
\usetikzlibrary{calc}
\usepackage{subcaption}
\usepackage{graphicx}
\graphicspath{{figures}}
\usepackage{booktabs}
\usepackage{upgreek}
\usepackage{commath}
\usepackage{amsmath,amsthm,amssymb,mathtools,mathrsfs}
\usepackage{cancel}
\usepackage{fontawesome5}
\usepackage{enumerate}
\usepackage{tensor}
\usepackage[font=footnotesize]{caption}
\usepackage{wasysym}

\usepackage[skins,theorems]{tcolorbox}
\tcbset{
    highlight math style={
        enhanced,
        coltext=black,
        colframe=black,
        colback=lightgray,
        arc=0pt,
        boxrule=.5pt
        }
}

% REFERENCES AND OTHERS
\usepackage{aas_macros}
\usepackage{natbib}
\bibpunct{(}{)}{;}{a}{}{,}

\usepackage{siunitx}
\sisetup{
    range-phrase=\text{--},
    range-units=single,
    separate-uncertainty=true,
    print-unity-mantissa=false
    }
\DeclareSIUnit{\gauss}{G}
\DeclareSIUnit{\jansky}{Jy}
\renewcommand{\figurename}{Fig.}

\usepackage{hyperref}
\hypersetup{
    % bookmarks=true,
    unicode=true,
    pdftoolbar=true,
    pdfmenubar=true,
    pdffitwindow=false,
    pdfstartview={FitH},
    pdftitle={ISI-Free Linear Combination Pulses with Better Performanc},
    pdfauthor={Erik Saez A.},
    pdfcreator={Erik Saez A.},
    pdfnewwindow=true,
    colorlinks=true,
    linkcolor=RoyalBlue,
    citecolor=RoyalBlue,
    urlcolor=RoyalBlue
    }

\title[Auxiliar \#3]{\bfseries Auxiliar \#3}
\subtitle{Análisis de sistemas dinámicos y estimación}
\author[Erik Saez A.]{Erik Saez A.}
\institute[UChile]{Department of Electrical Engineering \\ Universidad de Chile}

\date{\today}

\begin{document}

\begin{frame}
  \titlepage
  \centering
  \faIcon{envelope} \href{mailto:erik.saez@ug.uchile.cl}{erik.saez@ug.uchile.cl} \hspace{.2cm}
\end{frame}

\begin{frame}
  \frametitle{Contenidos}
  \centering
  \begin{columns}
    \begin{column}{0.4\textwidth}
      \tableofcontents
    \end{column}
    \begin{column}{0.5\textwidth}
      \begin{figure}
        \centering
        \includegraphics[width=\textwidth]{fcfm_die}
        \caption{Facultad de Ciencias Físicas y Matemáticas , Universidad de Chile.}
      \end{figure}
    \end{column}
  \end{columns}  
\end{frame}
%%%%%%%%%%%%%%%%%%%%%%%%%%%%%%%%%%%%%%%%%%

\section{Resumen}

%%%%%%%%%%%%%%%%%%%%%%%%
\begin{frame}{Función de transferencia}
\footnotesize
\begin{block}{¿Qué es y para qué sirve?}
  Describe la relación entrada–salida de un sistema LTI con condiciones iniciales nulas:
  $Y(s)=H(s)\,U(s)$ (o $Y(z)=H(z)\,U(z)$ en discreto). Resume la dinámica en el dominio transformado y permite:
  \begin{itemize}\itemsep2pt
    \item identificar polos y ceros (estabilidad y dinámica);
    \item analizar la respuesta en frecuencia y el desempeño;
    \item componer sistemas (cascada/paralelo/retroalimentación);
    \item pasar entre $H(\cdot)$ y realizaciones en variables de estado.
  \end{itemize}
\end{block}
\begin{columns}
  \begin{column}{0.52\textwidth}
    \begin{block}{Definición (continuo)}
      Para $\dot x=Ax+Bu$, $y=Cx+Du$:
      \[ H(s)=\frac{Y(s)}{U(s)}=C\,(sI-A)^{-1}B + D. \]
    \end{block}
  \end{column}
  \begin{column}{0.46\textwidth}
    \begin{block}{Discreto}
      Para $x[k+1]=Ax[k]+Bu[k]$, $y[k]=Cx[k]+Du[k]$:
      \[ H(z)=C\,(zI-A)^{-1}B + D. \]
    \end{block}
  \end{column}
\end{columns}
\end{frame}

%%%%%%%%%%%%%%%%%%%%%%%%
\begin{frame}{Variables de estado:}
\begin{block}{¿Qué es y por qué usarlo?}
  Representa sistemas dinámicos mediante un \textbf{vector de estado} $x$ (memoria/CI) y matrices $(A,B,C,D)$.
  Ventajas clave:
  \begin{itemize}\itemsep2pt
    \item Válido para \textbf{MIMO} (múltiples entradas/salidas) y para interconexiones (cascada, paralelo, realimentación).
    \item Maneja \textbf{condiciones iniciales} de forma explícita y separa respuesta libre/forzada con la MTE $\Phi(t)=e^{At}$.
    \item Base del \textbf{análisis estructural}: controlabilidad/observabilidad, Gramianos, descomposiciones.
  \end{itemize}
\end{block}

\begin{block}{Modelo LTI continuo (notación)}
\[
  \dot{x}=A\,x+B\,u, \qquad y=C\,x+D\,u,
\]
\[
  A\!\in\!\mathbb{R}^{n\times n},\; B\!\in\!\mathbb{R}^{n\times m},\; C\!\in\!\mathbb{R}^{p\times n},\; D\!\in\!\mathbb{R}^{p\times m}.
\]
\end{block}
\end{frame}

%%%%%%%%%%%%%%%%%%%%%%%%
\begin{frame}{Matriz de transición de estados (MTE)}
\footnotesize
\begin{columns}
  \begin{column}{0.52\textwidth}
    \begin{block}{Definición y propiedades}
      	\textbf{Recordemos:} $\Phi(t,t_0)$ (o $\Phi[k,k_0]$) es la \emph{solución fundamental} que propaga estados:
      \[x(t)=\Phi(t,t_0)\,x(t_0),\qquad \Phi(t_0,t_0)=I.\]
      	\textbf{Dinámica general}
      \begin{align}
        \frac{\partial}{\partial t}\,\Phi(t,t_0)&=A(t)\,\Phi(t,t_0), && \Phi(t_0,t_0)=I, \\[2pt]
      \Phi[k{+}1,k_0]&=A[k]\,\Phi[k,k_0], && \Phi[k_0,k_0]=I.
      \end{align}
    \end{block}

    \begin{block}{Caso LTI}
      Matrices constantes:
      \begin{align}
      \Phi(t,t_0)&=e^{A(t-t_0)}=\mathcal{L}^{-1}\!\{(sI-A)^{-1}\},\\
      \Phi[k,k_0]&=A^{\,k-k_0}=\mathcal{Z}^{-1}\!\{\,z\,(zI-A)^{-1}\}.
      \end{align}
    \end{block}
  \end{column}
  \begin{column}{0.46\textwidth}
    \begin{block}{Variación de parámetros}
      Entrada no nula (continuo):
      \[x(t)=\Phi(t,t_0)\,x(t_0)+\int_{t_0}^{t}\!\Phi(t,\tau)\,B(\tau)\,u(\tau)\,d\tau.\]
    \end{block}
  \end{column}
\end{columns}
\end{frame}


%%%%%%%%%%%%%%%%%%%%%%%%
\begin{frame}{Diagonalización}
\footnotesize
\begin{block}{Diagonalización}
  \noindent\textbf{¿Por qué diagonalizar?}
Escribir $A=TDT^{-1}$ desacopla la dinámica por modos y hace inmediata la MTE:
$\Phi(t)=e^{At}=T\,e^{Dt}\,T^{-1}$ (con $e^{Dt}$ diagonal). Esto simplifica el cálculo de $f(A)$ (p.\,ej., $A^k$, $e^{At}$), clarifica la interpretación (autovalores $\lambda_i$ como polos/estabilidad) y facilita análisis y diseño por modos (REN(C)/RESC, control y observación) al reducir operaciones
matriciales a exponenciales escalares.

\end{block}
\begin{columns}
  \begin{column}{0.52\textwidth}

    \begin{block}{Condición de diagonalizabilidad}
      $A$ es diagonalizable $\Longleftrightarrow$ la suma de \emph{multiplicidades geométricas} es $n$ (equiv.: para cada $\lambda$, mult. geom. $=$ mult. alg.).
    \end{block}
    \begin{block}{Receta}
    \begin{enumerate}\itemsep2pt
      \item Calcular el polinomio característico $p(\lambda)=\det(\lambda I-A)$ y sus raíces $\lambda_i$ (autovalores).
      \item Para cada $\lambda_i$, resolver $(A-\lambda_i I)v=0$ y obtener una base del subespacio propio $\mathcal N(A-\lambda_i I)$.
      \item Si se obtienen $n$ autovectores linealmente independientes, formar $T=[\,v_1\;\cdots\;v_n\,]$ y $D=\operatorname{diag}(\lambda_1,\dots,\lambda_n)$.
    \end{enumerate}
    \end{block}

  \end{column}

  \begin{column}{0.46\textwidth}
  

    \begin{block}{Conexión con MTE}
    Si $A=TDT^{-1}$, entonces $f(A)=Tf(D)T^{-1}$. En particular: $\Phi(t)=e^{At}=Te^{Dt}T^{-1}$ y $e^{Dt}=\operatorname{diag}(e^{\lambda_i t})$.
    \end{block}
  \end{column}
\end{columns}
\end{frame}


%%%%%%%%%%%%%%%%%%%%%%%%
\begin{frame}{Forma canónica de Jordan}
\footnotesize
\begin{columns}[T]
  \begin{column}{0.45\textwidth}
    \begin{block}{¿Qué es y cuándo aparece?}
      Toda matriz $A\in\mathbb{C}^{n\times n}$ es similar a una \textbf{forma canónica de Jordan}: $A=PJP^{-1}$, con
      $J=\operatorname{diag}\big(J_{m_1}(\lambda_1),\dots,J_{m_r}(\lambda_r)\big)$. Un \emph{bloque de Jordan} $J_m(\lambda)$
      tiene $\lambda$ en la diagonal y $1$ en la superdiagonal.
      Aparece cuando $A$ \textbf{no es diagonalizable} (multiplicidad geométrica menor que la algébrica).
    \end{block}
    \begin{block}{¿Para qué sirve?}
      \begin{itemize}\itemsep2pt
        \item Describe la estructura modal cuando faltan autovectores L.I.
        \item Facilita el cálculo de $f(A)$: $e^{At}$, $A^k$, $(sI{-}A)^{-1}$ a través de $J$.
        \item Explica términos $t^k e^{\lambda t}$ en las respuestas si hay bloques de tamaño $m$ ($k=0,\dots,m{-}1$).
      \end{itemize}
    \end{block}
  \end{column}
  \begin{column}{0.5\textwidth}
    \begin{block}{Bloque $J_m(\lambda)$}
      \[
      J_m(\lambda)=\begin{pmatrix}
        \lambda & 1 &  & 0\\
         & \lambda & \ddots & \\
         &  & \ddots & 1\\
        0 &  &  & \lambda
      \end{pmatrix}
      \]
    \end{block}
    \begin{block}{Exponencial de un bloque}
      \[
      e^{J_m(\lambda)\,t}=e^{\lambda t}
      \begin{pmatrix}
        1 & t & t^2/2! & \cdots & t^{m-1}/(m-1)!\\
        0 & 1 & t & \cdots & t^{m-2}/(m-2)!\\
        \vdots &  & \ddots & \ddots & \vdots\\
        0 & \cdots & 0 & 1 & t\\
        0 & \cdots & 0 & 0 & 1
      \end{pmatrix}
      \]
    \end{block}
  \end{column}
\end{columns}
\begin{block}{Consecuencia para $e^{At}$}
  $e^{At}=P\,e^{Jt}P^{-1}$, con $e^{Jt}=\operatorname{diag}\big(e^{J_{m_i}(\lambda_i)t}\big)$. Si existen bloques de tamaño $m$, aparecen términos $t^k e^{\lambda t}$ en la respuesta ($k=0,\dots,m{-}1$).
\end{block}
\end{frame}

%%%%%%%%%%%%%%%%%%%%%%%%
\begin{frame}{Respuesta impulsional y funciones base}
\footnotesize
\begin{columns}[T]
  \begin{column}{0.58\textwidth}
    \begin{block}{Respuesta impulsional $h$ (qué es)}
      Es la salida ante una entrada impulso unitario; actúa como \emph{núcleo de convolución}:
      \[ y_s(t)=(h*u)(t)=\int_{0}^{t} h(t-\tau)\,u(\tau)\,d\tau,\quad Y_s(s)=H(s)\,U(s). \]
    \end{block}
    \begin{block}{Cómo calcular $h$ (LTI)}
      Continuo: 
      \[ H(s)=C(sI{-}A)^{-1}B+D,\quad h(t)=C\,e^{At}\,B + D\,\delta(t)=\mathcal L^{-1}\!\{H(s)\}. \]
      Discreto:
      \[ H(z)=C(zI{-}A)^{-1}B+D,\quad h[0]=D,\; h[k]=C\,A^{k-1}B\ (k\ge1)=\mathcal Z^{-1}\!\{H(z)\}. \]
    \end{block}
  \end{column}
  \begin{column}{0.36\textwidth}
    \begin{block}{Funciones base (RENC)}
      Respuesta a entrada cero: $y_0(t)=C\,\Phi(t,t_0)\,x(t_0)$ con $\Phi(t,t_0)=e^{A(t-t_0)}$.
      Si $A$ tiene autovalores $\lambda_i$ con bloques de Jordan de tamaño $m_i$:
      \[ \{\, t^{k}e^{\lambda_i t}\;:\; i=1,\dots,r,\ k=0,\dots,m_i{-}1\,\}. \]
      En discreto (análogo): $\{\, \ell\text{-potencias}\cdot \lambda_i^{\,k}\,\}$, típicamente $\{\, k^{\ell}\lambda_i^{\,k}\,\}$ con $\ell=0,\dots,m_i{-}1$.
    \end{block}
    \begin{block}{Uso}
      \begin{itemize}\itemsep2pt
        \item $h$ determina la RESC vía convolución y caracteriza la BIBO en SISO.
        \item Las funciones base describen la RENC; revelan modos ($e^{\lambda t}$) y cadenas de Jordan ($t^k e^{\lambda t}$).
      \end{itemize}
    \end{block}
  \end{column}
\end{columns}
\end{frame}


%%%%%%%%%%%%%%%%%%%%%%%%
\begin{frame}{Estabilidad: BIBS y BIBO}
\footnotesize
\begin{columns}[T]
  \begin{column}{0.56\textwidth}
    \begin{block}{¿Qué miden?}
      \begin{itemize}\itemsep2pt
        \item \textbf{BIBS} (bounded–input bounded–state): con entradas y C.I. acotadas, el \emph{estado} permanece acotado (estabilidad interna).
        \item \textbf{BIBO} (bounded–input bounded–output): con entradas acotadas, la \emph{salida} permanece acotada (estabilidad externa).
      \end{itemize}
    \end{block}
    \begin{block}{Criterios prácticos (LTI)}
      \begin{itemize}\itemsep2pt
        \item \textbf{BIBS}: continuo $\Rightarrow \mathrm{Re}\{\lambda_i(A)\}<0$;\; discreto $\Rightarrow |\lambda_i(A)|<1$.
        \item \textbf{BIBO} (SISO): continuo $\int_0^{\infty}|h(t)|\,dt<\infty$ $\Leftrightarrow$ polos de $H(s)$ en $\mathrm{Re}\,s<0$;\\
        discreto $\sum_{k\ge 0}|h[k]|<\infty$ $\Leftrightarrow$ polos de $H(z)$ dentro del disco unidad.
      \end{itemize}
    \end{block}
  \end{column}
  \begin{column}{0.42\textwidth}
    \begin{block}{Relación y matices}
      \begin{itemize}\itemsep2pt
        \item \textbf{BIBS $\Rightarrow$ BIBO}. Si la realización es \emph{mínima} (controlable y observable), \textbf{BIBS $\Leftrightarrow$ BIBO}.
        \item La BIBO depende de $H(\cdot)$ (por $C,D$); la BIBS depende solo de $A$.
        \item En la \emph{frontera} (autovalores en eje imaginario o $|z|{=}1$) puede haber estabilidad \emph{marginal} (no BIBO si hay polos repetidos).
      \end{itemize}
    \end{block}
  \end{column}
\end{columns}
\end{frame}


%%%%%%%%%%%%%%%%%%%%%%%%
\begin{frame}{Controlabilidad y observabilidad: qué, cómo y para qué}
\footnotesize
\begin{columns}[T]
  \begin{column}{0.52\textwidth}
    \begin{block}{¿Qué miden?}
  	\textbf{Controlabilidad}: alcanzar cualquier estado con entradas adecuadas.\\
  	\textbf{Observabilidad}: reconstruir el estado a partir de $u(\cdot)$ y $y(\cdot)$.
    \end{block}
    \begin{block}{Kalman (criterio de rango)}
      \begin{align}
        \mathcal C &= [\,B\;AB\;\cdots\;A^{n-1}B\,], & \operatorname{rank}(\mathcal C){=}n &\iff \text{controlable},\\
        \mathcal O &= \begin{bmatrix} C\\ CA\\ \vdots\\ CA^{n-1}\end{bmatrix}, & \operatorname{rank}(\mathcal O){=}n &\iff \text{observable}.
      \end{align}
    \end{block}
  \end{column}
  \begin{column}{0.47\textwidth}
    \begin{block}{Gramianos (continuo, $A$ Hurwitz)}
      \begin{align}
        A W_c + W_c A^\top + B B^\top &= 0, & W_c\succ 0 &\iff \text{controlable},\\
        A^\top W_o + W_o A + C^\top C &= 0, & W_o\succ 0 &\iff \text{observable}.
      \end{align}
    \end{block}
    \begin{block}{Implicancias prácticas}
      \begin{itemize}\itemsep2pt
        \item \textbf{Realización mínima}: controlable y observable $\Rightarrow$ sin modos ocultos.
        \item \textbf{Ubicación de polos} (REN(C)): factible $\iff$ controlable; \textbf{observador} (RENC): factible $\iff$ observable.
        \item Modos incontrolables no pueden estabilizarse; modos inobservables no aparecen en $y$.
      \end{itemize}
    \end{block}
  \end{column}
\end{columns}
\end{frame}



%%%%%%%%%%%%%%%%%%%%%%%%
\section{Pregunta 1}
\begin{frame}{Pregunta \#1}
\begin{block}{Enunciado Pregunta \#1}
Considere un sistema modelado por la siguiente ecuación diferencial:
  \begin{equation}
    \ddot y + 2\dot y - 15 y = u.
  \end{equation}

  \begin{enumerate}
    \item Encuentre la función de transferencia del sistema.
    \item Formule el sistema en variables de estado.
    \item Obtenga la MTE del sistema y encuentre las funciones base.
    \item Encuentre la respuesta al impulso del sistema.
    \item Determine la estabilidad BIBS y BIBO del sistema.
  \end{enumerate}
\end{block}
\end{frame}
%%%%%%%%%%%%%%%%%%%%%%
\section{Pregunta 2}
\begin{frame}{Pregunta \#2}
  \begin{block}{Enunciado Pregunta \#2}
   Considere el siguiente sistema formulado en variables de estado:
\begin{equation}
  \dot{x}(t) =
  \begin{pmatrix}
    -2 & 1 & 0 & 0 \\
     0 & -2 & 1 & 0 \\
     0 &  0 & -2 & 0 \\
     0 &  0 &  0 & 1
  \end{pmatrix} x(t)
  +
  \begin{pmatrix}
    1 \\
    2 \\
    3 \\
    0
  \end{pmatrix} u(t),
  \qquad
  y(t) = \begin{pmatrix} 1 & 0 & 1 & 1 \end{pmatrix} x(t).
\end{equation}

\begin{enumerate}
  \item Encuentre la MTE y las funciones base del sistema.
  \item Encuentre la respuesta al impulso del sistema.
  \item Determine estabilidad BIBS y BIBO.
  \item Determine observabilidad y controlabilidad.
\end{enumerate}

  \end{block}
\end{frame}

%%%%%%%%%%%%%%%%%%%%%%


\end{document}
