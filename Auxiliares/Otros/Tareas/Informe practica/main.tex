% Template:     Informe LaTeX
% Documento:    Archivo principal
% Versión:      8.3.6 (23/08/2024)
% Codificación: UTF-8
%
% Autor: Pablo Pizarro R.
%        pablo@ppizarror.com
%
% Manual template: [https://latex.ppizarror.com/informe]
% Licencia MIT:    [https://opensource.org/licenses/MIT]

% CREACIÓN DEL DOCUMENTO
\documentclass[
	spanish, % Idioma: spanish, english, etc.
	letterpaper, oneside
]{article}

% INFORMACIÓN DEL DOCUMENTO
\def\documenttitle {Informe de practica II}
\def\documentsubtitle {}
\def\documentsubject {Trabajo en proyecto CHART}

\def\documentauthor {Erik Saez}
\def\coursename {Práctica Profesional II}
\def\coursecode {EL5115-1}

\def\universityname {Universidad de Chile}
\def\universityfaculty {Facultad de Ciencias Físicas y Matemáticas}
\def\universitydepartment {Departamento de Ingeniería Eléctrica}
\def\universitydepartmentimage {departamentos/die}
\def\universitydepartmentimagecfg {height=1.57cm}
\def\universitylocation {Santiago de Chile}

% INTEGRANTES, PROFESORES Y FECHAS
\def\authortable {
	\begin{tabular}{ll}
		Integrantes:
		& \begin{tabular}[t]{l}
			Erik Saez
		\end{tabular} \\
		Profesor:
		& \begin{tabular}[t]{l}
			Santiago Bradford V.
		\end{tabular} \\
		Ayudantes:
		& \begin{tabular}[t]{l}
			Pablo A. Cornejo
		\end{tabular} \\
		\multicolumn{2}{l}{} \\
		& \\
		\multicolumn{2}{l}{Fecha de realización: \today} \\
		\multicolumn{2}{l}{Fecha de entrega: \today} \\
		\multicolumn{2}{l}{\universitylocation}
	\end{tabular}
}

% IMPORTACIÓN DEL TEMPLATE
\input{template}

% INICIO DE PÁGINAS
\begin{document}

% PORTADA
\templatePortrait

% CONFIGURACIÓN DE PÁGINA Y ENCABEZADOS
\templatePagecfg

% RESUMEN O ABSTRACT
\begin{abstractd}

El presente informe detalla las actividades realizadas durante la práctica profesional en el Laboratorio de Ondas Milimétricas (MWL), específicamente en el marco del proyecto CHART. Este proyecto tuvo como objetivo diseñar y optimizar un arreglo de antenas capaz de detectar ráfagas rápidas de radio (FRBs) en un rango de frecuencias de 300 a 500 MHz. En particular, se desarrolló una antena Log-Periodic Dipole Array (LPDA), cuyo diseño y fabricación se realizaron siguiendo un proceso de parametrización y simulación avanzada mediante el uso de software especializado como HFSS.\\\\
Se destacan varias fortalezas a lo largo del proceso, incluyendo el conocimiento previo en simulación y la capacidad de adaptación al equipo de trabajo. Sin embargo, también se identificaron debilidades como la falta de experiencia en la fabricación de antenas, que generó algunos problemas iniciales que fueron corregidos mediante la colaboración con otros miembros del equipo.\\\\
Los resultados obtenidos durante la práctica contribuyeron de manera significativa al avance del proyecto CHART, logrando una antena funcional que cumplió con los parámetros de diseño establecidos. Asimismo, se adquirieron competencias en áreas clave como el trabajo en equipo, la comunicación efectiva y el análisis técnico, que serán de gran utilidad en futuros proyectos profesionales.

\end{abstractd}

% TABLA DE CONTENIDOS - ÍNDICE
\templateIndex

% CONFIGURACIONES FINALES
\templateFinalcfg

% ======================= INICIO DEL DOCUMENTO =======================

% Template:     Informe LaTeX
% Documento:    Archivo de ejemplo
% Versión:      8.3.6 (23/08/2024)
% Codificación: UTF-8
%
% Autor: Pablo Pizarro R.
%        pablo@ppizarror.com
%
% Manual template: [https://latex.ppizarror.com/informe]
% Licencia MIT:    [https://opensource.org/licenses/MIT]

% ------------------------------------------------------------------------------
% NUEVA SECCIÓN
% ------------------------------------------------------------------------------
% Las secciones se inician con \section, si se quiere una sección sin número se
% pueden usar las funciones \sectionanum (sección sin número) o la función
% \sectionanumnoi para crear el mismo título sin numerar y sin aparecer en el índice
\section{Lineas de transmision}
Una línea de transmisión microstrip es un tipo de línea de transmisión utilizada ampliamente en circuitos de alta frecuencia, especialmente en aplicaciones de microondas. Está formada por una cinta metálica (normalmente cobre) que se coloca sobre un sustrato dieléctrico, con un plano de masa debajo. El material dieléctrico entre la cinta y el plano de masa influye en la velocidad de propagación y las características de la señal transmitida, la representacion sigue el modelo clasico de lineas de tranmision.\\\\
El funcionamiento de una línea microstrip se basa en guiar las ondas electromagnéticas a través de la cinta metálica, con parte del campo eléctrico propagándose a través del dieléctrico y otra parte en el aire. A diferencia de otras líneas de transmisión, como las líneas coaxiales, el microstrip es más fácil de integrar en circuitos de microondas y RF (radiofrecuencia) porque puede fabricarse directamente en placas de circuito impreso.\\\\
La idea detrás de estas líneas es transmitir señales de alta frecuencia con bajas pérdidas y buena eficiencia, lo que es esencial en dispositivos como transmisores, receptores y otros sistemas de comunicación.Las antenas patch están directamente relacionadas con las líneas de transmisión microstrip porque suelen utilizarse en el mismo tipo de tecnología. En una antena patch, el parche metálico que actúa como elemento radiador se alimenta mediante una línea de transmisión microstrip. Este tipo de alimentación es ideal porque permite una integración directa de la antena en el mismo sustrato que el circuito, reduciendo el tamaño total del dispositivo y minimizando pérdidas en la conexión entre la antena y el resto del circuito. % Ejemplo, se puede borrar

% FIN DEL DOCUMENTO
\end{document}