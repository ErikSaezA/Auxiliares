% Template:     Control LaTeX
% Documento:    Archivo principal
% Versión:      5.3.2 (12/04/2024)
% Codificación: UTF-8
%
% Autor: Pablo Pizarro R.
%        pablo@ppizarror.com
%
% Manual template: [https://latex.ppizarror.com/controles]
% Licencia MIT:    [https://opensource.org/licenses/MIT]

% CREACIÓN DEL DOCUMENTO
\documentclass[
	spanish, % Idioma: spanish, english, etc.
	letterpaper, oneside
]{article}

% INFORMACIÓN DEL DOCUMENTO
\def\documenttitle {Control 3}
\def\evaluationindication {\textbf{}}

\def\documentauthor {Nombre del autor}
\def\coursename {Análisis y Diseño de Circuitos Eléctricos}
\def\coursecode {EL3101-2}

\def\universityname {Universidad de Chile}
\def\universityfaculty {Facultad de Ciencias Físicas y Matemáticas}
\def\universitydepartment {Departamento de Ingeniería eléctrica}
\def\universitydepartmentimage {departamentos/die}
\def\universitydepartmentimagecfg {height=1.75cm}
\def\universitylocation {Santiago de Chile}

% EQUIPO DOCENTE
\def\teachingstaff {
	\textbf{Profesor: Santiago Bradford V.} \\
	Auxiliares: Byron Castro, Rodrigo Catalán, Erik Sáez. \\
Ayudantes: Benjamín Bruhn, Joaquín Herrera, Nicolás Mayolafquén, César Olivares, Felipe Vargas, Simón Vidal. \\
}

% IMPORTACIÓN DEL TEMPLATE
\input{template}

% INICIO DE PÁGINAS
\begin{document}

% CONFIGURACIÓN DE PÁGINA Y ENCABEZADOS
\templatePagecfg

\begin{enumerate}
    \item 
    Para el circuito de la figura, si el interruptor ha estado conectado a la fuente de 10 [V] durante un tiempo muy largo y en un instante que denominaremos $t = 0$ pasa a conectar la fuente $v_s$. Si $v_s(t) = 6e^{-3t}u(t)$[V]  determine para el voltaje $v_c(t)$:

    \begin{enumerate}
     \item[a)] La respuesta de entrada cero para $v_c(t)$ \textbf{(3 puntos)}
     \item[b)] La respuesta de estado cero $v_c(t)$ \textbf{(3 puntos)}
    \end{enumerate}

    
    \begin{figure}[h!]
        \centering
        \includegraphics[width=0.5\linewidth]{img/Figura_4.png}
        \caption{Circuito P1.}
        \label{fig:p1}
    \end{figure}
    
    \item 
    \begin{enumerate}[label=(\alph*)]
        \item Diseñe un circuito sin utilizar inductores que tenga la siguiente función de transferencia \textbf{(2 puntos)}:
        \begin{equation}
            T(s) = \frac{250000}{s(s + 800)(s + 10)}
        \end{equation}
        
      Siguiendo con la exploración del laboratorio del control anterior, encuentras una nueva sección dedicada a simulaciones espaciales que te pide diseñar lo siguiente para poder salir de la sala del control.

        \item Diseñe un filtro pasa bajos con frecuencia de corte $\omega_c = 1800~\text{rad/s}$, utilizando un mínimo de 4 elementos donde al menos 2 deben estar en paralelo y 2 en serie (considerando como elemento resistencias, inductancias y capacitancias segun prefiera). Aplique factores de escala de ser necesario. \textbf{(2 puntos)}
    
        \item Diseñe un filtro pasa altos con frecuencia de corte $\omega_c = 3600~\text{rad/s}$, utilizando un mínimo de 4 elementos donde al menos 2 deben estar en paralelo y 2 en serie (considerando como elemento resistencias, inductancias y capacitancias segun prefiera). Aplique factores de escala de ser necesario. \textbf{(2 puntos)}

    \end{enumerate}

    \textit{Indicación: Todas las resistencias deben ser mayores o iguales a $10\,k\Omega$}

    \item Sea la siguiente función de transferencia:
    \begin{equation}
        H(s) = \frac{5(s + 1)}{s(s + 5)(s + 20)}
    \end{equation}
    Determine:
    \begin{enumerate}[label=(\alph*)] 
        \item Diagrama de magnitud \textbf{(2 puntos)}
        \item Diagrama de fase \textbf{(2 puntos)}
        \item En base a el diagrama de bode de magnitud, determine la funcion de transferencia: \textbf{(2 puntos)}
    \begin{figure}
        \centering
        \includegraphics[width=0.7\linewidth]{img/Figura_5.png}
        \caption{Diagrama de Bode.}
        \label{fig:p4}
    \end{figure}
    \end{enumerate}
    \end{enumerate}

\end{document}
