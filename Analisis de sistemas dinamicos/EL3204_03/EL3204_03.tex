\documentclass[
  11pt,
  letterpaper,
   addpoints,
   answers
  ]{exam}

\usepackage{../exercise-preamble}

\begin{document}

\noindent
\begin{minipage}{0.47\textwidth}
\includegraphics[width=\textwidth]{../fcfm_die}
\end{minipage}
\begin{minipage}{0.53\textwidth}
\begin{center} 
\large\textbf{Análisis de Sistemas Dinámicos y Estimación} (EL3103) \\
\large\textbf{Clase auxiliar 5} \\
\normalsize Prof.~Heraldo Rozas.\\
\normalsize Prof.~Aux.~Erik Saez - Maximiliano Morales
\end{center}
\end{minipage}

\vspace{0.5cm}
\noindent
\vspace{.85cm}

\begin{questions}
    %%%%%%%%%%%%%%%%%%%%%%%%%%%
    \question Considere el sistema caracterizado por la siguiente función de transferencia:
    \begin{align}
    H(s) = \frac{3(s + 2)}{s^2 - 2s - 15}
    \end{align}
    \begin{enumerate}
        \item Formule el sistema en variables de estado, y calcule MTE y funciones base.
        \item Calcule la respuesta al impulso, y determine estabilidad BIBS y BIBO.
        \item Escriba la expresión general para la respuesta del sistema ante una entrada arbitraria y para condiciones iniciales arbitrarias.
        \item Analice controlabilidad y observabilidad del sistema.
        \item Suponiendo que solamente tiene acceso a la salida del sistema y no al estado, diseñe un controlador que ubique los polos a lazo cerrado en $-5$ y $-3$.
    \end{enumerate}
    %%%%%%%%%%%%%%%%%%%%%%%%%%%
    \begin{solution}
        \subsection*{Resolucion 1.1}
        Se busca obtener la representacion en variables de estado del sistema,ademas de obtener la MTE.Para comenzar encontraremos los polos y ceros de la funcion de transferencia:
        \begin{align}
            H(s)= \frac{3(s+2)}{s^{2}-2s-15}= \frac{3(s+2)}{(s-5)(s+3)}
        \end{align}
        Una vez expresada de una manera factorizada la funcion de transferencia, es posible utilizar fracciones parciales,tal que:
        \begin{align}
            H(s) = \frac{A}{s-5} + \frac{B}{s+3}
        \end{align}
        Luego formaremos un sistema de ecuaciones tal que:
        \begin{align}
            A(s+3) + B(s-5) &= 3(s+2)\\
            (A+B)s + 3A-5B &= 3s + 6
        \end{align}
        Se obtiene que:
        \begin{align}
            A+B &= 3\\
            3A-5B &= 6
        \end{align}
        Por tanto se obtiene que $A=\frac{21}{8}$ y $B=\frac{3}{8}$, con lo que se obtiene que:
        \begin{align}
            H(s) = \frac{21}{8(s-5)} + \frac{3}{8(s+3)}
        \end{align}
        Dado que la definicion de funcion de transferencia viene dada por $H(s)=\frac{Y(s)}{U(s)}$,luego tenemos que:
        \begin{align}
            Y(s)= \frac{21}{8(s-5)}U(s) + \frac{3}{8(s+3)}U(s)
        \end{align}
        donde es posible escribir de manera conveniente la salida tal que:
        \begin{align}
            Y(s) &= 
            \underbrace{
                \begin{pmatrix}
                \frac{21}{8} & \frac{3}{8}
                \end{pmatrix}
            }_{= \mathbf{C}}
            \underbrace{
                \begin{pmatrix}
                \frac{U(s)}{s+5} \\
                \frac{U(s)}{s+3}
                \end{pmatrix}
            }_{= \mathbf{X(s)}}
            \end{align}
    La nocion sobre realizar esto,esque podemos determinar cual es la forma de la matriz C de manera directa y ademas poder conocer el vector de estados,en base a esto tendremos lo siguiente, con el fin de obtener las matrices A y B:
    \begin{align}
        \mathbf{X}(s) &= 
        \begin{pmatrix}
        X_1(s) \\
        X_2(s)
        \end{pmatrix}
        = 
        \begin{pmatrix}
        \frac{U(s)}{s+5} \\
        \frac{U(s)}{s+3}
        \end{pmatrix}.
        \end{align}
    Luego podemos analizar por componente,es decir que:
    \begin{align}
        X_{1}(s)= \frac{U(s)}{s-5}\\
        X_{1}(s)(s-5) = U(s)\\
        sX_{1}(s) - 5X_{1}(s) = U(s)
    \end{align}
    Aplicando la antitransformada,considerando que $\mathcal{L} \left\{ \frac{d^n}{dt^n} f(t) \right\} = s^n F(s)$,tenemos que la expresion en el dominio del tiempo vendra dada:
    \begin{align}
        \dot{x}_{1}(t) - 5x_{1}(t) = u(t)\\
        \dot{x}_{1}(t)  = 5x_{1}(t)+ u(t)
    \end{align}
    De manera analoga tenemos que para $X_{2}(s)$ se tiene que:
    \begin{align}
        X_{2}(s)= \frac{U(s)}{s+3}\\
        X_{2}(s)(s+3) = U(s)\\
        sX_{2}(s) + 3X_{2}(s) = U(s)
    \end{align}
    Aplicando la antitransformada,se tiene que:
    \begin{align}
        \dot{x}_{2}(t) + 3x_{2}(t) = u(t)\\
        \dot{x}_{2}(t) = -3x_{2}(t) + u(t)
    \end{align}
    Una vez obtenidas las variables de estado en el dominio del tiempo , tenemos que podemos formar la matriz A y B de la siguiente manera:
    \begin{align}
        \frac{d}{dt} 
        \begin{pmatrix}
        x_1 \\
        x_2
        \end{pmatrix}
        =
        \begin{pmatrix}
        5 & 0 \\
        0 & -3
        \end{pmatrix}
        \begin{pmatrix}
        x_1 \\
        x_2
        \end{pmatrix}
        +
        \begin{pmatrix}
        1 \\
        1
        \end{pmatrix} u.
    \end{align}
    Donde se tendra que la primera matriz sera la matriz A y la segunda matriz sera la matriz B,es decir:
    \begin{align}
        A = \begin{pmatrix}
        5 & 0 \\
        0 & -3
        \end{pmatrix}
        \quad
        B = \begin{pmatrix}
        1 \\
        1
        \end{pmatrix}
    \end{align}
    De esta manera se obtiene el vector de estados x(t) y las matrices A B y C que permiten realizar la formulacion en variables de estado.La cual es posible expresarla como:
    \begin{align}
        \dot{x}(t) &= \begin{pmatrix} 5 & 0 \\ 0 & -3 \end{pmatrix} x(t) + \begin{pmatrix} 1 \\ 1 \end{pmatrix} u(t) \\
        y(t) &= \begin{pmatrix} \frac{21}{8} & \frac{3}{8} \end{pmatrix} x(t)
    \end{align}
    Notamos que la matriz A es diagonal,lo que nos permite de manera directa obtener los matriz de transicion de estados,la cual vendra dada por:
    \begin{align}
        \Phi(t) = e^{At} = \begin{pmatrix} e^{5t} & 0 \\ 0 & e^{-3t} \end{pmatrix}
    \end{align}
    \subsection*{Resolucion 1.2}
    Se busca el analizar la respuesta al impulso ademas de la estabilidad BIBS y BIBO,para el primer caso tenemos:
    \begin{align}
        h(t)=C\phi(t)B
    \end{align}
    Luego reemaplzaando los valores obtenidos anteriormente se tiene que:
    \begin{align}
        y(t) &= \begin{pmatrix} \frac{21}{8} & \frac{3}{8} \end{pmatrix} 
        \begin{pmatrix} e^{5t} & 0 \\ 0 & e^{-3t} \end{pmatrix}
        \begin{pmatrix} 1 \\ 1 \end{pmatrix} \\
        &= \begin{pmatrix} \frac{21}{8} e^{5t} & \frac{3}{8} e^{-3t} \end{pmatrix} 
        \begin{pmatrix} 1 \\ 1 \end{pmatrix} \\
        &= \frac{21}{8} e^{5t} + \frac{3}{8} e^{-3t}
        \end{align}
    De esta manera se obtiene que la respuesta al impulso vendra dada por:
    \begin{align}
        h(t) = \frac{21}{8} e^{5t} + \frac{3}{8} e^{-3t}
    \end{align}
    Luego buscamos analizar la estabilidad BIBS y BIBO.
    \subsubsection*{Estabilidad BIBS}
    Esta estabilidad esta asociada a los estados de sus sistena (Es decir que estos no divergan para un $t \rightarrow \infty$).Observamos que la matriz A es diagonal,por lo que los valores propios(o polos) son directos,es decir que $\lambda_{1}=5$ y $\lambda_{2}=-3$,donde se observa directamente que $\lambda_{1}$ es positivo,lo que implica que nuestro sistema internamente es inestable, es decir que no es BIBS estable.
    \subsubsection*{Estabilidad BIBO}
    Para la estabilidad BIBO asociada a la salida nos interesa que se cumpla:
    \begin{align}
        \int_{0}^{\inf}|h(t)|dt= \int_{0}^{\infty} |h(t)| dt < \infty
    \end{align}
    Reemplazando anterior y considerando que tenemos un $t \in [0,\infty]$ luego el valor absoluto sera positivo,por tanto:
    \begin{align}
        \int_{0}^{\infty} |h(t)| \, dt &= \int_{0}^{\infty} \left( \frac{21}{8} e^{5t} + \frac{3}{8} e^{-3t} \right) dt \\
        &= \frac{21}{8} \int_{0}^{\infty} e^{5t} \, dt + \frac{3}{8} \int_{0}^{\infty} e^{-3t} \, dt \\
        &= \frac{21}{8} \left[ \frac{e^{5t}}{5} \right]_{0}^{\infty} + \frac{3}{8} \left[ \frac{e^{-3t}}{-3} \right]_{0}^{\infty} \\
        &= \frac{21}{8} \cdot \frac{1}{5} + \frac{3}{8} \cdot \frac{1}{3}.
    \end{align}
  
\end{solution}
    %%%%%%%%%%%%%%%%%%%%%%%%%%%
    \question Considere la siguiente funcion a tiempo discreto dada por:
    \begin{align}
        x(n+3) + 6x(n+2) + 11x(n+1) + 6x(n) = 2u(n+1) + 6u(n) 
    \end{align}
    \begin{enumerate}
        \item Obtenga la funcion de transferencia del sistema 
        \item Obtenga la respuesta al impulso 
    \end{enumerate}
    %%%%%%%%%%%%%%%%%%%%%%%%%%%
    \begin{solution}
        \subsection*{Resolucion 2.1}
        Se busca obtener la funcion de transferencia del sistema, pero es importante notar que ahora estmaos considerando sistemas a tiempo discreto, por lo que deberemos recurrir a la transformada Z en lugar de la transformada de Laplace, la cual se define como:
        \begin{align}
            \mathcal{Z}\{f(t)\} = F(z) = \sum_{t=0}^{\infty} f(t)z^{-t}
        \end{align}
        Al igual que el ejercicio anterior, utilziaremos sus propiedades de lienalidad , y en particular una de estas la cual nos habla de los retardos:
        \begin{align}
            \mathcal{Z}\{f(t-k)\} = z^{-k}F(z)
        \end{align}
        Por lo tanto tomando la transformada Z de la funcion a tiempo discreto se obtiene:
        \begin{align}
            z^{3}X(z) + 6z^{2}X(z) + 11zX(z) + 6X(z) = 2zU(z) + 6U(z)
        \end{align}
        Con lo que si factorizamos X(z) se obtiene que:
        \begin{align}
        X(z)(z^{3} + 6z^{2} + 11z + 6) &= U(z)(2z + 6)\\
        G(z)=\frac{X(z)}{U(z)}&= \frac{2z+6}{z^{3} + 6z^{2} + 11z + 6}
        \end{align}
        Con lo que la funcion de transferencia del sistema vendra dada por:
        \begin{align}
            G(z) = \frac{2z+6}{z^{3} + 6z^{2} + 11z + 6}
        \end{align}
        \subsection*{Resolucion 2.2}
        Similar a el problema anterior, se busca obtener la respuesta a el impulso que analogamente se cumple que $u(t)= \sigma(t)$ y que por tanto $U(z)=1$, luego se debera factorizar el polinomio obtenido con anterioridad con el fin de obtener las fracciones parciales, lo que se puede hacer de la siguiente manera:
        \begin{align}
            z^{3} + 6z^{2}+11z+6 = (z+1)(z+2)(z+3)
        \end{align}
        Con lo que:
        \begin{align}
            G(z) = \frac{2z+6}{(z+1)(z+2)(z+3)} = \frac{A}{z+1} + \frac{B}{z+2} + \frac{C}{z+3}
        \end{align}
        Formamos nuestros sitemas de ecuaciones los cuales vendran dados por:
        \begin{align}
            A(z+2)(z+3) + B(z+1)(z+3) + C(z+1)(z+2) &= 2z+6\\
            Az^{2} + 5Az + 6A + Bz^{2} + 4Bz + 3B + Cz^{2} + 3Cz + 2C &= 2z+6\\
            (A+B+C)z^{2} + (5A+4B+3C)z + (6A+3B+2C) &= 2z+6
        \end{align}
        Con lo que se forma un sistemas de ecuaciones dado por:
        \begin{align}
            A+B+C &= 0\\
            5A+4B+3C &= 2\\
            6A+3B+2C &= 6
        \end{align}\
        Dando como resultado que $A=2$, $B=-2$ y $C=0$, con lo que se obtiene que:
        \begin{align}
            G(z) = \frac{2}{z+1} - \frac{2}{z+2}
        \end{align}
        Dado que queremos obtener g(n) = $\mathcal{Z}^{-1}\{G(z)\}$, se tiene debemos aplicar la antitransformada de Z , pero notamos que no presenta la forma de ninguna de las antitransformadas conocidas, por lo que debemos realizar un ajuste tal que:
        \begin{align}
            G(z) = \frac{2}{z}\frac{z}{z+1} - \frac{2}{z}\frac{z}{z+2}
        \end{align}
        Luego al recurrir a las tablas de antitransformada se tienen lo siguiente:
        \begin{align}
            \mathcal{Z}^{-1} \left\{ \frac{z}{z+\alpha} \right\} &= (-\alpha)^{n}\\
            \mathcal{text}^{-1}\left\{z^{k}F(z)\right\}=f(n+k)
        \end{align}
        con lo que al aplicar la antitransformada se obtiene que tenemos:
        \begin{align}
            g(n) = 2(-1)^{n-1}- 2(-2)^{n-1}
        \end{align}
        Con lo que se obtiene la respuesta al impulso del sistema.
    \end{solution}
    %%%%%%%%%%%%%%%%%%%%%%%%%%%
    \question Considere la siguiente matriz dada por:
    \begin{align}
        A= \begin{bmatrix}
            5 & -8\\
            1 & -1
        \end{bmatrix}
    \end{align}
    \begin{enumerate}
        \item Realize una transformacion tal que $A = TDT^{-1}$ , donde D es una matriz diagonal de valores propios de A y T es una matriz de vectores propios, representadas por:
        \begin{align}
            D = \begin{bmatrix}
                \lambda_{1} & 0\\
                0 & \lambda_{2}
            \end{bmatrix}
            T = \begin{bmatrix}
                v_{1} & v_{2}
            \end{bmatrix}
        \end{align}
    \end{enumerate}
    %%%%%%%%%%%%%%%%%%%%%%%%%%%
    \begin{solution}
        \subsection*{Resolucion 3.1}
        Dada la matriz A se busca obtener una transformacion 
    Para encontrar los valores propios de A , se debe cumplir que para det(A-$\lambda$I) = 0, esto con el fin de que sea singular , es decir que la matriz $A-\lambda I$ no tenga inversa o equivalente a que su determinante sea nulo , para no obtener soluciones triviales en donde el vector propio v sea 0 , dado que por definicion estos son no nulos.
    \begin{align}
        (A-\lambda I)v = 0
    \end{align}
    Por lo tanto se tiene que:
    \begin{align}
        |A-\lambda I| = 0\\
        \begin{vmatrix}
            5-\lambda & -8\\
            1 & -1-\lambda
        \end{vmatrix} = 0\\
        (5-\lambda)(-1-\lambda) - (-8)(1) = 0\\
        \lambda^{2} - 4\lambda - 3 = 0
    \end{align}
    De esta manera se debera cumplir que:
    \begin{align}
        (\lambda -3)(\lambda -1) =0
    \end{align}
    Con lo que finalmente se obtiene que $\lambda_{1} = 3$ y $\lambda_{2} = 1$,y por tanto nuestra matriz D la cual viene dada:
    \begin{align}
        D = \begin{bmatrix}
            3 & 0\\
            0 & 1
        \end{bmatrix}
    \end{align}
     Se busca obtener los vectores propios asociados a estos valores propios, para esto calcularemos el vector proio asociado a $\lambda_{1}$ por tanto:
     \begin{align}
        (A - \lambda_1 I)\mathbf{v}_1 &= 0.
    \end{align}
    
    Desarrollando esta expresión, tenemos
    \begin{align}
        \begin{pmatrix}
        5 - 3 & -8 \\
        1 & -1 - 3
        \end{pmatrix}
        \begin{pmatrix}
        x \\
        y
        \end{pmatrix}
        &= \begin{pmatrix}
        0 \\
        0
        \end{pmatrix}.
    \end{align}
    Esto permite formar un sistema de ecuaciones para x e y , dado por:
    \begin{align}
        2x - 8y &= 0, \\
        x - 4y &= 0,
    \end{align}
    Es importante destacar que son linealmente dependientes, por lo que no existirá una solución única. Considerando esto, basta encontrar algún vector que satisfaga dicha relación. Por ejemplo, si consideramos \( y = 1 \), podemos ver que se debe tener \( x = 4 \). Así, el vector propio \( \mathbf{v}_1 \) es
    \begin{align}
        \mathbf{v}_1 &= \begin{pmatrix} 4 \\ 1 \end{pmatrix}.
    \end{align}
    Analogamente tenemos uqe para el segundo vector proipo se tiene que:
    \begin{align}
        (A - \lambda_2 I)\mathbf{v}_2 &= 0
    \end{align}
    \begin{align}
        \iff
        \begin{pmatrix}
        5 - 1 & -8 \\
        1 & -1 - 1
        \end{pmatrix}
        \begin{pmatrix}
        x \\
        y
        \end{pmatrix}
        &= \begin{pmatrix}
        0 \\
        0
        \end{pmatrix}.
    \end{align}
    Desarrollando, obtenemos el siguiente sistema de ecuaciones
    \begin{align}
        4x - 8y &= 0, \\
        x - 2y &= 0.
    \end{align}
    Nuevamente, podemos ver que las ecuaciones son linealmente dependientes, por lo que considerando \( y = 1 \) tenemos \( x = 2 \). Por lo tanto, el segundo vector propio es
    \begin{align}
        \mathbf{v}_2 &= \begin{pmatrix} 2 \\ 1 \end{pmatrix}.
    \end{align}
    Finalmente es posible el obtener la matriz T, la cual se define como:
    \begin{align}
        T = \begin{bmatrix}
            4 & 2\\
            1 & 1
        \end{bmatrix}
    \end{align}
    La cual esta asociado a los vectores propios, finalmente nos queda el obtener $T^{-1}$ , lo cual se puede hacer de la siguiente manera para una matriz de 2x2 
    \begin{align}
        \begin{bmatrix}
            m_{1} & m_{2}\\
            m_{3} & m_{4}
        \end{bmatrix}
    \end{align}
    
    Con lo que la matriz inversa se define como:
    \begin{align}
        M^{-1} = \frac{1}{\text{det}(M)}\begin{bmatrix}
            m_{4} & -m_{2}\\
            -m_{3} & m_{1}
        \end{bmatrix}
    \end{align}
    
    Por lo tanto, tenemos que:
    \begin{align}
        \text{det}(T) = 4 \cdot 1 - 2 \cdot 1 = 2
    \end{align}
    
    Con lo que se obtiene que para \( T^{-1} \):
    \begin{align}
        T^{-1} = \begin{bmatrix}
            \frac{1}{2} & 1\\
            \frac{-1}{2} & 2
        \end{bmatrix}
    \end{align}
    De esta manera se obtiene la transformación buscada dada por:
    \begin{align}
        A = T D T^{-1} = \begin{bmatrix}
            4 & 2\\
            1 & 1
        \end{bmatrix}
        \begin{bmatrix}
            3 & 0\\
            0 & 1
        \end{bmatrix}
        \begin{bmatrix}
            \frac{1}{2} & 1\\
            -\frac{1}{2} & 2
        \end{bmatrix}
    \end{align}

\end{solution}

    %%%%%%%%%%%%%%%%%%%%%%%%%%%
\end{questions}



\end{document}