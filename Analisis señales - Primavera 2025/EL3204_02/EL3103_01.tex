\documentclass[
  11pt,
  letterpaper,
   addpoints,
   answers
  ]{exam}

\usepackage{../exercise-preamble}

\begin{document}

\noindent
\begin{minipage}{0.47\textwidth}
\includegraphics[width=\textwidth]{../fcfm_die}
\end{minipage}
\begin{minipage}{0.53\textwidth}
\begin{center} 
\large\textbf{Análisis de Sistemas Dinámicos y Estimación} (EL3103) \\
\large\textbf{Clase auxiliar 3} \\
\normalsize Prof.~Heraldo Rozas.\\
\normalsize Prof.~Aux.~Erik Saez - Maximiliano Morales
\end{center}
\end{minipage}

\vspace{0.5cm}
\noindent
\vspace{.85cm}

\begin{questions}
    %%%%%%%%%%%%%%%%%%%%%%%%%%%
    \question Considere el sistema linealizado dado por la siguiente ecuacion diferencial:
    \begin{align}
        \ddot{\theta(t)} =\frac{g}{l}\theta(t) +\frac{1}{l}u(t)
    \end{align}
    \begin{enumerate}
        \item Descomponga la salida como la suma de la respuesta a entrada cero (RENC) y la respuesta a estado cero (RESC), en el dominio de Laplace.
        \item Obtenga la funcion de transferencia del sistema, y encuentre la respuesta al impulso ante condiciones iniciales nulas en el dominio del tiempo
        \item Obtenga la respeusta a entrada cero en el dominio del tiempo, y exprese la respuesta ante una entrada y condiciones inciiales arbitrarias
    \end{enumerate}
    %%%%%%%%%%%%%%%%%%%%%%%%%%%
    \begin{solution}
        \subsection*{Resolucion 1.1}
        Recordemos que la transformada de Laplace no es mas que una transformada del domino del tiempo al domino de las frecuencias complejas, la cual ademas presenta una variedad de propiedades utiles siendo una de las mas importantes la linealidad, es decir que presenta aditividad y homogeneidad.En particular en el curso utilizaremos la transformada de laplace unilateral la cual se define como:
        \begin{align}
            \mathcal{L}\{f(t)\} = F(s) = \int_{0}^{\infty} f(t)e^{-st}dt
        \end{align}
        Dado que se busca el obtener la funcion de transferencia en el dominio de laplace de la ecuacion de segundo grado, luego se tendra que:
        \begin{align}
            \mathcal{L}\{\ddot{\theta(t)}\} &= \mathcal{L}\{\frac{g}{l}\theta(t) +\frac{1}{l}u(t)\}\\
        \end{align}
        Una de las virtudes de la transformada de laplace en presencia de derivadas esque podemos utilizar la siguiente regla generica:
        \begin{align}
            \mathcal{L}\{\ddot{f(t)}\} = s^{n}\mathcal{L}\{f(t)\} - s^{n-1}f(0) - s^{n-2}\dot{f}(0) - ... - f^{(n-1)}(0)
        \end{align}
        Por lo que al aplicar a nuestro caso particular se tendra que:
        \begin{align}
         s^{2}\Theta(s) - s\Theta(0) - \dot{\Theta}(0) = \frac{g}{l}\Theta(s) + \frac{1}{l}U(s) 
        \end{align}
        Reordenando la expresion se llega a lo siguiente:
        \begin{align}
            \Theta(s) = \frac{s\Theta(0) + \dot\Theta(0) }{s^{2}-\frac{g}{l}} + \frac{1}{l(s^{2}-\frac{g}{l})}U(s)
        \end{align}
        Donde se logra identificar de manera directa la respuesta a entrada cero y la respuesta a estado cero, la cual es posible dada la linealidad de la transformada, por lo tanto tenemos que:
        \begin{align}
            \Theta_{RENC} &= \frac{s\Theta(0) + \dot\Theta(0) }{s^{2}-\frac{g}{l}}\\
            \Theta_{RESC} &= \frac{1}{l(s^{2}-\frac{g}{l})}U(s)
        \end{align}
        Con lo que la salida se puede expresar como:
        \begin{align}
            \Theta(s) = \Theta_{RENC} + \Theta_{RESC}
        \end{align}
        \subsection*{Resolucion 1.2}
        Se busca obtener la funcion de transferencia del sistema la cual se define como la relacion entre la salida y la entrada en el dominio de laplace, es de importancia destacar que esta funcion nos da una relacion entre la salida y las entradas , pero que no depende de estas ultimas, es decir que es una \textbf{Propiedad propia del sistema en particular} y que no cambia ni con la representacion las diferentes representaciones que hagamos de nuestros sistemas en variables de estado (Se vera con mas detalle en proximos auxiliares). Esta respuesta se analiza considerando la RESC, y recordando la definicion de esta , en esta caso en particular se cumplira cuando las condiciones iniciales sean nulas, por lo que se tendra que:
        \begin{align}
            \Theta_{RESC} = \frac{1}{l(s^{2}-\frac{g}{l})}U(s)
        \end{align}
        Con lo que:
        \begin{align}
            H(s) = \frac{Y(s)}{U(s)} = \frac{\Theta_{RESC}}{U(s)} = \frac{1}{l(s^{2}-\frac{g}{l})} 
        \end{align}
        Vemos que se cancelan las entradas, podemos reordenarla como :
        \begin{align}
            H(s) = \frac{\frac{1}{l}}{(s+\sqrt{\frac{g}{l}})(s-\sqrt{\frac{g}{l}})}
        \end{align}
        Donde identificamos polos.Luego se busca el obtener la respuesta al impulso , la cual se define como $u(t) = \delta(t)$ en el dominio del tiempo y $U(s) = 1$ en el dominio de laplace, por lo que se tendra que:
        \begin{align}
            H(s) = \frac{Y(s)}{U(s)}=\frac{\Theta_{RESC}}{1}
        \end{align}
        COn lo que debemos aplicar la antitransformada para obtener la respuesta
        \begin{align}
            h(t)= \mathcal{L}^{-1}\{H(s)\}
        \end{align}
        Para realizar esto, es conveniente descomponer la funcion de transferencia en fracciones parciales, lo que se puede hacer de la siguiente manera (Se define $\omega_{0} = \sqrt{\frac{g}{l}}$):
        \begin{align}
            H(s) = \frac{1/l}{(s+\omega_{0})(s-\omega_{0})} = \frac{A}{s+\omega_{0}} + \frac{B}{s-\omega_{0}}
        \end{align}
        Donde los metodos de resolucion para obtener las fracciones los habran visto en cursos pasados, por lo que tenemos que:
        \begin{align}
            A(s-\omega_{0}) + B(s+\omega_{0}) &= \frac{1}{l}\\
            As - A\omega_{0} + Bs + B\omega_{0} &= \frac{1}{l}\\
            (A+B)s + (B-A)\omega_{0} &= \frac{1}{l}
        \end{align}
        Para que esto se cumpla luego:
        \begin{align}
            (A+B)s=0\\
            (B-A)\omega_{0} = \frac{1}{l}
        \end{align}
    Por lo que se tendra que A = -B, con lo que reemplazando
    \begin{align}
        (B-A)\omega_{0} &= \frac{1}{l}\\
        2B\omega_{0} &= \frac{1}{l}\\
        B &= \frac{1}{2l\omega_{0}}
    \end{align}
    Con lo que $A=-\frac{1}{2l\omega_{0}}$ , de desta manera se puede reescribir la funcion de transferencia como:
    \begin{align}
        H(s) = \frac{1}{2l\omega_{0}}\left(\frac{1}{s-\omega_{0}} - \frac{1}{s+\omega_{0}}\right)
    \end{align}
    Utilizar fracciones parciales es util debido a que conocemos sus antitransformada para volver al dominio del tiempo , es decir:
    \begin{align}
        \mathcal{L}^{-1}\left\{\frac{1}{s-a}\right\} = e^{at}
    \end{align}
    Con lo que se obtiene finalmente que:
    \begin{align}
        h(t) = \mathcal{L}^{-1}\{H(s)\} = \frac{1}{2l\omega_{0}}\left(e^{\omega_{0}t} - e^{-\omega_{0}t}\right)
    \end{align}
    \subsection*{Resolucion 1.3}
    Se busca obtener la RENC en el dominio del tiempo ante condiciones iniciales arbitrarias, para esto tenemos que realizar el procedimiento analogo:
    \begin{align}
        \Theta_{RENC} = \frac{s\Theta(0) + \dot{\Theta}(0)}{(s+\omega_{0})(s-\omega_{0})} = \frac{\alpha}{s+\omega_{0}} + \frac{\beta}{s-\omega_{0}}
    \end{align}
    Donde se tiene que:
    \begin{align}
        \alpha(s-\omega_{0}) + \beta(s+\omega_{0}) &= s\Theta(0) + \dot{\Theta}(0)\\
        \alpha s - \alpha\omega_{0} + \beta s + \beta\omega_{0} &= s\Theta(0) + \dot{\Theta}(0)\\
        (\alpha + \beta)s + (\beta - \alpha)\omega_{0} &= s\Theta(0) + \dot{\Theta}(0)
    \end{align}
    Con lo que tenemos por tanto que:
    \begin{align}
        \alpha + \beta &= \Theta(0)\\
        (\beta - \alpha)\omega_{0} &=s\Theta(0) + \dot{\Theta}(0)   
    \end{align}
    Luego despejando $\alpha$ y $\beta$ se tiene que:
    \begin{align}
        \Theta_{RENC}(s) = \frac{\omega_{0}\Theta(0)- \dot{\Theta(0)}}{2\omega_{0}}\frac{1}{s+\omega_{0}} + \frac{\omega_{0}\Theta(0)+ \dot{\Theta(0)}}{2\omega_{0}}\frac{1}{s-\omega_{0}}
    \end{align}
    Con lo que aplicando la antitransformada para volver al dominio del tiempo se obtiene que:
    \begin{align}
        \theta_{RENC}(t) = \frac{\omega_{0}\Theta(0)- \dot{\Theta(0)}}{2\omega_{0}}e^{-\omega_{0}t} + \frac{\omega_{0}\Theta(0)+ \dot{\Theta(0)}}{2\omega_{0}}e^{\omega_{0}t}
    \end{align}
    Por ultimo tenemos que la para la respuesta total debemos tener el cuidado con lo siguiente.Sabemos que:
    \begin{align}
        \Theta(s) = \Theta_{RENC} + \Theta_{RESC}
    \end{align}
    Donde la RESC se define como:
    \begin{align}
        \Theta(s) = \Theta_{RESC} = H(s)U(s)
    \end{align}
    Lo cual viene de la convolucion en el dominio de Laplace que en el tiempo se expresa como :
    \begin{align}
    \mathcal{L}{\{f(t)*g(t)\}} = F(s)G(s)
    \end{align}
    Por lo que volviendo sobre nuestra ecuacion en el dominio del tiempo, esto implicara que 
    \begin{align}
        \theta_{resc}(t) = h(t)*u(t) 
    \end{align}
    Siendo u(t) una entrada arbitraria , luego tenemos que la respuesta total considerando esto ultimo viene expresada por:
    \begin{align}
        \theta(t) &= \frac{\omega_{0}\Theta(0)- \dot{\Theta(0)}}{2\omega_{0}}e^{-\omega_{0}t} + \frac{\omega_{0}\Theta(0)+ \dot{\Theta(0)}}{2\omega_{0}}e^{\omega_{0}t} + h(t)*u(t)\\
        &=\frac{\omega_{0}\Theta(0)- \dot{\Theta(0)}}{2\omega_{0}}e^{-\omega_{0}t} + \frac{\omega_{0}\Theta(0)+ \dot{\Theta(0)}}{2\omega_{0}}e^{\omega_{0}t} + \left( \frac{1}{2\omega_{0}l}(e^{\omega_{0}t} - e^{-\omega_{0}t})\right) * u(t)
    \end{align}
    Con lo que finalmente se obtiene la respuesta final de nuestro sistema en el dominio del tiempo.
    \end{solution}
    %%%%%%%%%%%%%%%%%%%%%%%%%%%
    \question Considere la siguiente funcion a tiempo discreto dada por:
    \begin{align}
        x(n+3) + 6x(n+2) + 11x(n+1) + 6x(n) = 2u(n+1) + 6u(n) 
    \end{align}
    \begin{enumerate}
        \item Obtenga la funcion de transferencia del sistema 
        \item Obtenga la respuesta al impulso 
    \end{enumerate}
    %%%%%%%%%%%%%%%%%%%%%%%%%%%
    \begin{solution}
        \subsection*{Resolucion 2.1}
        Se busca obtener la funcion de transferencia del sistema, pero es importante notar que ahora estmaos considerando sistemas a tiempo discreto, por lo que deberemos recurrir a la transformada Z en lugar de la transformada de Laplace, la cual se define como:
        \begin{align}
            \mathcal{Z}\{f(t)\} = F(z) = \sum_{t=0}^{\infty} f(t)z^{-t}
        \end{align}
        Al igual que el ejercicio anterior, utilziaremos sus propiedades de lienalidad , y en particular una de estas la cual nos habla de los retardos:
        \begin{align}
            \mathcal{Z}\{f(t-k)\} = z^{-k}F(z)
        \end{align}
        Por lo tanto tomando la transformada Z de la funcion a tiempo discreto se obtiene:
        \begin{align}
            z^{3}X(z) + 6z^{2}X(z) + 11zX(z) + 6X(z) = 2zU(z) + 6U(z)
        \end{align}
        Con lo que si factorizamos X(z) se obtiene que:
        \begin{align}
        X(z)(z^{3} + 6z^{2} + 11z + 6) &= U(z)(2z + 6)\\
        G(z)=\frac{X(z)}{U(z)}&= \frac{2z+6}{z^{3} + 6z^{2} + 11z + 6}
        \end{align}
        Con lo que la funcion de transferencia del sistema vendra dada por:
        \begin{align}
            G(z) = \frac{2z+6}{z^{3} + 6z^{2} + 11z + 6}
        \end{align}
        \subsection*{Resolucion 2.2}
        Similar a el problema anterior, se busca obtener la respuesta a el impulso que analogamente se cumple que $u(t)= \sigma(t)$ y que por tanto $U(z)=1$, luego se debera factorizar el polinomio obtenido con anterioridad con el fin de obtener las fracciones parciales, lo que se puede hacer de la siguiente manera:
        \begin{align}
            z^{3} + 6z^{2}+11z+6 = (z+1)(z+2)(z+3)
        \end{align}
        Con lo que:
        \begin{align}
            G(z) = \frac{2z+6}{(z+1)(z+2)(z+3)} = \frac{A}{z+1} + \frac{B}{z+2} + \frac{C}{z+3}
        \end{align}
        Formamos nuestros sitemas de ecuaciones los cuales vendran dados por:
        \begin{align}
            A(z+2)(z+3) + B(z+1)(z+3) + C(z+1)(z+2) &= 2z+6\\
            Az^{2} + 5Az + 6A + Bz^{2} + 4Bz + 3B + Cz^{2} + 3Cz + 2C &= 2z+6\\
            (A+B+C)z^{2} + (5A+4B+3C)z + (6A+3B+2C) &= 2z+6
        \end{align}
        Con lo que se forma un sistemas de ecuaciones dado por:
        \begin{align}
            A+B+C &= 0\\
            5A+4B+3C &= 2\\
            6A+3B+2C &= 6
        \end{align}\
        Dando como resultado que $A=2$, $B=-2$ y $C=0$, con lo que se obtiene que:
        \begin{align}
            G(z) = \frac{2}{z+1} - \frac{2}{z+2}
        \end{align}
        Dado que queremos obtener g(n) = $\mathcal{Z}^{-1}\{G(z)\}$, se tiene debemos aplicar la antitransformada de Z , pero notamos que no presenta la forma de ninguna de las antitransformadas conocidas, por lo que debemos realizar un ajuste tal que:
        \begin{align}
            G(z) = \frac{2}{z}\frac{z}{z+1} - \frac{2}{z}\frac{z}{z+2}
        \end{align}
        Luego al recurrir a las tablas de antitransformada se tienen lo siguiente:
        \begin{align}
            \mathcal{Z}^{-1} \left\{ \frac{z}{z+\alpha} \right\} &= (-\alpha)^{n}\\
            \mathcal{text}^{-1}\left\{z^{k}F(z)\right\}=f(n+k)
        \end{align}
        con lo que al aplicar la antitransformada se obtiene que tenemos:
        \begin{align}
            g(n) = 2(-1)^{n-1}- 2(-2)^{n-1}
        \end{align}
        Con lo que se obtiene la respuesta al impulso del sistema.
    \end{solution}
    %%%%%%%%%%%%%%%%%%%%%%%%%%%
    \question Considere la siguiente matriz dada por:
    \begin{align}
        A= \begin{bmatrix}
            5 & -8\\
            1 & -1
        \end{bmatrix}
    \end{align}
    \begin{enumerate}
        \item Realize una transformacion tal que $A = TDT^{-1}$ , donde D es una matriz diagonal de valores propios de A y T es una matriz de vectores propios, representadas por:
        \begin{align}
            D = \begin{bmatrix}
                \lambda_{1} & 0\\
                0 & \lambda_{2}
            \end{bmatrix}
            T = \begin{bmatrix}
                v_{1} & v_{2}
            \end{bmatrix}
        \end{align}
    \end{enumerate}
    %%%%%%%%%%%%%%%%%%%%%%%%%%%
    \begin{solution}
        \subsection*{Resolucion 3.1}
        Dada la matriz A se busca obtener una transformacion 
    Para encontrar los valores propios de A , se debe cumplir que para det(A-$\lambda$I) = 0, esto con el fin de que sea singular , es decir que la matriz $A-\lambda I$ no tenga inversa o equivalente a que su determinante sea nulo , para no obtener soluciones triviales en donde el vector propio v sea 0 , dado que por definicion estos son no nulos.
    \begin{align}
        (A-\lambda I)v = 0
    \end{align}
    Por lo tanto se tiene que:
    \begin{align}
        |A-\lambda I| = 0\\
        \begin{vmatrix}
            5-\lambda & -8\\
            1 & -1-\lambda
        \end{vmatrix} = 0\\
        (5-\lambda)(-1-\lambda) - (-8)(1) = 0\\
        \lambda^{2} - 4\lambda - 3 = 0
    \end{align}
    De esta manera se debera cumplir que:
    \begin{align}
        (\lambda -3)(\lambda -1) =0
    \end{align}
    Con lo que finalmente se obtiene que $\lambda_{1} = 3$ y $\lambda_{2} = 1$,y por tanto nuestra matriz D la cual viene dada:
    \begin{align}
        D = \begin{bmatrix}
            3 & 0\\
            0 & 1
        \end{bmatrix}
    \end{align}
     Se busca obtener los vectores propios asociados a estos valores propios, para esto calcularemos el vector proio asociado a $\lambda_{1}$ por tanto:
     \begin{align}
        (A - \lambda_1 I)\mathbf{v}_1 &= 0.
    \end{align}
    
    Desarrollando esta expresión, tenemos
    \begin{align}
        \begin{pmatrix}
        5 - 3 & -8 \\
        1 & -1 - 3
        \end{pmatrix}
        \begin{pmatrix}
        x \\
        y
        \end{pmatrix}
        &= \begin{pmatrix}
        0 \\
        0
        \end{pmatrix}.
    \end{align}
    Esto permite formar un sistema de ecuaciones para x e y , dado por:
    \begin{align}
        2x - 8y &= 0, \\
        x - 4y &= 0,
    \end{align}
    Es importante destacar que son linealmente dependientes, por lo que no existirá una solución única. Considerando esto, basta encontrar algún vector que satisfaga dicha relación. Por ejemplo, si consideramos \( y = 1 \), podemos ver que se debe tener \( x = 4 \). Así, el vector propio \( \mathbf{v}_1 \) es
    \begin{align}
        \mathbf{v}_1 &= \begin{pmatrix} 4 \\ 1 \end{pmatrix}.
    \end{align}
    Analogamente tenemos uqe para el segundo vector proipo se tiene que:
    \begin{align}
        (A - \lambda_2 I)\mathbf{v}_2 &= 0
    \end{align}
    \begin{align}
        \iff
        \begin{pmatrix}
        5 - 1 & -8 \\
        1 & -1 - 1
        \end{pmatrix}
        \begin{pmatrix}
        x \\
        y
        \end{pmatrix}
        &= \begin{pmatrix}
        0 \\
        0
        \end{pmatrix}.
    \end{align}
    Desarrollando, obtenemos el siguiente sistema de ecuaciones
    \begin{align}
        4x - 8y &= 0, \\
        x - 2y &= 0.
    \end{align}
    Nuevamente, podemos ver que las ecuaciones son linealmente dependientes, por lo que considerando \( y = 1 \) tenemos \( x = 2 \). Por lo tanto, el segundo vector propio es
    \begin{align}
        \mathbf{v}_2 &= \begin{pmatrix} 2 \\ 1 \end{pmatrix}.
    \end{align}
    Finalmente es posible el obtener la matriz T, la cual se define como:
    \begin{align}
        T = \begin{bmatrix}
            4 & 2\\
            1 & 1
        \end{bmatrix}
    \end{align}
    La cual esta asociado a los vectores propios, finalmente nos queda el obtener $T^{-1}$ , lo cual se puede hacer de la siguiente manera para una matriz de 2x2 
    \begin{align}
        \begin{bmatrix}
            m_{1} & m_{2}\\
            m_{3} & m_{4}
        \end{bmatrix}
    \end{align}
    
    Con lo que la matriz inversa se define como:
    \begin{align}
        M^{-1} = \frac{1}{\text{det}(M)}\begin{bmatrix}
            m_{4} & -m_{2}\\
            -m_{3} & m_{1}
        \end{bmatrix}
    \end{align}
    
    Por lo tanto, tenemos que:
    \begin{align}
        \text{det}(T) = 4 \cdot 1 - 2 \cdot 1 = 2
    \end{align}
    
    Con lo que se obtiene que para \( T^{-1} \):
    \begin{align}
        T^{-1} = \begin{bmatrix}
            \frac{1}{2} & 1\\
            \frac{-1}{2} & 2
        \end{bmatrix}
    \end{align}
    De esta manera se obtiene la transformación buscada dada por:
    \begin{align}
        A = T D T^{-1} = \begin{bmatrix}
            4 & 2\\
            1 & 1
        \end{bmatrix}
        \begin{bmatrix}
            3 & 0\\
            0 & 1
        \end{bmatrix}
        \begin{bmatrix}
            \frac{1}{2} & 1\\
            -\frac{1}{2} & 2
        \end{bmatrix}
    \end{align}

\end{solution}

    %%%%%%%%%%%%%%%%%%%%%%%%%%%
\end{questions}



\end{document}