% Template:     Control LaTeX
% Documento:    Archivo principal
% Versión:      5.3.2 (12/04/2024)
% Codificación: UTF-8
%
% Autor: Pablo Pizarro R.
%        pablo@ppizarror.com
%
% Manual template: [https://latex.ppizarror.com/controles]
% Licencia MIT:    [https://opensource.org/licenses/MIT]

% CREACIÓN DEL DOCUMENTO
\documentclass[
	spanish, % Idioma: spanish, english, etc.
	letterpaper, oneside
]{article}

% INFORMACIÓN DEL DOCUMENTO
\def\documenttitle {Examen}
\def\evaluationindication {\textbf{}}

\def\documentauthor {Nombre del autor}
\def\coursename {Análisis y Diseño de Circuitos Eléctricos}
\def\coursecode {EL3101-2}

\def\universityname {Universidad de Chile}
\def\universityfaculty {Facultad de Ciencias Físicas y Matemáticas}
\def\universitydepartment {Departamento de Ingeniería eléctrica}
\def\universitydepartmentimage {departamentos/die}
\def\universitydepartmentimagecfg {height=1.75cm}
\def\universitylocation {Santiago de Chile}

% EQUIPO DOCENTE
\def\teachingstaff {
	\textbf{Profesor: Santiago Bradford V.} \\
	Auxiliares: Byron Castro, Rodrigo Catalán, Erik Sáez. \\
Ayudantes: Benjamín Bruhn, Joaquín Herrera, Nicolás Mayolafquén, César Olivares, Felipe Vargas, Simón Vidal. \\
}

% IMPORTACIÓN DEL TEMPLATE
\input{template}

% INICIO DE PÁGINAS
\begin{document}

% CONFIGURACIÓN DE PÁGINA Y ENCABEZADOS
\templatePagecfg

\begin{enumerate}
    \item Para la red de la figura en régimen permanente senoidal, calcule la potencia disipada por la resistencia de $10\,\Omega$.


    \begin{figure}[h!]
        \centering
        \includegraphics[width=0.55\linewidth]{img/Examen_1_1.png}
        \caption{Circuito P1.}
        \label{fig:p1}
    \end{figure}
    
   \item Para la red de la figura, en régimen permanente, aplicando análisis nodal determine la corriente $I$ y el voltaje $V_x$. Las fuentes están dadas por:
    \begin{align*}
        V_f &= 20\sqrt{2}\cos(2t - 90^\circ) \\
        i_f &= 5\sqrt{2}\cos(4t)
    \end{align*}

    \begin{figure}[h!]
        \centering
        \includegraphics[width=0.5\linewidth]{img/Figura_6}
        \caption{Circuito P2.}
        \label{fig:p2}
    \end{figure}

    \item Una fuente simétrica y equilibrada de secuencia positiva suministra potencia a tres cargas, siendo una de ellas desconocida, tal como se muestra en la figura 3. Si el voltaje fase-fase es de $208~[V]$ efectivos a $50~[Hz]$, la corriente de línea $98,6~[A]$ efectivos, y el factor de potencia combinado de la carga es de $0,6756$ en atraso.
\begin{enumerate}
    \item[a.] Encuentre la carga desconocida en términos de $kW$ y $kVAR$.
    \item[b.] Determine la capacitancia $C$ por fase a poner en un banco de condensadores en $\Delta$, para corregir el factor de potencia del conjunto a $0,8944$. Realice un diagrama de potencia.
\end{enumerate}
  \begin{figure}[h!]
        \centering
        \includegraphics[width=0.7\linewidth]{img/Examen_1_3.png}
        \caption{Circuito P3.}
        \label{fig:p3}
    \end{figure}
    \end{enumerate}

\end{document}
