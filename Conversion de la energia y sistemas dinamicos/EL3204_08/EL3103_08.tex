\documentclass[
  11pt,
  letterpaper,
   addpoints,
   answers
  ]{exam}

\usepackage{../exercise-preamble}
\usepackage{float}
\begin{document}

\noindent
\begin{minipage}{0.47\textwidth}
\includegraphics[width=\textwidth]{../fcfm_die}
\end{minipage}
\begin{minipage}{0.53\textwidth}
\begin{center} 
\large\textbf{Conversión de la Energía y Sistemas Eléctricos } (EL4111-1) \\
\large\textbf{Clase auxiliar 8} \\
\small Prof.~Constanza Ahumada - Rodrigo Moreno.\\
\small Prof.~Aux.~Javiera Pacheco - Erik Sáez\\
\small Ayudantes.~Manuel Aceituno - Pamela Acuña - Alvaro Flores\\
\end{center}
\end{minipage}

\vspace{0.5cm}
\noindent
\vspace{.85cm}
%---------------------------------------------------------------------------------------------------------------------------------
\begin{questions}
  \question Sea el siguiente set de ecuaciones:
  \begin{itemize}
    \item ¿Cuál es la principal diferencia en los componentes utilizados entre las centrales generadoras convencionales y las centrales basadas en ERNC? Explica cómo esta diferencia afecta los sistemas de control utilizados en cada tipo de central.
    
    \item Explica cómo se genera la potencia en un aerogenerador y describe el rol de la velocidad del viento en la curva de potencia, incluyendo las velocidades de ``Cut-in'' y ``Cut-out''.
    
    \item  ¿Qué es la ley de Betz y cuál es su implicancia en la eficiencia máxima teórica de un aerogenerador?
    
    \item Describe los tipos de aerogeneradores de eje horizontal y de eje vertical, y menciona una ventaja y una desventaja de cada uno.
\end{itemize}
  %---------------------------------------------------------------------------------------------------------------------------------
    \question Considerando los datos de la siguiente tabla, construya las curvas V-I y V-P para el panel fotovoltaico para las siguientes condiciones (indicando los MPP's de cada uno):

    \begin{table}[h!]
        \centering
        \begin{tabular}{|c|c|}
            \hline
            \textbf{Parámetro} & \textbf{Valor} \\
            \hline
            $I_{sc}$ & 15 [A] \\
            $V_{oc}$ & 70 [V] \\
            $K_i$ & 0,0032 [A/K] \\
            $K_v$ & -0,123 [V/K] \\
            $\alpha$ & 1,3 \\
            $R_s$ & 0,221 [$\Omega$] \\
            $R_p$ & 415,405 [$\Omega$] \\
            $N_s$ & 30 \\
            \hline
        \end{tabular}
    \end{table}
    
    \begin{enumerate}
        \item Condiciones estándar (1000 W/m$^2$ y 25 [ºC])
        \item $G = 1000$ W/m$^2$ y $T = 10$[ºC]
        \item $G = 500$ W/m$^2$ y $T = 25$[ºC]
    \end{enumerate}
    
    Comente los resultados, ¿Tiene sentido lo obtenido?
    %---------------------------------------------------------------------------------------------------------------------------------
    \question 
    Una turbina eólica está acoplada a un generador de inducción trifásico de 560 kW, 50 Hz y 4 polos. La turbina tiene 47 metros de diámetro, una velocidad del viento nominal de 11 [m/s] y una caja de amplificación de velocidad de relación 1:52,6514.
    
    Considere el siguiente coeficiente de desempeño:
    \[
    C_p(\lambda) = 0.0013\lambda^3 - 0.0439\lambda^2 + 0.4083\lambda - 0.6703
    \]
    
    \begin{enumerate}
        \item Grafique el coeficiente de desempeño \( C_p \) en función de \( \lambda \).
        
        \item Suponiendo que la turbina funciona acoplada a un generador de velocidad fija (con deslizamiento de -3\%), grafique la potencia bruta (antes de \( C_p \)) y potencia obtenida desde la turbina (considerando \( C_p \)) para velocidades del viento entre cero y velocidad nominal. ¿Cuál es la velocidad de cut-in para este modo de funcionamiento?
        
        \item Para el caso anterior, grafique la curva de potencia considerando la velocidad de cut-in determinada, una potencia máxima a velocidad nominal y una velocidad cut-out de \( v = 25 \, \text{m/s} \).
    \end{enumerate}
\end{questions}
\newpage
%---------------------------------------------------------------------------------------------------------------------------------

\end{document}