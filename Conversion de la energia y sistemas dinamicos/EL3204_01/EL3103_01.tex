\documentclass[
  11pt,
  letterpaper,
   addpoints,
   answers
  ]{exam}

\usepackage{../exercise-preamble}

\begin{document}

\noindent
\begin{minipage}{0.47\textwidth}
\includegraphics[width=\textwidth]{../fcfm_die}
\end{minipage}
\begin{minipage}{0.53\textwidth}
\begin{center} 
\large\textbf{Análisis de Sistemas Dinámicos y Estimación} (EL3103) \\
\large\textbf{Clase auxiliar 1} \\
\normalsize Prof.~Heraldo Rozas.\\
\normalsize Prof.~Aux.~Erik Saez - Maximiliano Morales
\end{center}
\end{minipage}

\vspace{0.5cm}
\noindent
\vspace{.85cm}

\begin{questions}
    %%%%%%%%%%%%%%%%%%%%%%%%%%%
    \question Considere el sistema de la siguiente figura, donde se tiene un carro atado a un resorte con un sensor de distancia, capaz de medir la distancia del carro a la pared. Suponga que existe una fuerza de fricción viscosa con la superficie $F_f$ de la forma $F_f = b_1 \dot{z} + b_2 \dot{z}^2$.
    \begin{figure}[h!]
        \centering
        \includegraphics[width=0.5\textwidth]{Auxiliar_1_1}
    \end{figure}
    \begin{enumerate}
        \item Establezca hipótesis simplificatorias para el problema.
        \item Formule un modelo matemático del sistema que sea consistente con sus hipótesis.
        \item Encuentre el punto de operación que asegure $z = 1$ m.
    \end{enumerate}
    %%%%%%%%%%%%%%%%%%%%%%%%%%%

    %%%%%%%%%%%%%%%%%%%%%%%%%%%
    \question Considere el siguiente péndulo apoyado en un carro móvil, el cual se desliza por una barra.
    \begin{figure}[h!]
        \centering
        \includegraphics[width=0.5\textwidth]{Auxiliar_1_2}
    \end{figure}
    \begin{enumerate}
        \item Establezca hipótesis simplificatorias.
        \item Formule un modelo matemático, que capture la dinámica del sistema.
        \item Identifique entradas, salidas y estados en su modelo.
        \item Linealice en torno a $\theta = \pi$.
    \end{enumerate}
\end{questions}



\end{document}